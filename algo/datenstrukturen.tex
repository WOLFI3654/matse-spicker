\section{Elementare Datenstrukturen}

% ADTs
\begin{defi}{Abstrakte Datentypen (ADTs)}
    Anforderungen an die Definition eines Datentyps:
    \begin{itemize}
        \item \emph{Spezifikation} eines Datentyps unabhängig von der Implementierung
        \item Reduzierung der von außen sichtbaren Aspekte auf die \emph{Schnittstelle} des Datentyps
    \end{itemize}

    Daraus entstehen \textbf{zwei Prinzipien}:
    \begin{itemize}
        \item \emph{Kapselung:}
              \subitem Zu einem ADT gehört eine Schnittstelle.
              \subitem Zugriffe auf den ADT erfolgen ausschließlich über die Schnittstelle.
        \item \emph{Geheimnisprinzip:}
              \subitem Interne Realisierung eines ADT-Moduls bleibt verborgen.
    \end{itemize}
\end{defi}

\begin{defi}{Homogene Datenstruktur}
    In einer \emph{homogenen Datenstruktur} haben alle Komponenten den \emph{gleichen} Datentyp.
\end{defi}

\begin{defi}{Heterogene Datenstruktur}
    In einer \emph{heterogenen Datenstruktur} haben die Komponenten \emph{unterschiedliche} Datentypen.
\end{defi}

\begin{bonus}{ADTs in Java}
    Viele wichtige abstrakte Datentypen werden in Java durch \emph{Interfaces} beschrieben.

    Es gibt ein oder mehrere Implementierungen dieser Interfaces mit unterschiedlichen dahinter stehenden Konzepten.

    In Java: Package \texttt{java.util}

    Wichtig in der Vorlesung:

    \begin{tabular}{l|l|l}
        ADT                     & Grund-ADT/Interface & Java-Klassen                      \\
        \hline
        Feld                    &                     & (Felder), HashMap                 \\
        Liste                   & List                & ArrayList, LinkedList             \\
        Menge                   & Set                 & HashSet, TreeSet                  \\
        Prioritätswarteschlange &                     & PriorityQueue                     \\
        Stack                   & List                &                                   \\
        Queue                   & List                &                                   \\
        Deque                   & List                & Deque (Interface), ArrayDeque     \\
        Map                     & Set                 & Map (Interface), HashMap, TreeMap \\
        BidiMap                 & Map                 & BidiMap, BiMap (Interface)        \\
        MultiSet, Bag           & Map                 & Bag, Multiset (Interface)
    \end{tabular}
\end{bonus}

\subsection{Abstrakte Datentypen}

\begin{defi}{Array}
    Ein \emph{Array} hat folgende spezielle Eigenschaften:
    \begin{itemize}
        \item Feste Anzahl an Datenobjekten
        \item Auf jedes Objekt kann direkt lesend oder schreibend zugegriffen werden
    \end{itemize}

    \begin{center}
        \begin{tikzpicture}[
                %  -{Stealth[length = 2.5pt]},
                start chain,
                node distance = 0pt,
                StackBlock/.style={draw, minimum width=2em, minimum height=2em, outer sep=0pt, on chain},
            ]
            { start chain = going right
                \node [StackBlock, label=$0$] (0) {};
                \node [StackBlock, label=$1$] (1) {};
                \node [StackBlock, label=$2$] (2) {};
                \node [StackBlock, label=$3$] (3) {};
                \node [StackBlock, label=$\ldots$] (d1) {};
                \node [StackBlock, label=$k$] (k) {};
                \node [StackBlock, label=$\ldots$] (d2) {};
                \node [StackBlock, label=$n-1$] (n1) {};
                \node [StackBlock, label=$n$] (n) {};
            }
        \end{tikzpicture}
    \end{center}

    \textbf{Performance:}

    \begin{center}
        \begin{tabular}{c|c|c|c|c}
            Zugriff     & Suche       & Einf./Lösch. (Anfang) & Einf./Lösch. (Ende) & Einf./Lösch. (Mitte) \\
            \hline
            $\Theta(1)$ & $\Theta(n)$ & -                     & -                   & -                    \\
        \end{tabular}
    \end{center}
\end{defi}

\begin{defi}{Liste}
    Eine Liste besteht aus Elementen, die wie in einem Array linear angeordnet sind.
    Auf die Elemente einer Liste muss nicht wahlfrei zugegriffen werden können\footnote{Eine Ausnahme stellt der Listenanfang dar.}.

    Die Größe einer Liste ist nicht von Anfang an bekannt und sie kann beliebig verlängert bzw. verkürzt werden.
\end{defi}

\begin{defi}{Stack}
    \begin{itemize}
        \item Daten können an einem Ende hinzugefügt oder entnommen werden.
    \end{itemize}

    \begin{center}
        \begin{tikzpicture}[
                %  -{Stealth[length = 2.5pt]},
                start chain,
                node distance = 0pt,
                StackBlock/.style={draw, minimum width=2em, minimum height=2em, outer sep=0pt, on chain},
            ]
            { start chain = going right
                \node [StackBlock,draw=none] (0) {};

                \node [StackBlock,xshift=4em] (1) {};
                \node [StackBlock] (2) {};
                \node [StackBlock] (3) {};
                \node [StackBlock] (4) {};
                \node [StackBlock] (5) {};

                \node [StackBlock,xshift=4em,draw=none] (6) {};

                \draw[->] ([yshift=0.5em, xshift=0.25em] 5.east) [out=0, in=180] to ([yshift=0.5em, xshift=-0.25em] 6.west);
                \draw[->] ([yshift=-0.5em, xshift=-0.25em] 6.west) [out=180, in=0] to ([yshift=-0.5em, xshift=0.25em] 5.east);
            }
        \end{tikzpicture}
    \end{center}
\end{defi}

\begin{defi}{Queue}
    \begin{itemize}
        \item Daten können an einem Ende hinzugefügt und am anderen Ende entnommen werden.
    \end{itemize}

    \begin{center}
        \begin{tikzpicture}[
                %  -{Stealth[length = 2.5pt]},
                start chain,
                node distance = 0pt,
                StackBlock/.style={draw, minimum width=2em, minimum height=2em, outer sep=0pt, on chain},
            ]
            { start chain = going right
                \node [StackBlock,draw=none] (0) {};

                \node [StackBlock,xshift=4em] (1) {};
                \node [StackBlock] (2) {};
                \node [StackBlock] (3) {};
                \node [StackBlock] (4) {};
                \node [StackBlock] (5) {};

                \node [StackBlock,xshift=4em,draw=none] (6) {};

                \draw[<-] ([xshift=0.25em] 0.east) [out=0, in=180] to ([xshift=-0.25em] 1.west);
                \draw[->] ([xshift=-0.25em] 6.west) [out=180, in=0] to ([xshift=0.25em] 5.east);
            }
        \end{tikzpicture}
    \end{center}
\end{defi}

\begin{defi}{Deque (\glqq Double ended queue\grqq)}
    \begin{itemize}
        \item Daten können an beiden Enden hinzugefügt und entnommen werden.
    \end{itemize}

    \begin{center}
        \begin{tikzpicture}[
                %  -{Stealth[length = 2.5pt]},
                start chain,
                node distance = 0pt,
                StackBlock/.style={draw, minimum width=2em, minimum height=2em, outer sep=0pt, on chain},
            ]
            { start chain = going right
                \node [StackBlock,draw=none] (0) {};

                \node [StackBlock,xshift=4em] (1) {};
                \node [StackBlock] (2) {};
                \node [StackBlock] (3) {};
                \node [StackBlock] (4) {};
                \node [StackBlock] (5) {};

                \node [StackBlock,xshift=4em,draw=none] (6) {};

                \draw[->] ([yshift=0.5em, xshift=0.25em] 5.east) [out=0, in=180] to ([yshift=0.5em, xshift=-0.25em] 6.west);
                \draw[->] ([yshift=-0.5em, xshift=-0.25em] 6.west) [out=180, in=0] to ([yshift=-0.5em, xshift=0.25em] 5.east);

                \draw[<-] ([yshift=0.5em, xshift=0.25em] 0.east) [out=0, in=180] to ([yshift=0.5em, xshift=-0.25em] 1.west);
                \draw[<-] ([yshift=-0.5em, xshift=-0.25em] 1.west) [out=180, in=0] to ([yshift=-0.5em, xshift=0.25em] 0.east);
            }
        \end{tikzpicture}
    \end{center}
\end{defi}

\begin{bonus}{Prioritätswarteschlange}
    Eine \emph{Prioritätswarteschlange} ist eine Warteschlange, deren Elemente einen Schlüssel (\emph{Priorität}) besitzen.

    \textbf{Implementierung:}

    In Java dient zur Implementierung die Klasse \texttt{PriorityQueue}, alternativ auch \texttt{TreeSet}.
\end{bonus}

% Mengen
\begin{defi}{Menge}
    Eine \emph{Menge (Set)} ist eine Sammlung von Elementen des gleichen Datentyps.
    Innerhalb der Menge sind die Elemente ungeordnet.
    Jedes Element kann nur einmal in der Menge vorkommen.

    \textbf{Implementierung:}

    In Java ist \emph{Set} ein Interface, das unter anderem folgende Klassen implementiert:
    \begin{itemize}
        \item \texttt{TreeSet}: Basiert auf der Datenstruktur Rot-Schwarz-Baum, implementiert Erweiterung \texttt{SortedMap}.
        \item \texttt{HashSet}: Basiert auf der Datenstruktur Hashtabelle.
    \end{itemize}
\end{defi}

% Assoziative Felder
\begin{defi}{Assoziatives Feld}
    Ein \emph{assoziatives Feld} ist eine Sonderform des Feldes:
    \begin{itemize}
        \item Verwendet keinen numerischen Index zur Adressierung eines Elements.
        \item Verwendet zur Adressierung einen Schlüssel (z.B. \texttt{a["Meier"]}).
    \end{itemize}

    Assoziative Felder eignen sich dazu, Datenelemente in einer großen Datenmenge aufzufinden.
    Jedes Datenelement wird durch einen \emph{eindeutigen Schlüssel} identifiziert.

    \textbf{Implementierung:}

    In Java entspricht ein \emph{assoziatives Feld} dem Interface \texttt{java.util.Map}, das folgende Klassen implementiert:
    \begin{itemize}
        \item \texttt{TreeMap}: Basiert auf der Datenstruktur Rot-Schwarz-Baum, implementiert Erweiterung \texttt{SortedMap}.
        \item \texttt{HashMap}: Basiert auf der Datenstruktur Hashtabelle.
    \end{itemize}
\end{defi}

\subsection{Datenstrukturen}
% Listen
\begin{defi}{Einfach verkettete Liste}
    Im Vergleich zu einem Array kann eine \emph{Liste} schrumpfen und wachsen.

    \begin{center}
        \begin{tikzpicture}[
                %  -{Stealth[length = 2.5pt]},
                start chain,
                node distance = 0pt,
                StackBlock/.style={draw, minimum width=2em, minimum height=2em, outer sep=0pt, on chain},
            ]
            { start chain = going right
                \node [StackBlock, label=$0$] (0) {};
                \node [StackBlock, label=\texttt{next}] (0p) {};
                \node [StackBlock, label=$1$,xshift=2em] (1) {};
                \node [StackBlock, label=\texttt{next}] (1p) {};
                \node [StackBlock, label=$2$,xshift=2em] (2) {};
                \node [StackBlock, label=\texttt{next}] (2p) {};
                \node [StackBlock, label=$\ldots$,xshift=2em] (dots) {};
                \node [StackBlock] (dotsp) {};
                \node [StackBlock, label=$n$,xshift=2em] (n) {};
                \node [StackBlock, label=\texttt{next}] (np) {};
                \node [StackBlock, label=\texttt{null}, xshift=2em] (null) {};


                \draw[->] (0p.center) [out=0, in=180] to (1.west);
                \draw[->] (1p.center) [out=0, in=180] to (2.west);
                \draw[->] (2p.center) [out=0, in=180] to (dots.west);
                \draw[->] (dotsp.center) [out=0, in=180] to (n.west);
                \draw[->] (np.center) [out=0, in=180] to (null.west);
            }
        \end{tikzpicture}
    \end{center}

    \textbf{Performance:}

    \begin{center}
        \begin{tabular}{c|c|c|c|c}
            Zugriff     & Suche       & Einf./Lösch. (Anfang) & Einf./Lösch. (Ende)                                                                                 & Einf./Lösch. (Mitte)   \\
            \hline
            $\Theta(n)$ & $\Theta(n)$ & $\Theta(1)$           & $\Theta(1)/\Theta(n)$\footnote{$\Theta(1)$, wenn das letzte Element bekannt ist, $\Theta(n)$ sonst} & Suchzeit + $\Theta(1)$ \\
        \end{tabular}
    \end{center}
\end{defi}


\begin{defi}{Doppelt verkettete Liste}
    Im Vergleich zu einer einfach verketteten Liste besitzt die \emph{doppelt verkettete Liste} zusätzlich einen Verweis auf den Vorgänger.

    \begin{center}
        \begin{tikzpicture}[
                %  -{Stealth[length = 2.5pt]},
                start chain,
                node distance = 0pt,
                StackBlock/.style={draw, minimum width=2em, minimum height=2em, outer sep=0pt, on chain},
            ]
            { start chain = going right
                \node [StackBlock,label=\texttt{null}] (nulll) {};

                \node [StackBlock,label={[label distance=-.35ex]above:\texttt{prev}}, xshift=2em] (0p) {};
                \node [StackBlock,label=$0$] (0) {};
                \node [StackBlock,label=\texttt{next}] (0n) {};
                \node [StackBlock,label={[label distance=-.35ex]above:\texttt{prev}},xshift=2em] (1p) {};
                \node [StackBlock,label=$1$] (1) {};
                \node [StackBlock,label=\texttt{next}] (1n) {};
                \node [StackBlock,xshift=2em] (dotsp) {};
                \node [StackBlock,label=$\ldots$] (dots) {};
                \node [StackBlock] (dotsn) {};
                \node [StackBlock,label={[label distance=-.35ex]above:\texttt{prev}},xshift=2em] (np) {};
                \node [StackBlock,label=$n$] (n) {};
                \node [StackBlock,label=\texttt{next}] (nn) {};

                \node [StackBlock,label=\texttt{null}, xshift=2em] (nullr) {};

                %\draw[->] ([yshift=0.5em, xshift=0.25em] 5.east) [out=0, in=180] to ([yshift=0.5em, xshift=-0.25em] 6.west);

                % next arrows
                \draw[->] ([yshift=0.5em] 0n.center) [out=0, in=180] to ([yshift=0.5em] 1p.west);
                \draw[->] ([yshift=0.5em] 1n.center) [out=0, in=180] to ([yshift=0.5em] dotsp.west);
                \draw[->] ([yshift=0.5em] dotsn.center) [out=0, in=180] to ([yshift=0.5em] np.west);

                \draw[->] (nn.center) [out=0, in=180] to (nullr.west);

                % prev arrows
                \draw[<-] ([yshift=-0.5em] 0n.east) [out=0, in=180] to ([yshift=-0.5em] 1p.center);
                \draw[<-] ([yshift=-0.5em] 1n.east) [out=0, in=180] to ([yshift=-0.5em] dotsp.center);
                \draw[<-] ([yshift=-0.5em] dotsn.east) [out=0, in=180] to ([yshift=-0.5em] np.center);

                \draw[->] (0p.center) [out=180, in=0] to (nulll.east);
            }
        \end{tikzpicture}
    \end{center}

    \textbf{Performance:}

    \begin{center}
        \begin{tabular}{c|c|c|c|c}
            Zugriff     & Suche       & Einf./Lösch. (Anfang) & Einf./Lösch. (Ende) & Einf./Lösch. (Mitte)   \\
            \hline
            $\Theta(n)$ & $\Theta(n)$ & $\Theta(1)$           & $\Theta(1)$         & Suchzeit + $\Theta(1)$ \\
        \end{tabular}
    \end{center}
\end{defi}

% Dynamische Felder
\begin{defi}{Dynamisches Feld}
    Ein \emph{dynamisches Feld} besteht aus:
    \begin{itemize}
        \item Einem normalen Feld, das nicht vollständig gefüllt ist.
        \item Einem Zeiger, der anzeigt, welches das erste unbesetzte Element ist.
    \end{itemize}

    \begin{center}
        \begin{tikzpicture}[
            %  -{Stealth[length = 2.5pt]},
            start chain,
            node distance = 0pt,
            StackBlock/.style={draw, minimum width=2em, minimum height=2em, outer sep=0pt, on chain},
            ]
            { start chain = going right
            \node [StackBlock, fill=red!50, label=$0$] (0) {};
            \node [StackBlock, fill=red!50, label=$1$] (1) {};
            \node [StackBlock, fill=red!50, label=$2$] (2) {};
            \node [StackBlock, fill=red!50, label=$3$] (3) {};
            \node [StackBlock, fill=red!50, label=$\ldots$] (d1) {};
            \node [StackBlock, label=$k$] (k) {};
            \node [StackBlock, label=$\ldots$] (d2) {};
            \node [StackBlock, label=$n-1$] (n1) {};
            \node [StackBlock, label=$n$] (n) {};

            { [continue chain = going below]
            \chainin (k);
            \node[StackBlock,yshift=-1em] (pointer) {$k$};
            \draw[->] (pointer.north) [out=90, in=-90] to (k.south);
            }

            }
            %\begin{scope}[-{Stealth[length = 2.5pt]}]
            %\draw (1.north) [out=25, in=155] to (2.north);
            %\draw (1.north) [out=30, in=155] to (3.north);
            %\draw (1.north) [out=35, in=155] to (4.north);
            %\draw (6.north) [out=40, in=155] to (6.north);
            %\end{scope}
            %\draw[decorate,decoration={brace, amplitude=10pt, raise=5pt, mirror}]
            %(2.south west) to node[black,midway,below= 15pt] {$k$-elements} (7.south east);%

        \end{tikzpicture}
    \end{center}

    \textbf{Performance:}

    \begin{center}
        \begin{tabular}{c|c|c|c|c}
            Zugriff     & Suche       & Einf./Lösch. (Anfang) & Einf./Lösch. (Ende)                                                                                     & Einf./Lösch. (Mitte) \\
            \hline
            $\Theta(1)$ & $\Theta(n)$ & $\Theta(n)$           & $\Theta(1)/\Theta(n)$\footnote{Wenn das Feld schon voll ist, muss der komplette Inhalt kopiert werden.} & $\Theta(n)$          \\
        \end{tabular}
    \end{center}

    Damit ist ein dynamisches Feld gut für einen \emph{Stack} geeignet!
\end{defi}

% Zirkuläre dynamische Felder
\begin{defi}{Zirkuläres (dynamisches) Feld}
    Ein \emph{zirkuläres Feld} besitzt einen Speicher fester Größe.
    Dabei speichern zwei Zeiger jeweils den Anfang (\texttt{head}) des Speichers, bzw. auf die nächste freie Speicheradresse (\texttt{tail}) im Speicher.

    Wird ein Element am Anfang \glqq abgearbeitet\grqq, bewegt sich \texttt{head} eine Position weiter.
    Wird ein Element am Ende eingefügt, bewegt sich \texttt{tail} eine Position weiter.
    \begin{center}
        \begin{tikzpicture}[
            %  -{Stealth[length = 2.5pt]},
            start chain,
            node distance = 0pt,
            StackBlock/.style={draw, minimum width=2em, minimum height=2em, outer sep=0pt, on chain},
            ]
            { start chain = going right
            \node [StackBlock,label=above:$15$] (15) {};
            \node [StackBlock,label=above:$0$, fill=red!50] (0) {};
            \node [StackBlock,label=above:$1$, fill=red!50] (1) {};
            \node [StackBlock,label=above:$2$, fill=red!50] (2) {};
            \node [StackBlock,label=above:$3$, fill=red!50] (3) {};

            { [continue chain = going below]
            \chainin (3);
            \node [StackBlock,label=right:$4$, fill=red!50] (4) {};
            \node [StackBlock,label=right:$5$] (5) {};
            \node [StackBlock,label=below:$6$] (6) {};
            }

            { [continue chain = going left]
            \chainin (6);
            \node [StackBlock,label=below:$7$] (7) {};
            \node [StackBlock,label=below:$8$] (8) {};
            \node [StackBlock,label=below:$9$] (9) {};
            \node [StackBlock,label=below:$10$] (10) {};
            \node [StackBlock,label=below:$11$] (11) {};
            }

            { [continue chain = going above]
            \chainin (11);
            \node [StackBlock,label=left:$12$] (12) {};
            \node [StackBlock,label=left:$13$] (13) {};
            \node [StackBlock,label=above:$14$] (14) {};
            }


            { [continue chain = going above]
            \chainin (0);
            \node[StackBlock,yshift=2em,xshift=-0.5em,label=\texttt{head}] (head) {$0$};
            \draw[->] (head.south) [out=-90, in=90] to (0.north west);
            }

            { [continue chain = going right]
            \chainin (5);
            \node[StackBlock,yshift=0.5em,xshift=2em,label=above:\texttt{tail}] (tail) {$9$};
            \draw[->] (tail.west) [out=180, in=0] to (5.north east);
            }
            }
            %\begin{scope}[-{Stealth[length = 2.5pt]}]
            %\draw (1.north) [out=25, in=155] to (2.north);
            %\draw (1.north) [out=30, in=155] to (3.north);
            %\draw (1.north) [out=35, in=155] to (4.north);
            %\draw (6.north) [out=40, in=155] to (6.north);
            %\end{scope}
            %\draw[decorate,decoration={brace, amplitude=10pt, raise=5pt, mirror}]
            %(2.south west) to node[black,midway,below= 15pt] {$k$-elements} (7.south east);%

        \end{tikzpicture}
    \end{center}

    \textbf{Performance:} (dynamisch, bei unterliegender Datenstruktur Array)

    \begin{center}
        \begin{tabular}{c|c|c|c|c}
            Zugriff     & Suche       & Einf./Lösch. (Anfang)                                                                                   & Einf./Lösch. (Ende)                              & Einf./Lösch. (Mitte) \\
            \hline
            $\Theta(1)$ & $\Theta(n)$ & $\Theta(1)/\Theta(n)$\footnote{Wenn das Feld schon voll ist, muss der komplette Inhalt kopiert werden.} & $\Theta(1)/\Theta(n)$\footnote{Siehe Fußnote a.} & $\Theta(n)$          \\
        \end{tabular}
    \end{center}

    Damit ist ein zirkuläres (dynamisches) Feld gut für eine \emph{Queue/Deque} geeignet!
\end{defi}

% Verkettete Listen

\subsection{Hashing}

% Hashtabellen

\begin{defi}{Hashfunktion}
    Eine Hashfunktion oder Streuwertfunktion ist eine Abbildung $h : S \to I$, die eine große Eingabemenge, die Schlüssel $S$, auf eine kleinere Zielmenge, die Hashwerte $I$, abbildet.\footnote{Eine Hashfunktion ist daher im Allgemeinen nicht injektiv, aber surjektiv.}

    Die Bildmenge $h(S) \subseteq I$ bezeichnet die Menge der \emph{Hash-Indizes}.


    \begin{center}
        \begin{tikzpicture}[
            %  -{Stealth[length = 2.5pt]},
            start chain = going {right=of \tikzchainprevious.north east},
            KeyBlock/.style={minimum width=4em, minimum height=2em, outer sep=0pt, on chain},
            HashBlock/.style={minimum width=2em, minimum height=2em, outer sep=0pt, on chain},
            FunctionBlock/.style={minimum width=10em, minimum height=26em, outer sep=0pt, on chain, very thick,fill=blue!20},
            every node/.style={draw, label distance=0.5em},
            every on chain/.style={anchor=north west},
            node distance=4em
            ]
            {
            \node [KeyBlock, label=above:Schlüssel] (k0) {Jürgen};
            \node [FunctionBlock, label=above:Hashfunktion] (fun) {};
            \node [HashBlock, label=above:Hashwerte] (h00) {00};

            { [continue chain = going {below=of \tikzchainprevious.south west}, node distance=2em]
            \chainin (k0);
            \node [KeyBlock] (k1) {Felix};
            \node [KeyBlock] (k2) {Finn};
            \node [KeyBlock] (k3) {Tim};
            \node [KeyBlock] (k4) {Benno};
            \node [KeyBlock] (k5) {Lukas};
            \node [KeyBlock] (k6) {Julia};
            }

            { [continue chain = going {below=of \tikzchainprevious.south west}, node distance=0pt]
            \chainin (h00);
            \node [HashBlock] (h01) {01};
            \node [HashBlock] (h02) {02};
            \node [HashBlock] (h03) {03};
            \node [HashBlock] (h04) {04};
            \node [HashBlock] (h05) {05};
            \node [HashBlock] (h06) {06};
            \node [HashBlock] (h07) {07};
            \node [HashBlock] (h08) {08};
            \node [HashBlock] (h09) {09};
            \node [HashBlock] (h10) {10};
            \node [HashBlock] (hdots) {$\ldots$};
            \node [HashBlock] (h15) {15};
            }

            \draw[->] (k0.east) -- ++(2, 0) -- ($(h05.west)-(2,0)$) -- (h05.west);
            \draw[->] (k1.east) -- ++(2, 0) -- ($(h09.west)-(2,0)$) -- (h09.west);
            \draw[->] (k2.east) -- ++(2, 0) -- ($(h00.west)-(2,0)$) -- (h00.west);
            \draw[->] (k3.east) -- ++(2, 0) -- ($(h15.west)-(2,0)$) -- (h15.west);
            \draw[->] (k4.east) -- ++(2, 0) -- ($(h02.west)-(2,0)$) -- (h02.west);
            \draw[->] (k5.east) -- ++(2, 0) -- ($(h10.west)-(2,0)$) -- (h10.west);
            \draw[->] (k6.east) -- ++(2, 0) -- ($(h03.west)-(2,0)$) -- (h03.west);
            }
        \end{tikzpicture}
    \end{center}
\end{defi}

\begin{example}{Divisions-Hash}
    Die \emph{Divisionsrest-Methode (Divisions-Hash)} um Integer zu hashen wird definiert durch:
    $$
        h(x) = x \operatorname{mod} N
    $$

    Sie wird bevorzugt, wenn die Schlüsselverteilung nicht bekannt ist.
    Etwaige Regelmäßigkeiten in der Schlüsselverteilung sollte sich nicht in der Adressverteilung auswirken. Daher sollte $N$ eine Primzahl sein.
\end{example}

\begin{example}{Hashfunktionen für verschiedene Datentypen}
    \begin{itemize}
        \item Alle Datenypen: Verwenden der Speicheradresse
        \item Strings: ASCII/Unicode-Werte addieren (evtl. von einigen Buchstaben, evtl. gewichtet)
    \end{itemize}
\end{example}

\begin{defi}{Kollision}
    Sei $S$ eine Schlüsselmenge und $h$ eine Hashfunktion.
    Ist
    $$
        s_1, s_2 \in S, \ s_1 \neq s_2 : h(s_1) = h(s_2)
    $$
    so spricht man von einer \emph{Kollision}.

    Die Wahrscheinlichkeit von Kollisionen ist abhängig von der gewählten Hashfunktion.
    Hashfunktionen sollten also möglichst gut \emph{streuen}, aber dennoch effizient berechenbar sein.

    \begin{center}
        \begin{tikzpicture}[
            %  -{Stealth[length = 2.5pt]},
            start chain = going {right=of \tikzchainprevious.north east},
            KeyBlock/.style={minimum width=4em, minimum height=2em, outer sep=0pt, on chain},
            HashBlock/.style={minimum width=2em, minimum height=2em, outer sep=0pt, on chain},
            FunctionBlock/.style={minimum width=10em, minimum height=26em, outer sep=0pt, on chain,fill=blue!20},
            every node/.style={draw, label distance=0.5em},
            every on chain/.style={anchor=north west},
            node distance=4em
            ]
            {
            \node [KeyBlock, label=above:Schlüssel, very thick] (k0) {Jürgen};
            \node [FunctionBlock, label=above:Hashfunktion] (fun) {};
            \node [HashBlock, label=above:Hashwerte] (h00) {00};

            { [continue chain = going {below=of \tikzchainprevious.south west}, node distance=2em]
            \chainin (k0);
            \node [KeyBlock] (k1) {Felix};
            \node [KeyBlock] (k2) {Finn};
            \node [KeyBlock] (k3) {Tim};
            \node [KeyBlock] (k4) {Benno};
            \node [KeyBlock] (k5) {Lukas};
            \node [KeyBlock, very thick] (k6) {Julia};
            }

            { [continue chain = going {below=of \tikzchainprevious.south west}, node distance=0pt]
            \chainin (h00);
            \node [HashBlock] (h01) {01};
            \node [HashBlock] (h02) {02};
            \node [HashBlock] (h03) {03};
            \node [HashBlock] (h04) {04};
            \node [HashBlock, very thick] (h05) {05};
            \node [HashBlock] (h06) {06};
            \node [HashBlock] (h07) {07};
            \node [HashBlock] (h08) {08};
            \node [HashBlock] (h09) {09};
            \node [HashBlock] (h10) {10};
            \node [HashBlock] (hdots) {$\ldots$};
            \node [HashBlock] (h15) {15};
            }

            \draw[->, very thick, color=red] (k0.east) -- ++(2, 0) -- ($(h05.west)-(2,0)$) -- (h05.west);
            \draw[->] (k1.east) -- ++(2, 0) -- ($(h09.west)-(2,0)$) -- (h09.west);
            \draw[->] (k2.east) -- ++(2, 0) -- ($(h00.west)-(2,0)$) -- (h00.west);
            \draw[->] (k3.east) -- ++(2, 0) -- ($(h15.west)-(2,0)$) -- (h15.west);
            \draw[->] (k4.east) -- ++(2, 0) -- ($(h02.west)-(2,0)$) -- (h02.west);
            \draw[->] (k5.east) -- ++(2, 0) -- ($(h10.west)-(2,0)$) -- (h10.west);
            \draw[->, very thick, color=red] (k6.east) -- ++(2, 0) -- ($(h05.west)-(2,0)$) -- (h05.west);
            }
        \end{tikzpicture}
    \end{center}

\end{defi}

\begin{defi}{Kollisionsbehandlung}
    Um Kollisionen zu handhaben, existieren verschiedene Strategien:
    \begin{itemize}
        \item \emph{Hashing mit Verkettung}
              \begin{itemize}
                  \item Hashtabelle besteht aus $N$ linearen Listen
                  \item $h(s)$ gibt dann an, in welche Teilliste der Datensatz gehört
                  \item Daten werden innerhalb der Teillisten sequentiell gespeichert
              \end{itemize}
        \item \emph{Hashing mit offener Adressierung}
              \begin{itemize}
                  \item Suchen einer alternativen Position innerhalb des Feldes
                        \begin{enumerate}
                            \item Lineares Sondieren (Verschiebung um konstantes Intervall)
                            \item Doppeltes Hashing (Nutzen einer weiteren Hashfunktion)
                            \item Quadratisches Sondieren (Intervall wird quadriert)
                        \end{enumerate}
              \end{itemize}
    \end{itemize}
\end{defi}

\begin{defi}{Schrittzahl}
    Die \emph{Schrittzahl} $S(s)$, die nötig ist, um den Datensatz mit Schlüssel $s$ zu speichern bzw. wiederzufinden, setzt sich z.B. beim Hashing mit Verkettung zusammen aus:
    \begin{itemize}
        \item der Berechnung der Hash-Funktion und
        \item dem Aufwand für die Suche bzw. Speicherung innerhalb der Teilliste.
    \end{itemize}
\end{defi}

\begin{defi}{Füllgrad}
    Der \emph{Füllgrad} einer Hashtabelle ist der Quotient
    $$
        \alpha = \frac{n}{N}
    $$
    mit
    \begin{itemize}
        \item $N$ als Größe der Hashtabelle
        \item $n$ als Anzahl der gespeicherten Datensätze
    \end{itemize}
\end{defi}

\begin{example}{Schrittzahl beim Suchen in Teillisten}
    Bei idealer Speicherung entfallen $\alpha$ Elemente auf jede Teilliste.
    Dabei gilt:
    \begin{itemize}
        \item erfolgreiche Suche: $c_1 + c_2 \cdot \frac{\alpha}{2}$
        \item erfolglose Suche: $c_1 + c_2 \cdot \alpha$
    \end{itemize}
    Damit ist der Suchaufwand in $\bigo(\alpha) = \bigo(\frac{n}{N})$.

    Wird der Füllgrad $\alpha$ zu groß, sollte die Hashtabelle vergrößert werden.
\end{example}

\begin{defi}{Dynamisches Hashing}
    Um viele Kollisionen zu vermeiden, muss die Hashtabelle ab einem gewissen Füllgrad vergrößert werden.\footnote{nach Sedgewick sollte stets $\alpha < 0.5$ gelten}

    Als Folge muss die gesamte Hashtabelle aber auch neu aufgebaut werden.
\end{defi}

\begin{defi}{Offene Adressierung (Sondieren)}
    Beim Speichern wird bei \emph{Hashing mit offener Adressieren (Sondierung)} so lang ein neuer Hashindex berechnet, bis dort ein freier Speicherplatz vorhanden ist.

    Das Suchen funktioniert analog, allerdings ist das Löschen sehr aufwändig.

    \begin{itemize}
        \item Lineares Sondieren
              \begin{itemize}
                  \item Wird die Ersatzadresse bei jeder Kollision durch Erhöhen der alten Adresse um 1 berechnet, so spricht man von \emph{linearem Sondieren (linear probing)}.
                  \item Die $i$-te Ersatzadresse für einen Schlüssel $s$ mit Hashindex $h(s)$ wird also wie folgt berechnet:
                        $$
                            h_i(s) = (h(s) + i) \operatorname{mod} N
                        $$
              \end{itemize}
        \item Doppeltes Hashing
              \begin{itemize}
                  \item Schlüssel wird nicht um $1$ erhöht, sondern der Inkrement wird mit einer zweiten Hashfunktion berechnet.
                  \item Beseitigt praktisch die Probleme der primären und sekundären Häufung.
                  \item Nicht alle Felder werden durchprobiert. Im ungünstigsten Fall kann eine neues Element nicht eingefügt werden, auch wenn noch Felder frei sind.
              \end{itemize}
    \end{itemize}
\end{defi}

\begin{defi}{Primäre und sekundäre Häufung}
    Bei der \emph{primären Häufung (primary clustering)} ist die Wahrscheinlichkeit, dass Plätze in einem dichtbelegten Bereich eher besetzt werden, deutlich höher.
    Es kommt also zu Kettenbildung.

    Besonders häufig tritt primäre Häufung z.B. beim linearen Sondieren auf.

    \begin{center}
        \begin{tikzpicture}
            [
                %  -{Stealth[length = 2.5pt]},
                start chain,
                node distance = 0pt,
                StackBlock/.style={draw, minimum width=2em, minimum height=2em, outer sep=0pt, on chain},
            ]

            \node [draw, minimum width=2em, minimum height=2em] (val) {$s$};

            { start chain = going right
            \node [StackBlock,right=2cm of val,fill=red!30] (0) {};
            \node [StackBlock] (1) {};
            \node [StackBlock] (2) {};
            \node [StackBlock,fill=red!30] (3) {};
            \node [StackBlock,fill=red!30] (4) {};
            \node [StackBlock,fill=red!30] (5) {};
            \node [StackBlock,fill=red!30] (6) {};
            \node [StackBlock,fill=red!30] (7) {};
            \node [StackBlock] (8) {$s$};
            \node [StackBlock] (9) {};

            \draw[->] (val.south) [out=-30, in=-150] to (4.south);
            \draw[->] (4.south) [out=-45, in=-135] to (5.south);
            \draw[->] (5.south) [out=-45, in=-135] to (6.south);
            \draw[->] (6.south) [out=-45, in=-135] to (7.south);
            \draw[->] (7.south) [out=-45, in=-135] to (8.south);
            }
        \end{tikzpicture}
    \end{center}


    Die \emph{sekundäre Häufung (secondary clustering)} hängt von der Hashfunktion ab.
    Dabei durchlaufen zwei Schlüssel $h(s)$ und $h(s')$ stets dieselbe Sondierungsfolge.
    Sie behindern sich also auf den Ausweichplätzen.

    Besonders häufig tritt sekundäre Häufung z.B. beim quadratischen Sondieren auf.


    \begin{center}
        \begin{tikzpicture}
            [
                %  -{Stealth[length = 2.5pt]},
                start chain,
                node distance = 0pt,
                StackBlock/.style={draw, minimum width=2em, minimum height=2em, outer sep=0pt, on chain},
            ]

            \node [minimum width=2em, minimum height=2em] (val) {};

            { start chain = going right
            \node [StackBlock,right=2cm of val] (0) {};
            \node [StackBlock,fill=red!30] (1) {};
            \node [StackBlock,fill=red!30] (2) {};
            \node [StackBlock,fill=red!30] (3) {};
            \node [StackBlock] (4) {};
            \node [StackBlock,fill=red!30] (5) {};
            \node [StackBlock] (6) {};
            \node [StackBlock] (7) {};
            \node [StackBlock] (8) {$s$};
            \node [StackBlock] (9) {$s'$};

            { [continue chain = going above]
            \chainin (val);
            \node [StackBlock] (val1) {$s$};
            }

            { [continue chain = going below]
            \chainin (val);
            \node [StackBlock] (val2) {$s'$};
            }

            \draw[->] (val1.east) [out=0, in=150] to (2.north);
            \draw[->] (2.north) [out=45, in=135] to (3.north);
            \draw[->] (3.north) [out=45, in=135] to (5.north);
            \draw[->] (5.north) [out=45, in=135] to (8.north);

            \draw[->] (val2.east) [out=0, in=-150] to (2.south);
            \draw[->] (2.south) [out=-45, in=-135] to (3.south);
            \draw[->] (3.south) [out=-45, in=-135] to (5.south);
            \draw[->] (5.south) [out=-45, in=-135] to (8.south);
            \draw[->] (8.south) [out=-45, in=-135] to (9.south);
            }
        \end{tikzpicture}
    \end{center}
\end{defi}
