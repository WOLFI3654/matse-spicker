\section{Grundbegriffe}
\begin{defi}{Eigenschaften eines Algorithmus}
    \begin{itemize}
        \item \emph{Terminierung:} Der Algorithmus bricht nach \emph{endlichen vielen} Schritten ab.
        \item \emph{Determiniertheit:} Bei vorgegebener Eingabe wird ein eindeutiges \emph{Ergebnis} geliefert.
        \item \emph{Determinismus:} Eindeutige Vorgabe der \emph{Abfolge} der auszuführenden Schritte
    \end{itemize}
\end{defi}

\begin{defi}{Landau-Notation}
    Seien $f, g$ reellwertige Funktionen der reellen Zahlen.
    Dann gilt: \cite{wiki:Landau-Symbole}

    \begin{tabular}{l|l|l}
        Notation          & Definition       & Mathematische Definition                                                                                                  \\
        \hline
        $f \in \bigo(g)$  & obere Schranke   & $\exists C > 0 \exists x_0 > 0 \forall x > x_0 : \abs{f(x)} \leq C \cdot \abs{g(x)}$                                      \\
        $f \in \Omega(g)$ & untere Schranke  & $\exists c > 0 \exists x_0 > 0 \forall x > x_0 : c\cdot \abs{g(x)} \leq \abs{f(x)}$                                       \\
        $f \in \Theta(g)$ & scharfe Schranke & $\exists c > 0 \exists C > 0 \exists x_0 > 0 \forall x > x_0 : c\cdot \abs{g(x)} \leq \abs{f(x)} \leq C \cdot \abs{g(x)}$ \\
    \end{tabular}

    Anschaulicher gilt:

    \begin{tabular}{l|l}
        Notation          & Anschauliche Bedeutung                        \\
        \hline
        $f \in \bigo(g)$  & $f$ wächst nicht wesentlich schneller als $g$ \\
        $f \in \Omega(g)$ & $f$ wächst nicht wesentlich langsamer als $g$ \\
        $f \in \Theta(g)$ & $f$ wächst genauso schnell wie $g$
    \end{tabular}
\end{defi}

\begin{example}{Landau-Notation}
    Aus \cite{wiki:Landau-Symbole} :

    \begin{tabular}{l|l}
        Notation                & Beispiel                                          \\
        \hline
        $f \in \bigo(1)$        & Feststellen, ob eine Binärzahl gerade ist         \\
        $f \in \bigo(\log n)$   & Binäre Suche im sortierten Feld mit $n$ Einträgen \\
        $f \in \bigo(\sqrt{n})$ & Anzahl der Divisionen des naiven Primzahltests    \\
        $f \in \bigo(n)$        & Suche im unsortierten Feld mit $n$ Einträgen      \\
        $f \in \bigo(n\log n)$  & Mergesort, Heapsort                               \\
        $f \in \bigo(n^2)$      & Selectionsort                                     \\
        $f \in \bigo(n^m)$      &                                                   \\
        $f \in \bigo(2^{cn})$   & (Backtracking)                                    \\
        $f \in \bigo(n!)$       & Traveling Salesman Problem
    \end{tabular}
\end{example}

\begin{bonus}{Visualisierung Komplexitätsklassen}
    \begin{center}
        \begin{tikzpicture}[scale=1]
            \begin{axis}[
                    %view={45}{15},
                    width=15cm,
                    unit vector ratio*=1 1,
                    axis lines = middle,
                    grid=major,
                    ymin=0,
                    ymax=50,
                    xmin=0,
                    xmax=50,
                    %zmin=-1,
                    %zmax=10,
                    xlabel = $n$,
                    ylabel = $N$,
                    %zlabel = $z$,
                    %xtick style={draw=none},
                    %ytick style={draw=none},
                    %ztick style={draw=none},
                    xtick distance={5},
                    ytick distance={5},
                    %ztick distance={1},
                    %xticklabels=\empty,
                    %yticklabels=\empty,
                    %zticklabels=\empty,
                    disabledatascaling,
                    %stack dir=minus,
                    cycle list name=color list,
                    samples=250,
                    solid,
                    smooth,
                    line width=1.0pt,
                    no markers,
                    legend cell align={left},
                    reverse legend,
                ]

                \addplot +[domain=0:50]{0};         \addlegendentry{$\bigo(1)$};
                \addplot +[domain=0:50]{ln(x)};     \addlegendentry{$\bigo(\log(n))$};
                \addplot +[domain=0:50]{sqrt(x)};   \addlegendentry{$\bigo(\sqrt{n})$};
                \addplot +[domain=0:50]{x};         \addlegendentry{$\bigo(n)$};
                \addplot +[domain=0:50]{x * ln(x)}; \addlegendentry{$\bigo(n \log(n))$};
                \addplot +[domain=0:50]{x^2};       \addlegendentry{$\bigo(n^2)$};
                \addplot +[domain=0:10]{2^x};        \addlegendentry{$\bigo(2^n)$};
                \addplot +[domain=0:5]{facreal(x)}; \addlegendentry{$\bigo(n!)$};
            \end{axis}
        \end{tikzpicture}
    \end{center}
\end{bonus}
