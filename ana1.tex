\documentclass[german]{spicker}

\usepackage{amsmath}

\title{Analysis 1}
\author{Patrick Gustav Blaneck, Felix Racz}

\begin{document}
\maketitle
\tableofcontents
\newpage

\section{Grundlagen}
\subsection{Funktionen}
\subsubsection{Eigenschaften von Funktionen}
\begin{thirdboxl}
    \begin{defi}{Injektivität}
        $f(x) = f(x')\implies x = x'$
    \end{defi}
\end{thirdboxl}%
\begin{thirdboxm}
    \begin{defi}{Surjektivität}
        $\forall y, \exists x: x = f(y)$
    \end{defi}
\end{thirdboxm}%
\begin{thirdboxr}
    \begin{defi}{Bijektivität}
        $\forall y, \exists! x: x = f(y)$
    \end{defi}
\end{thirdboxr}%

\begin{algo}{Beweisen der Injektivität}
    \begin{enumerate}
        \item Behauptung: $f(x) = f(x')$
        \item Umformen auf eine Aussage der Form $x = x'$
    \end{enumerate}
\end{algo}

\begin{algo}{Beweisen der Surjektivität}
    \begin{enumerate}
        \item Aufstellen der Umkehrfunktion
        \item Zeigen, dass diese Umkehrfunktion auf dem gesamten Definitionsbereich definiert ist
    \end{enumerate}
\end{algo}

\begin{algo}{Beweisen der Bijektivität}
    \begin{enumerate}
        \item Injektivität beweisen
        \item Surjektivität beweisen
    \end{enumerate}
\end{algo}

\begin{bonus}{Tipps und Tricks}
    \begin{enumerate}
        \item Gilt eine Eigenschaft nicht, ist ein Gegenbeispiel oft einfach gefunden.
        \item Gilt eine Eigenschaft nicht, ist die Abbildung auch nicht bijektiv.
    \end{enumerate}
\end{bonus}

\newpage
\subsection{Polynome}

\begin{defi}{Polynom}
    Eine Funktion $p(x) = \sum^n_{i=0} a_i x^i$ mit $a_i, x \in \R ~ (\C), a_n \neq 0$ heißt \emph{Polynom vom Grad $n$}.
\end{defi}

\emph{Wichtiges Vorwissen:}
\begin{itemize}
    \item Polynomdivision
    \item Hornerschema
\end{itemize}

\subsubsection{Faktorisierung von Polynomen / Nullstellenberechnung}

\begin{halfboxl}
    \vspace{-\baselineskip}
    \begin{defi}{Abspalten von Linearfaktoren}
        Sei $x_0$ eine Nullstelle eines Polynoms $p(x)$, dann ist
        $$ p(x) = q(x) \cdot (x-x_0).$$
        Dabei ist $(x-x_0)$ ein abgespaltener Linearfaktor und $q(n)$ das entsprechend reduzierte Polynom mit $q(n) = \frac{p(x)}{x-x_0}$.
    \end{defi}
\end{halfboxl}%
\begin{halfboxr}
    \vspace{-\baselineskip}
    \begin{defi}{Faktorisierung}
        Sind $x_1, \ldots, x_n$ Nullstellen eines Polynoms $p(x)$, so ist
        $$ p(x) = a_n \cdot (x-x_1) \cdot \ldots \cdot (x-x_n)$$
        die Faktorisierung von $p(x)$.
    \end{defi}
\end{halfboxr}%

\subsubsection*{Polynome vom Grad 2}
\begin{halfboxl}
    \vspace{-\baselineskip}
    \begin{algo}{$pq$-Formel}
        \begin{enumerate}
            \item Polynom der Form $x^2 + px + q = 0$
            \item $x_{1,2} = -\frac{p}{2} \pm \sqrt{\left(\frac{p}{2}\right)^2 - q}$
        \end{enumerate}
    \end{algo}
\end{halfboxl}%
\begin{halfboxr}
    \vspace{-\baselineskip}
    \begin{algo}{Mitternachtsformel}
        \begin{enumerate}
            \item Polynom der Form $ax^2 + bx + c = 0$
            \item $x_{1,2} = \frac{-b \pm \sqrt{b^2 - 4ac}}{2a}$
        \end{enumerate}
    \end{algo}
\end{halfboxr}

\subsubsection*{Polynome vom Grad $n \geq 3$}
\begin{algo}{Raten einer Nullstelle bei $n = 3$}
    \begin{enumerate}
        \item Polynom der Form $p(x) = ax^3 + bx^2 + cx + d = 0$
        \item Nullstelle $x_1$ stets Teiler von $d$
        \item Einsetzen aller Teiler von $d$ in die Funktion (auch negative!)
        \item Polynomdivision durch Linearfaktor $(x-x_1)$: $\frac{p(x)}{x - x_1}$
        \item Lösen der quadratischen Gleichung
    \end{enumerate}
\end{algo}

\begin{algo}{Substitution bei geraden Exponenten}
    \begin{enumerate}
        \item Polynom der Form $ax^4 + bx^2 + c = 0$
        \item Substituiere $y := x^2$
        \item Lösen der quadratischen Gleichung $ay^2 + by + c = 0$
        \item $x_{1,2} = \pm \sqrt{y_1} \land x_{3,4} = \pm \sqrt{y_2}$
    \end{enumerate}
\end{algo}

\begin{bonus}{Besonderheiten bei $x \in \C$}
    \begin{itemize}
        \item Ist $x_i$ eine Nullstelle des Polynoms $p(x)$ mit \emph{reellen Koeffizienten}, dann ist auch $\overline{x_i}$ eine Nullstelle von $p(x)$.
    \end{itemize}
\end{bonus}

\subsection{Gebrochen rationale Funktionen}

\begin{defi}{Gebrochen rationale Funktionen}
    Seien $p_m(x)$ und $p_n(x)$ Polynome vom Grad $m$ bzw. $n$, dann heißt
    $$
        f(x) = \frac{p_m(x)}{p_n(x)}
    $$
    \emph{gebrochen rationale Funktion}.\\
    Im Fall $m<n$ heißt die Funktion \emph{echt gebrochen rational}, sonst \emph{unecht gebrochen rational}.
\end{defi}

\subsection{Gleichungen und Ungleichungen}

\subsection{Komplexe Zahlen}
\subsubsection{Rechenregeln für komplexe Zahlen in Polarkoordinaten}

\subsubsection{Eigenschaften von $e^{i\pi}$}

\subsubsection{Radizieren von komplexen Zahlen}

\subsubsection{Faktorisierung von Polynomen mit komplexen Koeffizienten}

\section{Folgen und Reihen}

\section{Konvergenz von Folgen, Reihen und Funktionen}

\section{Differentialrechnung}

\section{Integration}

\end{document}