\documentclass[german]{spicker}

\usepackage{amsmath}
\usepackage{polynom}

\title{Analysis 1}
\author{Patrick Gustav Blaneck, Felix Racz}

\begin{document}
\maketitle
\tableofcontents
\newpage

\section{Grundlagen}
\subsection{Funktionen}
\subsubsection{Eigenschaften von Funktionen}
\begin{thirdboxl}
    \begin{defi}{Injektivität}
        $f(x) = f(x')\implies x = x'$
    \end{defi}
\end{thirdboxl}%
\begin{thirdboxm}
    \begin{defi}{Surjektivität}
        $\forall y, \exists x: x = f(y)$
    \end{defi}
\end{thirdboxm}%
\begin{thirdboxr}
    \begin{defi}{Bijektivität}
        $\forall y, \exists! x: x = f(y)$
    \end{defi}
\end{thirdboxr}%

\begin{algo}{Beweisen der Injektivität}
    \begin{enumerate}
        \item Behauptung: $f(x) = f(x')$
        \item Umformen auf eine Aussage der Form $x = x'$
    \end{enumerate}
\end{algo}

\begin{algo}{Beweisen der Surjektivität}
    \begin{enumerate}
        \item Aufstellen der Umkehrfunktion
        \item Zeigen, dass diese Umkehrfunktion auf dem gesamten Definitionsbereich definiert ist
    \end{enumerate}
\end{algo}

\begin{algo}{Beweisen der Bijektivität}
    \begin{enumerate}
        \item Injektivität beweisen
        \item Surjektivität beweisen
    \end{enumerate}
\end{algo}

\begin{bonus}{Tipps und Tricks}
    \begin{itemize}
        \item Gilt eine Eigenschaft nicht, ist ein Gegenbeispiel oft einfach gefunden.
        \item Gilt eine Eigenschaft nicht, ist die Abbildung auch nicht bijektiv.
    \end{itemize}
\end{bonus}

\subsection{Polynome}

\begin{defi}{Polynom}
    Eine Funktion $p(x) = \sum^n_{i=0} a_i x^i$ mit $a_i, x \in \R ~ (\C), a_n \neq 0$ heißt \emph{Polynom vom Grad $n$}.
\end{defi}

\subsubsection{Faktorisierung von Polynomen / Nullstellenberechnung}

\begin{halfboxl}
    \vspace{-\baselineskip}
    \begin{defi}{Abspalten von Linearfaktoren}
        Sei $x_0$ eine Nullstelle eines Polynoms $p(x)$, dann ist
        $$ p(x) = q(x) \cdot (x-x_0).$$
        Dabei ist $(x-x_0)$ ein abgespaltener Linearfaktor und $q(n)$ das entsprechend reduzierte Polynom mit $q(n) = \frac{p(x)}{x-x_0}$.
    \end{defi}
\end{halfboxl}%
\begin{halfboxr}
    \vspace{-\baselineskip}
    \begin{defi}{Faktorisierung}
        Sind $x_1, \ldots, x_n$ Nullstellen eines Polynoms $p(x)$, so ist
        $$ p(x) = a_n \cdot (x-x_1) \cdot \ldots \cdot (x-x_n)$$
        die Faktorisierung von $p(x)$.
    \end{defi}
\end{halfboxr}%

\subsubsection*{Polynome vom Grad 2}
\begin{halfboxl}
    \vspace{-\baselineskip}
    \begin{algo}{$pq$-Formel}
        \begin{enumerate}
            \item Polynom der Form $x^2 + px + q = 0$
            \item $x_{1,2} = -\frac{p}{2} \pm \sqrt{\left(\frac{p}{2}\right)^2 - q}$
        \end{enumerate}
    \end{algo}
\end{halfboxl}%
\begin{halfboxr}
    \vspace{-\baselineskip}
    \begin{algo}{Mitternachtsformel}
        \begin{enumerate}
            \item Polynom der Form $ax^2 + bx + c = 0$
            \item $x_{1,2} = \frac{-b \pm \sqrt{b^2 - 4ac}}{2a}$
        \end{enumerate}
    \end{algo}
\end{halfboxr}

\subsubsection*{Polynome vom Grad $n \geq 3$}
\begin{algo}{Raten einer Nullstelle bei $n = 3$}
    \begin{enumerate}
        \item Polynom der Form $p(x) = ax^3 + bx^2 + cx + d = 0$
        \item Nullstelle $x_1$ stets Teiler von $d$
        \item Einsetzen aller Teiler von $d$ in die Funktion (auch negative!)
        \item Polynomdivision durch Linearfaktor $(x-x_1)$: $\frac{p(x)}{x - x_1}$
        \item Lösen der quadratischen Gleichung
    \end{enumerate}
\end{algo}

\begin{algo}{Substitution bei geraden Exponenten}
    \begin{enumerate}
        \item Polynom der Form $ax^4 + bx^2 + c = 0$
        \item Substituiere $y := x^2$
        \item Lösen der quadratischen Gleichung $ay^2 + by + c = 0$
        \item $x_{1,2} = \pm \sqrt{y_1} \land x_{3,4} = \pm \sqrt{y_2}$
    \end{enumerate}
\end{algo}

\begin{bonus}{Besonderheiten bei $x \in \C$}
    \begin{itemize}
        \item Ist $x_i$ eine Nullstelle des Polynoms $p(x)$ mit \emph{reellen Koeffizienten}, dann ist auch $\overline{x_i}$ eine Nullstelle von $p(x)$.
    \end{itemize}
\end{bonus}

\subsection{Gebrochen rationale Funktionen}

\begin{defi}{Gebrochen rationale Funktionen}
    Seien $p_m(x)$ und $p_n(x)$ Polynome vom Grad $m$ bzw. $n$, dann heißt
    $$
        f(x) = \frac{p_m(x)}{p_n(x)}
    $$
    \emph{gebrochen rationale Funktion}.\\
    Im Fall $m<n$ heißt die Funktion \emph{echt gebrochen rational}, sonst \emph{unecht gebrochen rational}.
\end{defi}

\subsubsection{Polynomdivision}
\begin{algo}{Polynomdivision}
    Gegeben ist \emph{unecht gebrochen rationale Funktion} $f(x) = \frac{p_m(x)}{p_n(x)}$
    \begin{enumerate}
        \item \emph{Dividiere} die größten Exponenten aus beiden Polynomen
        \item \emph{Mutipliziere} Ergebnis mit Divisor zurück
        \item \emph{Subtrahiere} Ergebnis vom Dividenden
        \item Wiederhole, bis:
              \subitem Ergebnis 0 ist, oder
              \subitem Grad des Ergebnisses kleiner ist als Grad des Divisors (ergibt \emph{Rest})
    \end{enumerate}
\end{algo}

\begin{bonus}{Polynomdivision Beispiel}
    \polylongdiv[style=C]{x^3+x^2-1}{x-1}
\end{bonus}

\subsubsection{Hornerschema}
\begin{algo}{Hornerschema}
    Gegeben ist \emph{Polynom} $p_m(x)$ und ein \emph{Wert} $x_0$

    Vorbereitung:
    \begin{itemize}
        \item Erstelle eine Tabelle mit $m + 2$ Spalten und 3 Zeilen
        \item Erste Zelle frei lassen und dann Koeffizienten $a_m, a_{m-1}, \ldots, a_0$ in die erste Zeile schreiben
        \item In die erste Zelle der zweiten Zeile kommt $x_0$
    \end{itemize}

    Anwendung (beginnend in zweiter Zelle der dritten Zeile):
    \begin{enumerate}
        \item Erster Koeffizient der ersten Zeile in die dritte Zeile
        \item \emph{Multipliziere} Zahl der ersten Spalte mit diesem Koeffizienten
        \item Schreibe Ergebnis in zweite Zeile, unterhalb des nächsten Koeffizienten
        \item \emph{Addiere} Ergebnis mit diesem Koeffizienten
        \item Wiederhole 2-4 bis zum Schluss
    \end{enumerate}

    Ergebnis:
    \begin{itemize}
        \item Wert des Polynoms $p_m(x_0)$ in letzter Zelle der letzten Zeile
        \item Bei Wert $p_m(x_0) = 0$ steht in der letzten Zeile das Polynom nach Abspalten des Linearfaktors $(x-x_0)$
    \end{itemize}
\end{algo}

\begin{bonus}{Hornerschema Beispiel}
    Gegeben: $p_4(x) = 2x^4-3x^3+4x^2-5x+2$ an der Stelle $x_0 = 1$

    \polyhornerscheme[x=1]{2x^4-3x^3+4x^2-5x+2}

    Ergebnis: $p_4(1) = 0 \implies (x-1) \text{ ist Linearfaktor von } p_4(x) \text{ und } \frac{p_4(x)}{x-1} = 2x^3 = x^2 + 3x -2$
\end{bonus}

\begin{bonus}{Tipps und Tricks}
    \begin{itemize}
        \item Polynomdivision und Hornerschema funktionieren auch sehr gut mit komplexen Zahlen
        \item Bei mehreren abzuspaltenden Linearfaktoren bietet sich das Hornerschema sehr gut an
    \end{itemize}
\end{bonus}

\subsubsection{Partialbruchzerlegung}

\begin{algo}{Partialbruchzerlegung}
    Gegeben: \emph{Echt gebrochen rationale Funktion}  $f(x) = \frac{p_m(x)}{p_n(x)}$
    \begin{enumerate}
        \item Berechne Nullstellen des \emph{Nennerpolynoms} $x_0, \ldots, x_k \in \R$
        \item Verschiedene Fälle:
              \subitem Relle Nullstellen:
              \subsubitem $x_i$ ist einfache Nullstelle $\implies \frac{A}{x-x_1}$
              \subsubitem $x_i$ ist $r$-fache Nullstelle $\implies  \frac{A_1}{x-x_1} +  \frac{A_2}{(x-x_1)^2} + \ldots + \frac{A_r}{(x-x_1)^r}$
              \subitem Nichtrelle Nullstellen:
              \subsubitem Einfacher quadratischer Term $\implies \frac{Ax + B}{x^2+px+q}$
              \subsubitem $r$-facher quadratischer Term $\implies \frac{A_1x + B_1}{x^2+px+q} + \frac{A_2x + B_2}{(x^2+px+q)^2} + \ldots + \frac{A_rx + B_r}{(x^2+px+q)^r}$
        \item Koeffizientenvergleich:
              \begin{enumerate}
                  \item Brüche gleichnamig machen (Multipliziere beide Seiten mit Nennerpolynom)
                  \item Potenzen von $x$ zusammenfassen
                  \item Gleichungssystem lösen
                  \item Lösungen in Ansatz einsetzen
              \end{enumerate}
    \end{enumerate}
\end{algo}

\begin{bonus}{Besonderheiten in $\C$}
    \begin{itemize}
        \item Für Partialbrüche ohne relle Nullstellen können wir in $\C$ stets Nullstellen finden. Das Verfahren erfolgt dann analog mit komplexen Nullstellen.
    \end{itemize}
\end{bonus}

\begin{bonus}{Tipps und Tricks}
    \begin{itemize}
        \item Partialbruchzerlegung ist erst bei einer \emph{echt gebrochen rationale Funktion} sinnvoll
        \item Ist die Funktion unecht gebrochen rational, führe zuerst eine Polynomdivision durch und zerlege dann den Rest in die Partialbrüche
    \end{itemize}
\end{bonus}

\subsection{Ungleichungen}

\begin{algo}{Berechnen einer Lösungsmenge bei Ungleichungen}
    Gegeben: Ungleichung mit Bezug auf Variable $x$
    \begin{enumerate}
        \item Für jeden Betrag $\left| a(x) \right|$, eine Fallunterscheidung machen für
              \subitem $a(x) \geq 0 \implies \left| a(x) \right| = a(x)$
              \subitem $a(x) < 0 \implies \left| a(x) \right| = -a(x)$
        \item Ungleichungen nach $x$ auflösen
        \item Jeder Fall $i$ erzeugt eine Lösungsmenge $L_i$
        \item Lösungsmenge $L = \bigcup^n_{i = 1} L_i$, wobei $n$ die Anzahl der betrachteten Fälle ist
    \end{enumerate}
\end{algo}

\begin{bonus}{Tipps und Tricks}
    \begin{itemize}
        \item $n$ Beträge in der Gleichung können zu $2^n$ Fällen führen.
        \item Es kann vorkommen, dass ein Fall einer Fallunterscheidung unerreichbar ist, z.B. für $x > 5 \land x < 1$. Die Lösungsmenge $L$ ist dann leer ($L = \emptyset$).
        \item Radizieren (Wurzelziehen) ist in Ungleichungen nur erlaubt, wenn danach der \emph{Betrag} der Wurzel betrachtet wird
        \item Quadrieren einer Ungleichung `erzeugt' potentiell ein falsches Ergebnis. Nach dem Quadrieren sollte man also jedes Ergebnis prüfen.
        \item Multiplikation mit negativen Zahlen sollte vermieden werden, da das Umdrehen des Ungleichheitszeichens schnell für Flüchtigkeitsfehler sorgen kann.
    \end{itemize}
\end{bonus}

\subsection{Komplexe Zahlen}
\subsubsection{Rechenregeln für komplexe Zahlen in Polarkoordinaten}
\todo{To Do}

\subsubsection{Radizieren von komplexen Zahlen}
\todo{To Do}

\section{Folgen und Reihen}

\section{Konvergenz von Folgen, Reihen und Funktionen}

\section{Differentialrechnung}

\section{Integration}

\end{document}