\section{Interpolation}

\subsection{Einleitung}

\begin{defi}{Interpolation}
    Der Begriff \emph{Interpolation} bescheibt eine Klasse von Problemen und Verfahren. 
    
    Zu gegebenen diskreten Daten (z. B. Messwerten) soll eine stetige Funktion (die sogenannte \emph{Interpolante} oder \emph{Interpolierende}) gefunden werden, die diese Daten abbildet. 
    
    Man sagt dann, die Funktion \emph{interpoliert} die Daten. 
\end{defi}

\subsection{Polynominterpolation}

\begin{defi}{Vandermonde-Matrix}
    Die \emph{Vandermonde-Matrix} spielt bei der Interpolation von Funktionen eine wichtige Rolle: 
    
    Wenn an den Stützstellen $(x_0, x_1, \ldots, x_n)$ die Funktionswerte $(y_0, y_1, \ldots, y_n)$ durch ein Polynom $p$ vom Grad $n$ (oder kleiner) interpoliert werden sollen, dann führt der Ansatz 
    \[
        p_n(x) = a_0 + a_1 x^1 + a_2 x^2 + \ldots + a_{n} x^{n}
    \]
    auf das lineare Gleichungssystem 
    \[
        \underbrace{\begin{pmatrix}
            1 & x_0 & \cdots & x_0^n \\ 
            \vdots & \vdots & & \vdots \\ 
            1 & x_n & \cdots & x_n^n 
        \end{pmatrix}}_{V(x_0, \ldots, x_n)}
        \underbrace{\begin{pmatrix}
            a_0 \\ \vdots \\ a_n
        \end{pmatrix}}_{\alpha}
        =
        \underbrace{\begin{pmatrix}
            y_0 \\ \vdots \\ y_n     
        \end{pmatrix}}_{y}
    \]
    mit einer Vandermonde-Matrix $Vx_0, \ldots, x_n)$ als Koeffizientenmatrix.\footnote{Beachte: Für ein Polynom vom Grad $n$ haben wir $n+1$ Koeffizienten!}

    Ist $x_i \neq x_j$ ($\forall i \neq j$) so gibt es genau ein Polynom $p_n(x)$ mit Grad $\leq n$, das $(x_0, y_0), \ldots, (x_n, y_n)$ interpoliert. 
\end{defi}

\begin{defi}{Vandermonde-Determinante}
    Die Determinante der Vandermonde-Matrix wird auch \emph{Vandermonde-Determinante} genannt.
    Sie hat den Wert 
    \[
        \det V(x_0, \ldots, x_n) = \prod_{1 \leq i < j \leq n} (x_j - x_i) 
    \]

    Insbesondere ist die Vandermonde-Matrix genau dann regulär, wenn die $x_i$ paarweise verschieden sind. 
\end{defi}

\begin{defi}{Horner-Schema}
    Um Zwischenstellen eines Polynoms $p_n(x)$ auszuwerten, nutzt man in der Regel nicht direkt die Form 
    \[
        p_n(x) = a_0 + a_1 x + \ldots + a_n x^n    
    \]
    Diese Auswertung ist teuer und durch die vielen Summationen sehr anfällig für Rundungsfehler.

    Besser ist hier das \emph{Horner-Schema}: 
    \begin{itemize}
        \item $p_n(x)$ wird wie folgt umgeformt: 
        \begin{alignat*}{1}
            p_n (x) & = a_0 + a_1 x + a_2 x^2 + \ldots + a_n x^n \\ 
            & = a_0 + x \left( a_1 + a_2 x + \ldots + a_n x^{n-1} \right) \\ 
            & = a_0 + x \left( a_1 + x \left( a_2 + \ldots + a_n x^{n-2} \right) \right) \\ 
            & = \ \ldots \\ 
            & = \underbrace{a_0 + x \underbrace{( a_1 + x \underbrace{( a_2 + x \underbrace{( \ldots + x \underbrace{( a_{n-1} + x \underbrace{a_n}_{q_0} )}_{q_1} )}_{\ldots} )}_{q_{n-2}} )}_{q_n-1}}_{q_n}
        \end{alignat*}
        \item Berechne schrittweise von innen nach außen: 
        \begin{alignat*}{1}
            q_0 & = a_n \\ 
            q_1 & = a_{n-1} + x \cdot q_0 \\
            q_2 & = a_{n-2} + x \cdot q_1 \\
            & \ldots
        \end{alignat*}
        also
        \[ 
            q_0 = a_n, \quad q_k = a_{n-k} + x \cdot q_{k-1}, \quad k = 1, \ldots, n
        \] 
        und damit 
        \[ 
            p_n(x) = q_n
        \] 
    \end{itemize}
\end{defi}

\begin{defi}{Lagrange-Polynom}
    Wir betrachten $x_0, \ldots, x_n$ und definieren das $j$-te \emph{Lagrange-Polynom} als 
    \[ 
    L_j(x) = \frac{x - x_0}{x_j - x_0} \cdot \frac{x - x_{j-1}}{x_j - x_{j-1}} \cdot \frac{x - x_{j+1}}{x_j - x_{j+1}} \cdot \ldots \cdot \frac{x - x_n}{x_j - x_n}
    \] 
    
    $L_j$ ist wohldefiniert, wenn die $x_i$ paarweise verschieden sind und hat Grad $n$.

    Außerdem gilt 
    \[
        L_j(x_i) = 
        \begin{cases}
            1 & \text{für} \ i = j \\ 
            0 & \text{sonst}
        \end{cases}    
    \]

    Mit den Lagrange-Polynomen $L_j$ können wir das Interpolationspolynom einfach ermitteln mit 
    \[
        p_n(x) = \sum_{j = 0}^{n} y_j \cdot L_j(x)    
    \]

    Bei der Benutzung der Lagrange-Polynome muss kein Gleichungssystem gelöst werden, allerdings müssen sämtliche $L_j$ verändert werden, wenn ein weiterer Datenpunkt hinzukommt.
\end{defi}

\begin{example}{Lagrange-Polynom}
    TODO
\end{example}

\begin{defi}{Newton-Interpolationspolynom}

\end{defi}

\subsection{Splines}

\subsection{B-Splines}

\subsection{Interpolierende Kurven}     