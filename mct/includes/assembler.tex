\section{Assembler}

\begin{defi}{Assembler}
    \emph{Assembler} wird genutzt, um zeitkritische Optimierungen zu erzielen.

    Assembler Code besteht im Wesentlichen aus 4 Spalten:
    \begin{itemize}
        \item \emph{Label} bzw. Sprungmarken sind optional und ein Alias für die na der Stelle befindlichen Adresse.
              Sie enden immer mit einem \enquote{:}.
        \item \emph{Mnemonic} ist ein Alias für einen auszuführenden Maschinenbefehl, wobei die Operanden definieren, mit welchen Daten gearbeitet werden.
        \item \emph{Operanden}
        \item \emph{Kommentare} können beliebig platziert werden und starten immer mit \enquote{@}.
    \end{itemize}
\end{defi}

\begin{defi}{Datentypen}
    \begin{lstlisting}[language={[x86masm]Assembler}]
        R0      ; Register
        [ptr]   ; *(ptr), Wert an der Stelle ptr
        #4      ; Dezimalzahl  4
        #0xA    ; Dezimalzahl 10
    \end{lstlisting}

    Diese Typen lassen sich kombinieren.
    z. B.:
    \begin{lstlisting}[language={[x86masm]Assembler}]
        [R1, #4]            ; *(R1 + 4)
        [R1, #4, LSL #2]    ; *(R1 + (4 << 2))
    \end{lstlisting}
\end{defi}

\begin{defi}{Shift-Operationen}
    \begin{lstlisting}[language={[x86masm]Assembler}]
        #4, LSL #2  ; 4 << 2: 100 (4) -> 10000 (16)
        #4, LSR #2  ; 4 >> 2: 100 (4) ->     1  (1)
    \end{lstlisting}
\end{defi}

\begin{defi}{MOV}
    \begin{lstlisting}[language={[x86masm]Assembler}]
        MOV     R0, R1  ; R0 =  R1
        MVN     R0, R1  ; R0 = !R1
    \end{lstlisting}
\end{defi}

\begin{defi}{Datentransfer}
    \begin{lstlisting}[language={[x86masm]Assembler}]
        LDR     R0, [R1]    ; Lade 32 bit Wort an Adresse R1 nach R0
        STR     R0, [R1]    ; Speichere 32 bit Wort R0 an die Adresse R1  
    \end{lstlisting}
\end{defi}

\begin{defi}{Logische Operanden}
    \begin{lstlisting}[language={[x86masm]Assembler}]
        AND     R0, R1, R2  ; R0 = R1 &  R2
        BIC     R0, R1, R2  ; R0 = R1 & !R2
        ORR     R0, R1, R2  ; R0 = R1 |  R2
        ORN     R0, R1, R2  ; R0 = R1 | !R2
        EOR     R0, R1, R2  ; R0 = R1 ^  R2
    \end{lstlisting}
\end{defi}

\begin{defi}{Arithmetische Operationen}
    \begin{lstlisting}[language={[x86masm]Assembler}]
        ADD     R0, R1, <op2>       ; R0 = R1 + <op2>
        SUB     R0, R1, <op2>       ; R0 = R1 - <op2>
        RSB     R0, R1, <op2>       ; R0 = <op2> - R1

        MUL     R0, R1, <op2>       ; R0 = R1 * <op2>
        MLA     R0, R1, R2, <op2>   ; R0 = <op2> + R1 * R2
        MLS     R0, R1, R2, <op2>   ; R0 = <op2> - R1 * R2
        UDIV    R0, R1, <op2>       ; R0 = R1 / <op2> (unsigned)
        SDIV    R0, R1, <op2>       ; R0 = R1 / <op2> (signed)

        CMP     R0, <op2>           ; R0 - <op2>
        CMN     R0, <op2>           ; R0 + <op2>
    \end{lstlisting}
\end{defi}

\begin{defi}{Stack}
    Der \emph{Stack} ist ein \emph{Last-In-First-Out (LIFO)} Speicher.

    Er enthält 32 bit Worte.
    Der \emph{Stack Pointer (SP)} (R13) wird dekrementiert, wenn Daten auf den Stack gelegt werden und inkrementiert, wenn Daten entnommen werden.

    \begin{lstlisting}[language={[x86masm]Assembler}]
        PUSH    <reglist>   ; Register werden auf den Stack gelegt
        POP     <reglist>   ; Register werden von dem Stack entnommen
                            ; <reglist> ist eine Komma-separierte Liste von Registern,
                            ; die auch ganze Bereiche (mit -) enthalten kann.
                            ; SP oder PC duerfen nicht enthalten sein.
    \end{lstlisting}

    Der Stack sollte immer \enquote{ausgeglichen} sein, d. h. eine Funktion sollte gleiche Anzahl von \texttt{PUSH-} und \texttt{POP}-Anweisungen besitzen.
    Keine Stack-Zugriffe außerhalb des vorgesehenen Speicherbereichs.
\end{defi}

\begin{defi}{Sprungbefehle}
    \begin{lstlisting}[language={[x86masm]Assembler}]
        B   label   ; goto Label
        BX  R0      ; goto Adresse in R0
        BL  label   ; goto Label und speicher Ruecksprungadresse (LR)
    \end{lstlisting}
\end{defi}

\begin{bonus}{Methodensprünge}
    Beim Wechseln der Methode werden gesetzte Register der alten auf den Stack gepusht.
    Die Parameter der neuen Methode werden in die Register geschrieben und bearbeitet.

    Wenn die Methode beendet wird, wird der Rückgabewert in das Register \texttt{R0} gespeichert.
    Der Prozessor springt zurück an die Adresse \texttt{LR} und poppt die gespeicherten Register zurück.
\end{bonus}

\begin{bonus}{Assembler Direktiven}
    \emph{Assembler Direktiven} beginnen immer mit einem \enquote{.}.

    Folgende Direktiven erzeugen \emph{keinen} Code:
    \begin{itemize}
        \item \texttt{.thumb}: Übersetze in Thumb/Thumb2 Maschinencode
        \item \texttt{.text}: Lade folgenden Code in die text-Sektion
        \item \texttt{.global main}: Deklariere main als für den Linker sichtbare Funktion
        \item \texttt{.asmfunc}: Start einer Assembler-Unterfunktion
        \item \texttt{.endasmfunc}: Ende einer Assembler-Unterfunktion
        \item \texttt{.align 4}: Erhöhe die aktuelle Adresse auf ein vielfaches von 4
        \item \texttt{.end}: Ende des Assembler Codes
        \item \texttt{.equ}: Definition eines Wertes als symbolischer Namen
    \end{itemize}

    Folgende Direktiven erzeugen Code:
    \begin{itemize}
        \item \texttt{.word}: Lege ein 32 bit Wert an der aktuellen Adresse ab
    \end{itemize}
\end{bonus}

\begin{example}{Assembler}
    Wenn wir P1.0\footnote{Auf dem MSP432 die rote LED} als Output setzen wollen, können wir die Register wie folge manipulieren:

    \begin{lstlisting}[language=c]
        #include msp.h
        #define BIT0 (uint16_t) (0x0001) // 0000 0001

        P1->SEL0 &= ~BIT0 // .... ...0 - Select GPIO mode
        P1->SEL1 &= ~BIT0 // .... ...0 - Select GPIO mode

        P1->DIR |= BIT0   // .... ...1 - Set as output

        P1->OUT ^= BIT0   // .... ...1 - Set P1.0
    \end{lstlisting}

    \begin{lstlisting}[language={[x86masm]Assembler}]
        LDR     R0, P1SEL0
        LDRB    R1, [R0]
        BIC     R1, R1, #1      ; R1 &= !(0000 00001)
        STRB    R1, [R0]        ; P1->SEL0 &= ~BIT0

        LDR     R0, P1SEL1
        LDRB    R1, [R0]
        BIC     R1, R1, #1      ; R1 &= !(0000 00001)
        STRB    R1, [R0]        ; P1->SEL1 &= ~BIT0

        LDR     R0, P1DIR
        LDRB    R1, [R0]
        ORR     R1, R1, #1      ; R1 |= 0000 00001
        STRB    R1, [R0]        ; P1->DIR |= BIT0

        LDR     R0, P1OUT
        LDRB    R1, [R0]
        EOR     R1, R1, #1      ; R1 ^= 0000 00001
        STRB    R1, [R0]        ; P1->OUT ^= BIT0
    \end{lstlisting}
\end{example}