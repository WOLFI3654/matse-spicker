\section{HTML}

\begin{defi}{HTML}
    \emph{HyperText Markup Language (HTML)} ist die Beschreibungssprache des WWWs.

    Hypertext bedeutet, dass die Dokumente sich gegenseitig aufeinander verwiesen.
    Es ist eine universelle, plattformunabhängige Markup-Sprache.

    HTML besteht aus:
    \begin{itemize}
        \item Elementen (text, button, Überschriften etc.)
        \item Strukturen (div, span etc.)
        \item Verweise (Hyperlink)
        \item referenzierte Elemente (Grafiken, Multimedia etc.)
    \end{itemize}
\end{defi}

\begin{defi}{Funktionsweise HTML}
    Auszeichnungselemente sind hierarchisch gegliedert.
    Sie können ineinander Verschachtelt sein, müssen jedoch in einer eindeutigen Eltern-Kind-Beziehung stehen.

    z. B.:

    \begin{lstlisting}[language=HTML5]
        <h1> Pokemon </h1>
        <ul>
            <li> Glumanda </li>
            <li> Pikachu </li>
            <li> ... </li>
        </ul>
    \end{lstlisting}

    \begin{lstlisting}[language=HTML5]
        <div> <h1> Glumanda </h1> </div>  <!-- OK -->
        <div> <h1> Glumanda </div> </h1>  <!-- NOT OK -->
    \end{lstlisting}

    Man kann einen Baum als Repräsentation des Dokumentes nutzen (DOM-Baum).
\end{defi}

\begin{defi}{HTML Tags}
    HTML-Elemente werden durch sogenannte \emph{Tags} markiert.

    z. B.: \texttt{<h1> ... </h1>} oder \texttt{<h1 />} ohne Inhalt

    Tags können mittels Attributen präzisiert werden:

    z. B.: \texttt{<img src="glumanda.png" />}
\end{defi}

\begin{example}{HTML}
    \begin{lstlisting}[language=HTML5]
        <!DOCTYPE html>
        <html>
        <title>
            <meta charset="utf-8">
            <title> Pokemon </title>
        </title>

        <body>
            <h1> pokemon </h1>
            <p> Glumanda </p>
        </body>
        </html>
    \end{lstlisting}
\end{example}

\begin{bonus}{Logisch vs. Physisch}
    HTML wird genutzt um Elemente mit logischen Auszeichnungen voneinander zu unterscheiden:

    \begin{lstlisting}[language=HTML5]
        <h{1-8}> Ueberschrift <h{1-8}>
        <p> Absatz </p>
        <strong> stark betont </strong>
        <em> betont (emphasized) </em>
        <code> Quellcode </code>
        <samp> Beispiel </samp>
        <cite> Zitat </cite>
        <dfn> Definition </dfn>
        <abbr> abgekuerzte Schreibweise </abbr>
    \end{lstlisting}

    \begin{lstlisting}[language=HTML5]
        <ul> <!-- unordered List -->
            <li> ... </li>
        </ul>
    \end{lstlisting}

    \begin{lstlisting}[language=HTML5]
        <ol> <!-- ordered List -->
            <li> ... </li>
        </ol>
    \end{lstlisting}

    \begin{lstlisting}[language=HTML5]
        <table>
            <tr> <th> ID </th> <th> Name </th> </tr>
            <tr> <td> 4 </td> <td> Glumanda </td> </tr>
            <tr> <td> ... </td> <td> ... </td> </tr>
        </table>
    \end{lstlisting}

    HTML kann aber \textbf{soll nicht} genutzt werden, um physische Auszeichnungen zu setzten:

    \begin{lstlisting}[language=HTML5]
        <br> <!-- Zeilenumbruch -->
        <b> fett </b>
        <i> kursiv </i>
        <u> unterstrichen </u>
        <strike> durchgestrichen </strike>
        <big> groesser </big>
        <small> kleiner </small>
        <sup> hochgestellt </sup>
        <sub> tiefgestellt </sub>
    \end{lstlisting}
\end{bonus}

\begin{bonus}{Eingebettete Elemente}
    \begin{lstlisting}[language=HTML5]
        <img src="path/name.{JPEG, GIF, PNG, BMP}" alt="Beschreibung">
        
        <a href="pokemon.html"> Verweistext </a>
        <a href="https://paddel.xyz/pokemon.html"> Ein absoluter Link </a>
        <a href="mailto:pokeverteiler@paddel.xyz"> E-Mail </a>

        <h1 id="h_glumanda"> Glumanda </h1>
        <a href="#h_glumanda"> Link zu Glumanda </a>
    \end{lstlisting}
\end{bonus}

\begin{bonus}{Container}
    Container werden genutzt um Elemente zu gruppieren oder zu stylen.
    Meist sollten aussagekräftigere Alternativen verwendet werden, wenn solche Verfügbar sind.

    \begin{lstlisting}[language=HTML5]
        <div> Generischer block-Container </div>
        <span> Generischer inline-Container </span>
    \end{lstlisting}
\end{bonus}

\begin{bonus}{Formulare}
    Formulare dienen zum Versenden von Informationen an einen Server.
    Die Informationen entstammen meist aus Eingabefeldern.

    \begin{lstlisting}[language=HTML5]
        <form action="path" method="post">
            <label for="name"> Name: </label>
            <input type="text" id="name" name="name">
            <label for="pw"> Passwort: </label>
            <input type="password" id="pw" name="pw">
            <button type="submit">
        </form>
    \end{lstlisting}

    \begin{itemize}
        \item \texttt{text}: Einfaches Eingabefeld
        \item \texttt{password}: Eingabefeld, was die Eingabe verschleiert
        \item \texttt{number}: Eingabefeld für eine Zahl
        \item \texttt{date}: Eingabefeld für ein Datum
        \item \texttt{datetime}: Eingabefeld für ein Datum und eine Uhrzeit
        \item \texttt{email}: Eingabefeld für eine email
        \item \texttt{url}: Eingabefeld für eine URL
        \item \texttt{submit}: Button zum Absenden
        \item \texttt{radio}
        \item \texttt{checkbox}
        \item \texttt{select}: Dropdown Auswahlbox
              \begin{itemize}
                  \item \texttt{option}
              \end{itemize}
        \item \texttt{range}: Slider zur Auswahl in einem festgelegten Bereich
        \item \texttt{file}
        \item \texttt{hidden}: Versteckt
    \end{itemize}
\end{bonus}