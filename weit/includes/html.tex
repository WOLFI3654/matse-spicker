\section{HTML}

\begin{defi}{HTML}
    \emph{HyperText Markup Language (HTML)} ist die Beschreibungssprache des Internets.

    Hypertext bedeutet, dass die Dokumente sich gegenseitig aufeinander verweisen.
    Es ist eine universelle, plattformunabhängige Markup-Sprache.

    HTML besteht aus:
    \begin{itemize}
        \item Elementen (Text, Buttons, Überschriften etc.)
        \item Strukturen (\texttt{div}, \texttt{span} etc.)
        \item Verweise (Hyperlinks)
        \item referenzierte Elemente (Grafiken, Multimedia etc.)
    \end{itemize}
\end{defi}

\begin{defi}{Aufbau HTML}
    Auszeichnungselemente sind hierarchisch gegliedert.
    Sie können ineinander verschachtelt sein, müssen jedoch in einer eindeutigen Eltern-Kind-Beziehung stehen.

    Man kann einen Baum als Repräsentation des Dokumentes nutzen (\emph{DOM-Baum}).
\end{defi}

\begin{example}{Aufbau HTML}
    \begin{lstlisting}[language=html5]
        <h1> Pokemon </h1>
        <ul>
            <li> Glumanda </li>
            <li> Pikachu </li>
            <li> ... </li>
        </ul>
    \end{lstlisting}

    \begin{lstlisting}[language=html5]
        <div> <h1> Glumanda </h1> </div>  <!-- OK -->
        <div> <h1> Glumanda </div> </h1>  <!-- NOT OK -->
    \end{lstlisting}
\end{example}

\begin{defi}{HTML-Tag}
    HTML-Elemente werden durch sogenannte \emph{Tags} markiert, z. B.:

    \begin{lstlisting}[language=html5]
        <h1> ... </h1>
    \end{lstlisting}

    \begin{lstlisting}[language=html5]
        <h1 /> <!-- Ohne Inhalt -->
    \end{lstlisting}

    Tags können mit Attributen präzisiert werden, z. B.:

    \begin{lstlisting}[language=html5]
        <img src="glumanda.png" />
    \end{lstlisting}

    \begin{lstlisting}[language=html5]
        <a href="https://paddel.xyz/pokemon"> Link </a>
    \end{lstlisting}
\end{defi}

\begin{example}{HTML}
    \begin{lstlisting}[language=html5]
        <!DOCTYPE html>
        <html>
        <title>
            <meta charset="utf-8">
            <title> Pokemon </title>
        </title>

        <body>
            <h1> pokemon </h1>
            <p> Glumanda </p>
        </body>
        </html>
    \end{lstlisting}
\end{example}

\begin{bonus}{Logische vs. physische Auszeichnung}
    HTML wird genutzt, um Elemente mit \emph{logischen Auszeichnungen} voneinander zu unterscheiden:

    \begin{lstlisting}[language=html5]
        <h1> Ueberschrift 1 <h1>
        <h2> Ueberschrift 2 <h2>
        <!-- ... -->
        <p> Absatz </p>
        <strong> Stark betont </strong>
        <em> Betont (emphasized) </em>
        <code> Quellcode </code>
        <samp> Beispiel </samp>
        <cite> Zitat </cite>
        <dfn> Definition </dfn>
        <abbr> Abgekuerzte Schreibweise </abbr>
    \end{lstlisting}

    Unordered List:
    \begin{lstlisting}[language=html5]
        <ul>
            <li> ... </li>
        </ul>
    \end{lstlisting}

    Ordered List:
    \begin{lstlisting}[language=html5]
        <ol>
            <li> ... </li>
        </ol>
    \end{lstlisting}

    Tabelle:
    \begin{lstlisting}[language=html5]
        <table>
            <tr> 
                <th> ID </th> 
                <th> Name </th> 
            </tr>
            <tr> 
                <td> 4 </td> 
                <td> Glumanda </td> 
            </tr>
            <tr> 
                <td> ... </td> 
                <td> ... </td> 
            </tr>
        </table>
    \end{lstlisting}

    HTML kann, aber soll nicht genutzt werden, um \emph{physische Auszeichnungen} zu setzen:

    \begin{lstlisting}[language=html5]
        <br> <!-- Zeilenumbruch -->
        <b> fett </b>
        <i> kursiv </i>
        <u> unterstrichen </u>
        <strike> durchgestrichen </strike>
        <big> groesser </big>
        <small> kleiner </small>
        <sup> hochgestellt </sup>
        <sub> tiefgestellt </sub>
    \end{lstlisting}
\end{bonus}

\begin{example}{Eingebettetes Element}
    Bilder:
    \begin{lstlisting}[language=html5]
        <img src="path/to/image.png" alt="Bildbeschreibung">
    \end{lstlisting}

    Links:
    \begin{lstlisting}[language=html5]
        <a href="pokemon.html"> Verweistext </a>
        <a href="https://paddel.xyz/pokemon.html"> Absoluter Link </a>
        <a href="mailto:pokeverteiler@paddel.xyz"> E-Mail </a>
        
        <h1 id="h_glumanda"> Glumanda </h1>
        <a href="#h_glumanda"> Link zu Glumanda </a>
    \end{lstlisting}
\end{example}

\begin{bonus}{Container}
    \emph{Container} werden genutzt, um Elemente zu gruppieren oder zu stylen.

    Meist sollten aussagekräftigere Alternativen verwendet werden, wenn solche verfügbar sind.

    \begin{lstlisting}[language=html5]
        <div> Generischer Block-Container </div>
        <span> Generischer Inline-Container </span>
    \end{lstlisting}
\end{bonus}

\begin{bonus}{Formular}
    \emph{Formulare} dienen zum Versenden von Informationen an einen Server.

    Die Informationen entstammen meist aus Eingabefeldern.

    \begin{lstlisting}[language=html5]
        <form action="path/to/function" method="post">
            <input type="text" id="name" name="name">
            <input type="password" id="pw" name="pw">
            <label for="name"> Name: </label>
            <label for="pw"> Passwort: </label>
            <button type="submit">
        </form>
    \end{lstlisting}

    Formular-Elemente:
    \begin{itemize}
        \item \texttt{text} (einfaches Eingabefeld)
        \item \texttt{password} (Eingabefeld, was die Eingabe verschleiert)
        \item \texttt{number}
        \item \texttt{date}
        \item \texttt{datetime}
        \item \texttt{email}
        \item \texttt{url}
        \item \texttt{submit} (Button zum Absenden)
        \item \texttt{radio} (Einfachauswahl)
        \item \texttt{checkbox} (Mehrfachauswahl)
        \item \texttt{select}, \texttt{option} (Dropdown-Auswahlbox)
        \item \texttt{range} (Slider in festgelegtem Bereich)
        \item \texttt{file}
        \item \texttt{hidden} (Nicht sichtbar)
    \end{itemize}
\end{bonus}