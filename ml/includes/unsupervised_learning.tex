\section{Unsupervised Learning}

\subsection{Dimensionsreduktion}

\begin{defi}{Dimensionsreduktionstechnik}
    \emph{Dimensionsreduktionstechniken} reduzieren die Anzahl der Features (Merkmalen), die für Supervised oder Unsupervised Learning zur Verfügung stehen.
    \footnote{
        Oft mit dem Ziel, wenig Informationen dabei zu verlieren.
    }

    Ein Risiko besteht darin, dass prädikative Informationen für das Supervised Learning verloren gehen können.

    Chancen sind aber:
    \begin{itemize}
        \item Schnellere Berechnungen durch Weglassen irrelevanter Informationen
        \item Ermöglichung von Visualisierung der Daten (und damit der Gewinnung von Einsichten) im Rahmen einer explorativen Analyse (Data Science)
        \item Ermöglichung besserer Generalisierung von Lernmodellen durch Entgegenwirken der Curse of Dimensionality (Fluch der Dimensionalität)
    \end{itemize}
\end{defi}

\begin{bonus}{Curse of Dimensionality}
    Der \emph{Curse of Dimensionality} im Kontext des Machine Learnings das Phänomen, nach dem das Volumen des Feature-Raumes dramatisch zunimmt mit der Anzahl der Dimensionen, so dass die $N$ Datenpunkte (der Trainingsmenge) nur noch spärlich (sparse) im Raum liegen.

    Typische Folgen sind:
    \begin{itemize}
        \item Overfitting-Tendenz steigt
        \item Rechenzeit steigt
        \item Lernmodelle generalisieren schlecht
    \end{itemize}

    Wenn Daten nur noch spärlich den Feature-Raum ausfüllen, dann wird es schwierig, Modelle an die Daten anzupassen, die den Raum adäquat beschreiben
\end{bonus}

\begin{defi}{Principal Component Analysis}
    Die \emph{Hauptkomponentenzerlegung} (\emph{Principal Component Analysis}, PCA) zählt zu den bekanntesten Dimensionsreduktionsverfahren.

    PCA findet neue Achsen (Komponenten) für die Daten, so dass die Daten bezüglich dieser Achsen eine \emph{möglichst große Varianz} aufweisen.

    Achsen, bezüglich derer die Daten kaum (oder keine) Varianz anweisen, können später weggelassen werden (Dimensionsreduktion).

    Neue Achsen werden durch orthogonale Transformation erzeugt, d.h. Vektorlängen und Winkel bleiben erhalten.
    \footnote{
        Präziser: das innere Produkt bleibt erhalten.
    }

    Sei die \emph{Kovarianzmatrix} $S$ gegeben mit:
    \[
        S = \frac{1}{N} \sum_{n=1}^N (\mathbf{x}_n - \conj{\mathbf{x}_n}) (\mathbf{x}_n - \conj{\mathbf{x}_n})^T
    \]

    Dann gilt:
    \begin{itemize}
        \item Eigenwertzerlegung der Kovarianzmatrix $S$ liefert Eigenwerte $\lambda_j$ und -vektoren $\mathbf{u}_j$.
        \item Wir sortieren nach Größe der Eigenwerte und erhalten die Richtungen der PCA über die zu den Eigenwerten assoziierten Eigenvektoren von S.
    \end{itemize}

    PCA kann als Transformation interpretiert werden, die die Achsen des Koordinatensystems so rotiert, dass die Varianz der Projektion der Daten auf der ersten Achse maximiert wird.

    Der Mittelwert der Daten beeinflusst die PCA-Koordinaten:
    \begin{itemize}
        \item Daten mit Schwerpunkt (Mittelwert), der nicht dem Koordinatenursprung entspricht, haben nach PCA-Transformation in den neuen Koordinaten typischerweise Offsets, die meist nicht gut interpretierbar sind.
        \item PCA-Richtungen ändern sich nicht unter Änderung des Schwerpunkts der Daten. Grund: Kovarianzmatrix ist invariant\footnote{Die Kovarianzmatrix verändert sich nicht.} unter Transformation der Mittelwerte.
        \item Manche Software-Implementierungen nutzen für die PCA nicht die Kovarianzmatrix. Dort kann die PCA von den Mittelwerten der Daten abhängen.
    \end{itemize}

    \emph{Jedes Merkmal sollte auf den Mittelwert $0$ zentriert werden}:
    \[
        \tilde{x}_i = x_i - \conj{x}, \forall i
    \]

    Die Skalierung der Daten beeinflusst PCA:
    \begin{itemize}
        \item Die Skalen, in der ein Merkmal angegeben wird beeinflusst die Varianz dieses Merkmals.
        \item Die Ergebnisse der PCA ist abhängig von den Varianzen der Merkmale.
    \end{itemize}

    \emph{Jedes Merkmal sollte auf eine Varianz von $1$ skaliert werden, damit die Ergebnisse der PCA nicht von der (eventuell) willkürlichen Wahl der Skalen der Merkmale abhängig ist}:
    \[
        \tilde{x}_i = \frac{x_i}{\sigma}, \forall i
    \]
\end{defi}

\begin{bonus}{Standardisierung}
    \emph{Standardisierung} bzw. \emph{z-scoring} beschreibt die Transformation eines Merkmals $X$, so dass es Mittelwert $0$ und Varianz $1$ aufweist.
    Das Merkmal wird also \emph{zentriert} und \emph{skaliert}.

    Sei $\conj{X}$ der Mittelwert und $\sigma$ die Standardabweichung von $X$.

    Das \emph{standardisierte Merkmal} bzw. der \emph{z-score} $Z$ wird erzeugt durch:
    \[
        Z = \frac{X - \conj{X}}{\sigma}
    \]

    In der Praxis liegen sind Daten oft in Matrizen organisiert, in denen jede Spalte einem Merkmal entsprechen.
    In diesem Fall standardisiert man jedes Merkmal, indem jede Spalte separat auf Mittelwert 0 und Varianz 1 transformiert wird.
\end{bonus}

\begin{defi}{Proportion of Variance Explained}
    \emph{Proportion of Variance Explained} (PVE) beschreibt den Bruchteil der Gesamtvarianz der Merkmale, die über die PCA-Komponenten repräsentiert wird.

    Die Varianz der $j$-ten Komponente ist:\footnote{Entspricht also dem jeweiligen Eigenwert.}
    \[
        \Var(Z_j) = \Var(\mathbf{u}_j^T \mathbf{x}) = \mathbf{u}_j^T \mathbf{S} \mathbf{u}_j = \mathbf{u}_j^T \lambda_j \mathbf{u}_j = \lambda_j
    \]

    Die Gesamtvarianz über alle Merkmale ist dann:
    \[
        \Var_\text{total} = \sum_{j=1}^D \Var(Z_j) = \sum_{j=1}^D \lambda_j
    \]

    Dann ist die $PVE(j)$ der $j$-ten PCA-Komponente definiert als:
    \[
        \PVE(j) = \frac{\Var(Z_j)}{\Var_\text{total}} = \frac{\lambda_j}{\sum_{j=1}^D \lambda_j}
    \]
\end{defi}

\begin{defi}{Kernel PCA}
    PCA ist oft nicht hilfreich, wenn Daten nichtlinear strukturiert sind.
    Die Idee von \emph{Kernel PCA} ist, dass ähnlich wie bei Support Vector Machines der Kernel Trick angewandt wird.

    Der Vorgang ist wie folgt:
    \begin{enumerate}
        \item Wählen eines Kernels, z.B. Gaußscher RBF Kernel:
              \[
                  K(\mathbf{x}_i, \mathbf{x}_j) = \exp\left( -\gamma \norm{\mathbf{x}_i - \mathbf{x}_j}^2 \right)
              \]
        \item Bestimmen und zentrieren der Kernelmatrix:
              \[
                  \tilde{K} = K - \mathbf{1}_N K - K \mathbf{1}_N + \mathbf{1}_N K \mathbf{1}_N \quad \text{mit} \quad (K)_{ij} = K(\mathbf{x}_i, \mathbf{x}_j), \, (\mathbf{1}_N)_{ij} = \frac{1}{N}
              \]
        \item Lösen des Eigenwertproblems:
              \[
                  \tilde{K} \vec{\alpha}_j = N \lambda_j \vec{\alpha}_j = \tilde{\lambda}_j \vec{\alpha}_j
              \]
        \item Normieren der Eigenvektoren:
              \[
                  \vec{\alpha}_j \to \tilde{\vec{\alpha}}_j = \frac{1}{\sqrt{\tilde{\lambda}_j}} \frac{\vec{\alpha}_j}{\norm{\vec{\alpha}_j}} \quad \forall j
              \]
        \item Projizieren der Daten auf die neue $j$-te kPCA-Achse:
              \[
                  \Phi(\mathbf{x})^T \tilde{\mathbf{u}}_j = \sum_l \tilde{\alpha}_{jl} K(\mathbf{x}, \mathbf{x}_l)
              \]
    \end{enumerate}

    kPCA erlaubt nichtlinearen Dimensionsreduktion!
\end{defi}

% \begin{defi}{Kernel PCA}
%     PCA ist oft nicht hilfreich, wenn Daten nichtlinear strukturiert sind.
%     Die Idee von \emph{Kernel PCA} ist, dass ähnlich wie bei Support Vector Machines der Kernel Trick angewandt wird.

%     Der Vorgang ist wie folgt:
%     \begin{enumerate}
%         \item Umformulieren der PCA mit nichtlinearer Transformation:
%               \begin{itemize}
%                   \item $N$ Datensätze mit $D$ Merkmalen
%                   \item Kovarianzmatrix $S$ wird umformuliert mit $\mathbf{x}_i \to \Phi(\mathbf{x}_i)$. Dann ist $S$:\footnote{Die Differenz mit $\conj{\mathbf{x}}$ bzw. $\conj{\Phi}(\mathbf{x})$ fällt weg, da der Mittelwert $0$ sein wird aufgrund der folgenden Standardisierung.}
%                         \[
%                             S = \frac{1}{N} \sum_{i=1}^N (\Phi(\mathbf{x}_i) - \conj{\Phi}(\mathbf{x})) (\Phi(\mathbf{x}_i) - \conj{\Phi}(\mathbf{x}))^T = \frac{1}{N} \sum_{i=1}^N \Phi(\mathbf{x}_i) \Phi(\mathbf{x}_i)^T
%                         \]
%               \end{itemize}
%         \item Darstellung der Eigenvektoren als Linearkombination der transformierten Features:
%               \begin{itemize}
%                   \item Wir erhalten die Achsen der (klassischen) PCA durch die Eigenvektoren $u_j$ der Kovarianzmatrix. Es gilt:
%                         \[
%                             S \mathbf{u}_j = \lambda_j \mathbf{u}_j \quad \text{und} \quad \mathbf{u}_j = \sum_l \alpha_{jl} \Phi(\mathbf{x}_l)
%                         \]
%                   \item[$\implies$] Das Auffinden der PCA-Richtungen entspricht dem Bestimmen der Koeffizienten.
%               \end{itemize}
%         \item Finden der Eigenvektoren:
%               \begin{itemize}
%                   \item Es gilt ($\tilde{\lambda}_j := N \lambda_j$):
%                         \[
%                             K \vec{\alpha}_j = N \lambda_j \vec{\alpha}_j = \tilde{\lambda}_j \vec{\alpha}_j
%                         \]
%                   \item Die Koeffizienten $\alpha_{jl}$ entsprechen also den Komponenten der Eigenvektoren der Kernel-Matrix und sind dadurch bestimmbar.
%               \end{itemize}
%         \item Normierung der Eigenvektoren:
%               \begin{itemize}
%                   \item Aus der Normierung der Eigenvektoren (bei der klassischen PCA) ergibt sich eine Normierungsbedingung für die Koeffizienten $\alpha_{jl}$:
%                         \[
%                             \vec{\alpha}_j \to \tilde{\vec{\alpha}}_j = \frac{1}{\sqrt{\tilde{\lambda}_j}} \frac{\vec{\alpha}_j}{\norm{\vec{\alpha}_j}} \quad \forall j
%                         \]
%               \end{itemize}
%         \item Projektion auf kPCA Achsen:
%               \begin{itemize}
%                   \item Wir betrachten die Projektion eines beliebigen Featurevektors $\mathbf{x}$ aus dem Featureraum auf die $j$-te Achse der Kernel PCA (kPCA):
%                         \[
%                             \Phi(\mathbf{x})^T \tilde{\mathbf{u}}_j = \sum_l \tilde{\alpha}_{jl} K(\mathbf{x}, \mathbf{x}_l)
%                         \]
%               \end{itemize}
%         \item Zentrieren der Features:
%               \begin{itemize}
%                   \item Normieren der Features:
%                         \[
%                             \tilde{\Phi}(\mathbf{x}_i) = \Phi(\mathbf{x}_i) - \frac{1}{N} \sum_l \Phi(\mathbf{x}_l)
%                         \]
%                   \item Zentrieren der Kernelmatrix (mit $\mathbf{1}_N$ $N \times N$ Matrix, in der jeder Eintrag $\nicefrac{1}{N}$ ist):
%                         \[
%                             \tilde{K} = K - \mathbf{1}_N K - K \mathbf{1}_N + \mathbf{1}_N K \mathbf{1}_N
%                         \]
%               \end{itemize}
%     \end{enumerate}
% \end{defi}

\begin{defi}{Linear Discriminant Analysis}
    Gegeben seien:
    \begin{itemize}
        \item Daten $(\mathbf{x}_i, y_i)$ ($i=1,\ldots,N$), wobei $(\mathbf{x}_i)_j$ das $j$-te Feature des $i$-ten Datenpunktes ist.
        \item Zwei Klassen $\mathcal{C}_1$ und $\mathcal{C}_2$ (Klassifikationsproblem)
        \item $N_k$ Anzahl der Datenpunkte in Klasse $K$
    \end{itemize}

    Ziel der \emph{Linear Discriminant Analysis} (LDA) ist es, eine Richtung $\mathbf{u}_1$ zu finden, die die Klassen $C_1$ und $C_2$ gut trennt.

    Sei $\mathbf{u}_1$ ein Richtungsvektor.
    Die Projektion eines Datenpunktes auf $\mathbf{u}_1$ ist:
    \[
        \tilde{\mathbf{x}}_i = \mathbf{u}_1^T \mathbf{x}_i
    \]

    Sei $\mathbf{m}^{(k)}$ der mittlere Vektor (Mittelpunkt) aller Datenpunkte der Klasse $k$:
    \[
        \tilde{\mathbf{m}}^{(k)} = \frac{1}{N_k} \sum_{n \in \mathcal{C}_k}^N \mathbf{x}_n
    \]
    Seine Projektion ist dann:
    \[
        \tilde{\mathbf{m}}^{(k)} = \mathbf{u}_1^T \tilde{\mathbf{m}}^{(k)}
    \]

    Die Trennbarkeit der Klassen lässt sich dann quantifizieren durch:
    \[
        \Delta \tilde{\mathbf{m}} = \tilde{\mathbf{m}}^{(k)}_1 - \tilde{\mathbf{m}}^{(k)}_2
    \]

    Gesucht werden soll dann die Richtung $\mathbf{u}_1$, die den Abstand zwischen beiden projizierten Mittelpunkten maximiert.
\end{defi}

\subsection{Clusteringverfahren}