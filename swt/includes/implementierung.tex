\section{Implementierung}

\begin{bonus}{Komplexität der Implementierungsphase}
    \begin{itemize}
        \item Kenntnisse über Algorithmen und Datenstrukturen notwendig
        \item Viele Notationen und Progammiersprachen mit unterschiedlichen Ansätzen verfügbar
        \item Viele Werkzeuge zur SDL-Unterstützung
        \item Zunehmen auch Security-Aspekte
        \item Technologie sehr schnelllebig, oft komplex
    \end{itemize}
\end{bonus}

\begin{defi}{Aufgaben der Implementierungsphase}
    \begin{enumerate}
        \item \textbf{Komponenten implementieren}
              \begin{itemize}
                  \item Geeignete Datenstrukturen wählen
                  \item Algorithmen wählen
                  \item in Programmiersprache formulieren bzw. codieren
              \end{itemize}
        \item \textbf{Komponenten dokumentieren}
              \begin{itemize}
                  \item Wie erledigt die Komponente ihre Aufgabe? (Problemlösung)
                  \item Implementierungsentscheidungen begründen
                  \item Angaben zu Zeit- und Speicherkomplexität
              \end{itemize}
        \item \textbf{Komponenten prüfen (vs. Entwurf)}
              \begin{itemize}
                  \item Testumgebung einrichten
                  \item Testdaten erfassen
                  \item Testläufe durchführen
                  \item Verifizieren
              \end{itemize}
    \end{enumerate}
\end{defi}

\begin{defi}{Code Conventions}
    \emph{Code Conventions} geben u.a. Vorhaben zu:
    \begin{itemize}
        \item \textbf{Formatierung}
              \begin{itemize}
                  \item Klammern, Tabs, Blockeinrückung
              \end{itemize}
        \item \textbf{Namenskonventionen}
              \begin{itemize}
                  \item Bezeichner-Konventionen
                  \item Typgebunden
              \end{itemize}
        \item \textbf{Dokumentierung}
              \begin{itemize}
                  \item Inline
                  \item Javadoc
              \end{itemize}
        \item \textbf{Annotationen}
              \begin{itemize}
                  \item Reihenfolge
                  \item Optionale Angaben
              \end{itemize}
        \item \textbf{Ausnahmebehandlung}
              \begin{itemize}
                  \item Handling
                  \item Checked vs. Unchecked
              \end{itemize}
        \item \textbf{Stil}
              \begin{itemize}
                  \item Reihenfolge von Membern, Operationen
              \end{itemize}
    \end{itemize}
\end{defi}

\begin{defi}{Prinzipien der systematischen Programmierung}
    \begin{enumerate}
        \item \emph{Das Prinzip der Verbalisierung}
        \item \emph{Das Prinzip der problemadäquaten Datentypen}
        \item Das Prinzip der Verfeinerung
        \item Das Prinzip der strukturierten Programmierung
        \item Das Prinzip des defensiven Programmieren
        \item Das Prinzip der integrierten Dokumentation
    \end{enumerate}
\end{defi}

\begin{defi}{Das Prinzip der Verbalisierung}
    Das Ziel des \emph{Prinzips der Verbalisierung} ist es, Ideen und Konzepte des Programmieruns im Programm möglichst \emph{deutlich}, \emph{gut sichtbar} zu machen und zu \emph{dokumentieren}.

    \begin{itemize}
        \item \textbf{Regel 1:} Ein geeigneter Name sollte die \emph{semantisch funktionale Rolle} des Bezeichners widerspiegeln.
        \item \textbf{Regel 2:} Ein geeigneter Name sollte \emph{einheitlicher Stil- und Namenskonvention} folgen.
        \item \textbf{Regel 3:} Ein geeigneter Name sollte \emph{leicht zu merken sein}.
    \end{itemize}
\end{defi}

\begin{defi}{Das Prinzip der problemadäquaten Datentypen}
    Das Ziel des \emph{Prinzips der problemadäquaten Datentypen} ist es, dass sich Daten- und Kontrollstrukturen eines Problems möglichst in der programmiersprachlichen Lösung \emph{unverfälscht widerspiegeln}.

    \begin{itemize}
        \item \textbf{Regel 1:} Bereits bestehende Datentypen sollten \emph{wiederverwendet} werden.
        \item \textbf{Regel 2:} Wertebereiche der Datentypen sollten \emph{festgelegt} und \emph{eingeschränkt} werden.
        \item \textbf{Regel 3:} Wenn Regel 1 nicht möglich ist, sollten neue, \emph{benutzerdefinierte} Datentypen erstellt werden.
    \end{itemize}
\end{defi}