\section{Einführung}

\begin{defi}{Datenkommunikation}
    Die \emph{Datenkommunikation} beschäftigt sich mit dem immateriellen Transport digitaler Daten zwischen Endsystemen.

    Hierbei sind alle hierzu benötigten Verfahren und Regeln Bestandteil der Datenkommunikation.

    Vorteile durch Datenkommunikation sind:
    \begin{itemize}
        \item zurückgreifen auf fremde bzw. entfernte Ressourcen und Daten
        \item Kostensenkung durch gemeinsame Nutzung von Betriebsmitteln
        \item Informationsgewinn durch entfernten Zugriff
    \end{itemize}

    Dazu benötigt es:
    \begin{itemize}
        \item Effiziente Methoden zum Datenaustausch zwischen Kommunikationspartnern
        \item Absprachen bzw. Regeln zur gemeinsamen Nutzung der Infrastruktur
        \item Kommunikationsdienste zur Übertragung von Informationen in verteilten Umgebungen
    \end{itemize}
\end{defi}

\begin{defi}{Kommunikationssystem}
    Im engeren Sinn ist ein \emph{Kommunikationssystem} eine Einrichtung bzw. eine Infrastruktur für die Übermittlung von Informationen.

    Kommunikationssysteme stellen dazu Nachrichtenverbindungen zwischen mehreren Endstellen her.
\end{defi}