\section{Sliding Window}

\begin{defi}{Sliding Window}
    Der Begriff \emph{Sliding Window} bezeichnet bei der Datenflusskontrolle in Rechnernetzen ein Fenster, das einem Sender die Übertragung einer bestimmten Menge von Daten ermöglicht, bevor eine Bestätigung zurückerwartet wird.\footnote{Netzwerkprotokolle, die auf Sliding Windows basieren, werden \emph{Sliding-Window-Protokolle} genannt.}

    Das Sliding Window verfolgt das Ziel, die Kapazitäten der Leitung und des Empfängers optimal auszulasten, das heißt so viele Datenpakete (Datenframes) wie möglich zu senden.

    Dabei stellt das Verzögerungs-Bandbreite-Produkt die maximale in der Übertragung befindliche Datenmenge dar, die gesendet werden kann, ohne auf die erste Bestätigung (\texttt{ACK}) zu warten.
\end{defi}

\begin{defi}{Sliding Window (Sender)}
    Beim Schiebefensterverfahren führt der Sender permanent eine Liste von aufeinanderfolgenden Sequenznummern, die der Anzahl der Frames, die er senden darf, entspricht.

    Sobald ein Datenpaket dem Empfänger erfolgreich zugestellt wird, sendet dieser dafür eine Bestätigung (\texttt{ACK}) zurück, die den Sender dazu veranlasst, ein weiteres Frame zu übertragen.

    Falls der Sender innerhalb des Timeouts jedoch kein \texttt{ACK} erhält, versucht er das Frame erneut zu übertragen.
    Unter der Voraussetzung, dass das Verzögerungs-Bandbreiten-Produkt bereits erreicht ist, kann dann aber kein neues Frame übertragen werden, d. h., es kommt zu einem Stau in der Pipe.

    Das Sendefenster verschiebt sich mit jeder eingehenden Bestätigung, indem das bestätigte Frame aus dem Fenster herausfällt und ein neu zu sendendes Frame in das Fenster aufgenommen wird.
    Dadurch enthält das Fenster immer nur unbestätigte Frames.

    Für den Fall, dass Frames während der Übertragung verloren gehen, muss der Sender alle Datenpakete in seinem Speicher halten, um sie erneut übertragen zu können.
\end{defi}

\begin{defi}{Sliding Window (Empfänger)}
    Analog zum Fenster des Senders verwaltet auch der Empfänger ein Schiebefenster.

    Beide Fenster müssen aber nicht unbedingt die gleiche Größe haben, da diese im Laufe der Zeit durch das Senden und Empfangen von Frames variieren kann.

    Die Größe des Sendefensters bestimmt sich durch das vom Empfänger angegebene Maximum sowie durch die Netzbelastung.

    Jede Bestätigung für ein erfolgreich übertragenes Frame enthält einen Wert, der angibt, für welche Menge an weiteren Datenpaketen der Empfänger noch Kapazität frei hat.
\end{defi}

\begin{defi}{Verzögerungs-Bandbreiten-Produkt}
    Das \emph{Verzögerungs-Bandbreiten-Produkt}  ist eine Eigenschaft einer Datenverbindung. Es berechnet sich als das Produkt aus der Verzögerung (Einheit: Sekunde) und der Datenübertragungsrate (Einheit: Bit pro Sekunde).
    Das Ergebnis, eine in Bit oder Byte angegebene Datenmenge entspricht der Anzahl der Daten, die sich auf dem Weg befinden, wenn ein oder mehrere Sender den Übertragungskanal füllen.

    Ein hoher Durchsatz kann nur dann erzielt werden, wenn der Sender hinreichend große Datenmengen versenden kann, bevor dieser innehalten und auf eine Empfangsbestätigung durch den Empfänger warten muss.
    Ist die Menge an gesendeten Daten verglichen mit dem Verzögerungs-Bandbreiten-Produkt gering, dann kommt das Protokoll nicht an die maximale Effizienz heran.

    Der Nutzungsgrad $\rho$ ist wie folgt definiert:
    \[
        \rho := n \cdot \frac{\frac{L}{R}}{\frac{L}{R} + RTT}
    \]

    mit
    \begin{itemize}
        \item $L$: Länge eines Segments in Bits.
        \item $R$: Übertragungsrate des Kanals in Bits pro Sekunde.
        \item $\nicefrac{L}{R}$: Zeit zur Übertragung eines Segments in Sekunden.
        \item RTT: \emph{Rount Trip Time}
        \item $B := R\rho$: erzielbare Bandbreite
        \item $W := n \cdot L$: Fensterbreite in Bits
    \end{itemize}

    Die erzielbare Bandbreite $R_\rho = B$ ist dann:
    \[
        B := R\rho = n \cdot \frac{L}{\frac{L}{R} + RTT}
    \]

    Es gilt:
    \[
        \frac{L}{R} \ll \text{RTT}, \quad B \leq \frac{W}{\text{RTT}}
    \]
\end{defi}

\begin{example}{Sliding Window}
    \emph{Beispiel 1:}

    Die Anwendung möchte eine Rate von 10 Mbps erreichen.
    Wir kennen die Round-Trip-Zeit, die bei 8ms liegt.
    Die Segmente seien 500 Byte groß.

    Wie groß muss $W$ sein?

    Es gilt:
    \[
        W \geq 10 \text{ Mbps} \cdot 0.008 \text{ s} = \num{80000} \text{ Bit} = \num{10000} \text{ Byte}
    \]
    \[
        \implies 20 \text{ Segmente mit } 500 \text{ Byte}
    \]

    \emph{Beispiel 2:}

    Die Anwendung möchte eine Rate von 1 Gbps erreichen.
    Wir kennen die Round-Trip-Zeit, die bei 240ms liegt.
    Die Segmente seien 1 KB groß.

    Wie groß muss $W$ sein?

    Es gilt:
    \[
        W \geq 10 \text{ Gbps} \cdot 0.24 \text{ s} = \num{240000000} \text{ Bit} = \num{30000000} \text{ Byte}
    \]
    \[
        \implies \num{30000} \text{ Segmente mit } 1 \text{ KB}
    \]

\end{example}

\begin{defi}{Ereignisse beim Empfänger}
    \begin{itemize}
        \item Empfang eines Out-of-Order-Segments
              \begin{itemize}
                  \item Eigentlich wurde ein anderes Segment erwartet.
                  \item Je nach Implementierung einfach ignoriert bzw. verworfen.
                  \item Sender überträgt das Segment nach Ablauf des Timeouts erneut.
              \end{itemize}
        \item Empfang eines fehlerbehafteten Segments
              \begin{itemize}
                  \item Keine Reaktion, Sender überträgt das Segment nach Ablauf des Timeouts erneut.
                  \item Falls unterstützt: Verschicken einer \enquote{negativen} Bestätigung \texttt{NAK}.
              \end{itemize}
    \end{itemize}
\end{defi}

\begin{defi}{Ereignisse beim Sender}
    \begin{itemize}
        \item Timeout
        \item Empfang einer nicht erwarteten Bestätigung (\emph{out of order} \texttt{ACK})
        \item Empfang einer negativen Bestätigung \texttt{NAK}
    \end{itemize}

    Die Reaktion ist die Identifikation und das erneute Verschicken des fehlerhaften Segments:
    \begin{itemize}
        \item \emph{Go-Back-N}: Ab dem Fehlerfall alles neu versenden.
        \item \emph{Selective Repeat} bzw. \emph{Selective Reject}: Nur das fehlerhafte Segment neu versenden.
    \end{itemize}
\end{defi}

\begin{bonus}{Unterschied zum Stop-and-Wait-Algorithmus}
    Beim Stop-and-Wait-Algorithmus, der ebenso wie das Schiebefensterverfahren ein ARQ-Protokoll ist, wartet der Sender nach der Übertragung eines Frames auf eine Bestätigung, bevor er das nächste Frame überträgt.
    Kommt innerhalb der Wartezeit keine Bestätigung, überträgt der Sender das Frame erneut.

    Im Gegensatz zu diesem Algorithmus, der dem Sender nur jeweils ein ausstehendes Frame auf der Verbindungsleitung gestattet und dadurch ineffizient ist, kann der Sliding-Window-Algorithmus mehrere Frames gleichzeitig übertragen.

    Der Stop-and-Wait-Algorithmus ist ein Spezialfall des Sliding-Window-Algorithmus.
    Beide funktionieren identisch, wenn die Größe des Sendefensters beim Schiebefenster-Algorithmus auf 1 Frame eingestellt ist.
\end{bonus}