\section{Sliding Window}

\begin{defi}{Sliding Window}
    Der Begriff \emph{Sliding Window} bezeichnet bei der Datenflusskontrolle in Rechnernetzen ein Fenster, das einem Sender die Übertragung einer bestimmten Menge von Daten ermöglicht, bevor eine Bestätigung zurückerwartet wird.

    Netzwerkprotokolle, die auf Sliding Windows basieren, werden \emph{Sliding-Window-Protokolle} genannt.

    Das Sliding Window verfolgt das Ziel, die Kapazitäten der Leitung und des Empfängers optimal auszulasten, das heißt so viele Datenpakete (Datenframes) wie möglich zu senden.

    Dabei stellt das Verzögerungs-Bandbreite-Produkt die maximale in der Übertragung befindliche Datenmenge dar, die gesendet werden kann, ohne auf die erste Bestätigung (\texttt{ACK}) zu warten.
\end{defi}