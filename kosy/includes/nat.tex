\section{NAT}

\begin{bonus}{Konfiguration von IPv4-Adressen}
    \emph{IPv4-Adressen} müssen in jedem Endgerät, Router und sonstigen Netzwerkkomponenten konfiguriert werden.

    Jeder Knoten muss folgende Daten besitzen:
    \begin{itemize}
        \item IP und Netzwerkmaske
        \item Standard-Gateway
        \item DNS-Server
    \end{itemize}

    Grundsätzlich existieren zwei Möglichkeiten dazu:
    \begin{itemize}
        \item Manuelle Konfiguration
        \item Konfiguration über ein Protokoll (z. B. DHCP)
    \end{itemize}
\end{bonus}

\begin{bonus}{Probleme IPv4}
    Wir hatten bereits geklärt, dass IPv4 nur $2^{32}$ IP-Adressen zur Verfügung hat.

    Da bereits relativ zeitnah nach dem Anpassen der Subnetzgrößen klar wurde, dass die gewonnene Anzahl der daraus resultieren IP-Adressen nicht ausreicht, hat man sich eine provisorische Lösung überlegt.

    Diese
\end{bonus}
