\documentclass[german]{spicker}

\addbibresource{sql.bib}

\usepackage{arydshln}

\title{Datenbanken, SQL}
\author{Tim Wende}
\makeindex[intoc]
\makeindex[intoc, name=Beispiele,title=Beispiele]

\begin{document}
\maketitle
\tableofcontents
\newpage

\rowcolors{2}{gray!15}{white}

\section{Selektion, Projektion}

\begin{sql}{USE}
    \texttt{USE} legt ein Default-Schema fest.
    So kann bei Tabellen in SQL-Ausdrücken die explizite Angabe des Schemas entfallen.

    Möchte man also z.B. die Datenbank \texttt{db\_pokemon} auswählen, so nutzt man:

    \begin{minted}{mysql}
USE db_pokemon;
    \end{minted}
\end{sql}

\begin{defi}{Projektion}
    Eine \emph{Projektion} wählt Spalten durch Angabe von Attributen der Tabelle.
\end{defi}

\begin{defi}{Selektion}
    Eine \emph{Selektion} filtert Zeilen durch Angabe eines Kriteriums, welches je Zeile erfüllt sein muss.
\end{defi}

\begin{sql}{SELECT}
    \texttt{SELECT} gibt alle Datensätze einer Tabelle aus.

    Möchte man sich z.B. alle Pokémon aus der Tabelle \texttt{pokemon} anzeigen lassen, so nutzt man:

    \mint{mysql}|SELECT * FROM pokemon;|

    \lstinputlisting[style=BashOutputStyle]{includes/code/out_code_select_all.txt}

    Möchte man eine \emph{Projektion} auf die Ergebnisse der Abfrage anwenden, sich also nur bestimmte Spalten anzeigen lassen, kann man dies im \texttt{SELECT} spezifizieren\footnote{Der \texttt{*} dient als Platzhalter für alle Attribute}:

    \mint{mysql}|SELECT ID, Name, PrimaerTyp, SekundaerTyp FROM pokemon;|

    \lstinputlisting[style=BashOutputStyle]{includes/code/out_code_select_projektion.txt}

    Die Reihenfolge der Datensätzer ist hierbei nicht vorgegeben.
\end{sql}

\begin{sql}{WHERE}
    Möchte man eine \emph{Selektion} auf die Ergebnisse der Abfrage anwenden, also für jedes Element eine Bedingung vorraussetzen, nutzt man \texttt{WHERE}.

    Beispielsweise werden im folgendem Befehl alle Pokémon mit dem Typ Feuer ausgegeben:

    \mint{mysql}|SELECT * FROM pokemon WHERE PrimaerTyp = 'Feuer';|

    \lstinputlisting[style=BashOutputStyle]{includes/code/out_code_select_where.txt}
\end{sql}

\begin{sql}{AS}
    Um komplexere Ausdrücke zu kürzen oder Zweideutigkeiten aufzulösen, kann man Tabellen mit einem \emph{Alias} versehen.

    Beispielsweise wäre im folgenden Befehl die Tabelle \texttt{pokemon} unter \texttt{p} erreichbar:

    \mint{mysql}|SELECT p.* FROM pokemon AS p;|

    Das Schlüsselwort \texttt{AS} ist optional, d.h folgende Abfrage ist äquivalent:
    
    \mint{mysql}|SELECT p.* FROM pokemon p;|
\end{sql}

\begin{sql}{ORDER BY}
    \texttt{ORDER BY} sortiert die Ergebnismenge nach Kriterien auf- (\texttt{ASC}) oder absteigend (\texttt{DESC}).
    
    Sollten mehrere Kriterien angegeben sein, so wird die Menge nach dem ersten gegebenen Kriterium sortiert.
    Wenn nach diesem Schritt mehrere Einträge dieses Kriterium gleichermaßen erfüllen, wird diese Untermenge nach dem zweiten Kriterium sortiert, usw.

    \texttt{NULL} ist immer der erste Eintrag.
    
    Wenn man sich also alle Pokémon nach Namen sortiert ausgeben möchte, so erhält man\footnote{ASC ist dabei der \texttt{Default} Wert}:

    \begin{minted}{mysql}
SELECT * FROM pokemon ORDER BY Name;
SELECT * FROM pokemon ORDER BY Name ASC;
    \end{minted}

    \lstinputlisting[style=BashOutputStyle]{includes/code/out_bonus_select_order_by_name.txt}

    Wenn man alternativ alle Pokémon erst nach Primärtyp (\texttt{ASC}), dann nach Sekundärtyp (\texttt{DESC}) sortiert ausgeben möchte, erhält man:

    \mint{mysql}|SELECT * FROM pokemon ORDER BY PrimaerTyp ASC, SekundaerTyp DESC;|

    \lstinputlisting[style=BashOutputStyle]{includes/code/out_bonus_select_order_by_typ.txt}
\end{sql}

\subsection{Funktionen}

\begin{sql}{CONCAT}
    \texttt{CONCAT} kombiniert verschiedene Strings zu einem.

    In folgendem Beispiel werden Primärtyp und Sekundärtyp zu einem Eintrag zusammengefasst:

    \begin{minted}{mysql}
SELECT
    ID, Name,
    CONCAT('[', PrimaerTyp, ', ', SekundaerTyp, ']') AS 'Typ'
FROM pokemon
    WHERE SekundaerTyp IS NOT NULL;
    \end{minted}

    \lstinputlisting[style=BashOutputStyle]{includes/code/out_bonus_select_concat.txt}
\end{sql}

\begin{sql}{ROUND}
    \texttt{ROUND} rundet numerische Werte.
    Der zweite Parameter gibt die gewünschte Anzahl an Nachkommastellen an.

    In folgendem Beispiel wird das Gewicht eines Pokémon auf ganzzahlige Werte gerundet:

    \begin{minted}{mysql}
SELECT
    ID, Name, Gewicht,
    ROUND(Gewicht, 0)
FROM pokemon;
    \end{minted}

    \lstinputlisting[style=BashOutputStyle]{includes/code/out_bonus_select_round.txt}
\end{sql}

\begin{sql}{COUNT}
    \texttt{COUNT} zählt alle nicht-\texttt{NULL} Datensätze.
    Sollte \texttt{*} angegeben werden, beachtet er alle Entitäten, welche mindestens ein nicht-\texttt{NULL} Attribut besitzen.

    Dudurch, dass \texttt{ID} und \texttt{PrimaerTyp} nicht \texttt{NULL} sein können, erhält man mit folgenden äquivalenten Befehlen die Anzahl aller Pokémon:

    \begin{minted}{mysql}
SELECT COUNT(*) FROM pokemon;
SELECT COUNT(ID) FROM pokemon;
SELECT COUNT(PrimaerTyp) FROM pokemon;
    \end{minted}

    \lstinputlisting[style=BashOutputStyle]{includes/code/out_bonus_select_count_all.txt}

    Möchten wir nun alle Pokémon mit einem Sekundärtypen zählen, so erhalten wir: 

    \mint{mysql}|SELECT COUNT(SekundaerTyp) FROM pokemon;|

    \lstinputlisting[style=BashOutputStyle]{includes/code/out_bonus_select_count.txt}
\end{sql}

\subsection{Operatoren}

\begin{sql}{IS NULL}
    \texttt{IS NULL} bzw. \texttt{IS NOT NULL} werden genutzt, um mit \texttt{NULL}-Einträgen zu arbeiten.

    In der folgenden Abfrage werden Pokémon mit exakt einem Typen zurückgegeben:

    \begin{minted}{mysql}
SELECT
    ID, Name, PrimaerTyp, SekundaerTyp
FROM pokemon
    WHERE SekundaerTyp IS NULL;
    \end{minted}

    \lstinputlisting[style=BashOutputStyle]{includes/code/out_bonus_select_null.txt}
\end{sql}

\begin{sql}{COALESCE}
    \texttt{COALESCE} gibt den ersten nicht-\texttt{NULL} Eintrag aus der Parameterliste zurück.
    Dementsprechend sind folgende SQL-Befehle äquivalent:

    \begin{minted}{mysql}
SELECT 'Ausgabe' AS 'Beispiel';
SELECT COALESCE(NULL, NULL, ..., NULL, 'Ausgabe') AS 'Beispiel';
    \end{minted}

    \lstinputlisting[style=BashOutputStyle]{includes/code/out_bonus_select_coalesce_ausgabe.txt}

    In diesem Beispiel wird aus der Tabelle \texttt{entwicklung} jede \texttt{NULL}-Tageszeit mit dem String \texttt{Immer} ersetzt:

    \begin{minted}{mysql}
SELECT
    Von, Zu, Tageszeit,
    COALESCE(Tageszeit, 'Immer') AS Entwicklungszeitraum
FROM entwicklung;
    \end{minted}

    \lstinputlisting[style=BashOutputStyle]{includes/code/out_bonus_select_coalesce.txt}
\end{sql}

\begin{sql}{Bedingungen}
    \emph{Bedingungen} werden von Selektionen, sowie von \texttt{IF}-Abfragen genutzt:

    \begin{minted}{mysql}
SELECT * FROM <Tabelle> WHERE <Bedingung>;
SELECT IF(<Bedingung>, <Wert wenn TRUE>, <Wert wenn FALSE>);
    \end{minted}

    Verschiedene Bedingungen lassen sich mit \texttt{AND} und \texttt{OR} verbinden, sowie mit \texttt{NOT} negieren.

    Operatoren, welche für die meisten Datentypen definiert sind:

    \begin{itemize}
        \item \texttt{=} überprüft auf exakte Gleichheit.
        \item \texttt{IN} überprüft, ob der Wert in einer angegebenen Liste zu finden ist.
    \end{itemize}

    Explizit für String Werte ist definiert:

    \begin{itemize}
        \item \texttt{LIKE}: Benutzt reguläre Ausdrücke zum Vergleichen:
            \begin{itemize}
                \item \texttt{\%} lässt beliebig viele beliebige Zeichen zu.
                \item \texttt{\_} lässt exakt ein beliebiges Zeichen zu.
            \end{itemize}
    \end{itemize}

    Des Weiteren kann man für numerische Werte folgende Operatoren nutzen:

    \begin{itemize}
        \item \texttt{$<$, $<=$, $>$, $>=$}
        \item \texttt{BETWEEN X AND Y} überprüft, ob der Wert zwischen den angegebenen Grenzen \texttt{X} und \texttt{Y} (beide inklusive) liegt.
    \end{itemize}

    Achtung: Ausdrücke sind \texttt{NULL}, wenn einer der Operanden \texttt{NULL} ist.
\end{sql}

\begin{sql}{DISTINCT}
    \texttt{DISTINCT} berücksichtigt keine doppelten Daten.

    Wenn man sich jede vorhandene Kombination an Typen anzeigen lassen möchte, so erhält man in folgender Abfrage doppelte Werte:

    \mint{mysql}|SELECT PrimaerTyp, SekundaerTyp FROM pokemon;|

    \lstinputlisting[style=BashOutputStyle]{includes/code/out_bonus_select_distinct_all.txt}

    Diese entfallen, wenn man das Schlüsselwort \texttt{DISTINCT} ergänzt:

    \mint{mysql}|SELECT DISTINCT PrimaerTyp, SekundaerTyp FROM pokemon;|

    \lstinputlisting[style=BashOutputStyle]{includes/code/out_bonus_select_distinct.txt}
\end{sql}

\subsection{Beispiele}

\begin{example}{SELECT}
    Ermitteln Sie, wieviele verschiedene Typkombinationen in der Tabelle \texttt{pokemon} vorhanden sind und geben Sie die Anzahl aus.

    \exampleseparator

    \begin{minted}{mysql}
SELECT
    COUNT(
        DISTINCT PrimaerTyp,
        COALESCE(SekundaerTyp, 'NULL')
    )
FROM pokemon;
    \end{minted}

    \lstinputlisting[style=BashOutputStyle]{includes/code/out_example_select_typ.txt}
\end{example}

\begin{example}{SELECT}
    Ermitteln Sie den BMI\footnote{$\text{BMI} = \frac{\text{Gewicht}}{\text{Größe}^2}$} aller Pokémon, welche mehr als 50kg wiegen.
    
    Runden Sie das Ergebnis auf 2 Nachkommastellen.

    \exampleseparator

    \begin{minted}{mysql}
SELECT
    ID, Name, Gewicht, Groesse,
    ROUND(Gewicht / (Groesse * Groesse), 2) AS BMI
FROM pokemon
    WHERE Gewicht > 50
    ORDER BY BMI DESC;
    \end{minted}

    \lstinputlisting[style=BashOutputStyle]{includes/code/out_example_select_bmi.txt}
\end{example}

\begin{example}{SELECT}
    Wieviele Pokémon-namen beginnen mit einem \texttt{P}?
    
    Und wieviele enthalten ein \texttt{P}?
    
    Ermitteln Sie die jeweilige Anzahl.

    \exampleseparator

    \mint{mysql}|SELECT COUNT(*) FROM pokemon WHERE Name LIKE 'P%';|

    \lstinputlisting[style=BashOutputStyle]{includes/code/out_example_select_count_p_start.txt}

    \mint{mysql}|SELECT COUNT(*) FROM pokemon WHERE Name LIKE '%P%';|

    \lstinputlisting[style=BashOutputStyle]{includes/code/out_example_select_count_p_in.txt}
\end{example}
\section{Subselects}

\begin{defi}{Unterabfrage}
    \emph{Unterabfragen} bzw. Subselects sind verschachtelte Abfragen.
    Dabei stellen die inneren Abfragen Daten bereit, welche von den Äußeren genutzt werden.
    
    Unterabfragen werden genutzt, wenn die zu vergleichenden Werte sich dynamisch verändern oder noch unbekannt sind.

    Unterabfragen lassen sich in verschiedene Kategorien aufteilen:

    \begin{itemize}
        \item Einzeilige bzw. mehrzeilige Unterabfragen
        \item Korrelierte bzw. unkorrelierte Unterabfragen
        \item Skalare bzw. nicht skalare Unterabfragen
    \end{itemize}

    Wenn man von Unterabfragen spricht, so meint man meist nicht korrelierte, skalare Unterabfragen, da diese die simpelste Implementierung sind.
\end{defi}

\begin{example}{Unterabfrage}
    So könnte man z.B. alle \texttt{ID}s von Pokémon, welche sich aus Glumanda entwickeln, folgendermaßen herausfinden:

    Zuerst müssen wir die ID von Glumanda herausfinden:

    \begin{lstlisting}[language=mysql]
        -- ID von 'Glumanda' herausfinden
        SELECT ID
        FROM pokemon
        WHERE Name = 'Glumanda';
    \end{lstlisting}

    \setcounter{rownum}{0}
    \begin{tabular}{I|I}
        \rowcolor{gray!35}
        & \multicolumn{1}{T}{ID} \\\hline
        1 & 4 \\
    \end{tabular}

    Diese nutzen wir nun manuell um alle Entwicklungen herauszufinden:

    \begin{lstlisting}[language=mysql]
        -- ID von 'Glumanda' nutzen um alle Entwicklungen herauszufinden
        SELECT *
        FROM entwicklung
        WHERE Von = 4;
    \end{lstlisting}

    \setcounter{rownum}{0}
    \begin{tabular}{I|IIIITTT}
        \rowcolor{gray!35}
        & \multicolumn{1}{T}{Von} & \multicolumn{1}{T}{Zu} & \multicolumn{1}{T}{Level} & \multicolumn{1}{T}{Item} & \multicolumn{1}{T}{GetragenesItem} & \multicolumn{1}{T}{Tageszeit} \\\hline
        1 & 4 & 5 & 16 & NULL & NULL & NULL \\
    \end{tabular}

    Diese beiden Abfragen lassen sich nun zu einer Abfrage zusammenfassen:

    \begin{lstlisting}[language=mysql]
        -- ID von 'Glumanda' nutzen um alle Entwicklungen herauszufinden
        SELECT *
        FROM entwicklung
        WHERE Von = (
            -- ID von 'Glumanda' herausfinden
            SELECT ID
            FROM pokemon
            WHERE Name = 'Glumanda'
        );
    \end{lstlisting}
\end{example}

\begin{defi}{Mehrzeilige Unterfrage}
    In \emph{mehrzeiligen Unterabfragen} geben die inneren Abfragen mehrere Entitäten zurück.
    
    So kann man z.B. nicht auf exakte Gleichheit überprüfen, sondern muss Mengenoperationen (wie z.B. \texttt{IN}) nutzen.
\end{defi}

\begin{example}{Unterabfrage (mehrzeilig)}
    Das Resultat des ersten Beispiels nutzen wir nun um den Namen der Entwicklung herauszufinden:

    \begin{lstlisting}[language=mysql]
        SELECT *
        FROM pokemon
        WHERE ID = (
            -- Eine Entitaet
            SELECT Zu
            FROM entwicklung
            WHERE Von = (
                SELECT ID
                FROM pokemon
                WHERE Name = 'Glumanda'
            )
        );
    \end{lstlisting}

    \setcounter{rownum}{0}
    \begin{tabular}{I|ITIIITT}
        \rowcolor{gray!35}
        & \multicolumn{1}{T}{ID} & \multicolumn{1}{T}{Name} & \multicolumn{1}{T}{Groesse} & \multicolumn{1}{T}{Gewicht} & \multicolumn{1}{T}{Generation} & \multicolumn{1}{T}{PrimaerTyp} & \multicolumn{1}{T}{SekundaerTyp} \\\hline
        1 & 5 & Glutexo & 1.1 & 19 & 1 & Feuer & NULL \\
    \end{tabular}

    Sollten wir nun alternativ die Entwicklungen von Evoli anzeigen lassen, können wir die Bedingung \texttt{=} nicht benutzen, da die Abfrage aus \texttt{entwicklung} eine Menge an Entitäten zurück gibt.
    Dementsprechend muss man das bekannte \texttt{IN} verwenden.

    \begin{lstlisting}[language=mysql]
        SELECT *
        FROM pokemon
        WHERE ID IN (
            -- Eine Menge an Entitaeten
            SELECT Zu
            FROM entwicklung
            WHERE Von = (
                SELECT ID
                FROM pokemon
                WHERE Name = 'Evoli'
            )
        );
    \end{lstlisting}

    \setcounter{rownum}{0}
    \begin{tabular}{I|ITIIITT}
        \rowcolor{gray!35}
        & \multicolumn{1}{T}{ID} & \multicolumn{1}{T}{Name} & \multicolumn{1}{T}{Groesse} & \multicolumn{1}{T}{Gewicht} & \multicolumn{1}{T}{Generation} & \multicolumn{1}{T}{PrimaerTyp} & \multicolumn{1}{T}{SekundaerTyp} \\\hline
        1 & 134 & Aquana & 1 & 29 & 1 & Wasser & NULL \\
        2 & 135 & Blitza & 0.8 & 24.5 & 1 & Elektro & NULL \\
        3 & 136 & Flamara & 0.9 & 25 & 1 & Feuer & NULL \\
        4 & 196 & Psiana & 0.9 & 26.5 & 2 & Psycho & NULL \\
        5 & 197 & Nachtara & 1 & 27 & 2 & Unlicht & NULL \\
        6 & 470 & Folipurba & 1 & 25.5 & 4 & Pflanze & NULL \\
        7 & 471 & Glaziola & 0.8 & 25.9 & 4 & Eis & NULL \\
        8 & 700 & Feelinara & 1 & 23.5 & 6 & Fee & NULL \\
    \end{tabular}
\end{example}

\begin{defi}{Korrelierte Unterabfrage}
    In \emph{korrelierten Unterabfragen} (\enquote{Correlated Subselects}) verweist die innere Abfragen auf die Äußere.\footnote{Dies ist vergleichbar mit einer Variable, welche in einem inneren Scope erneut genutzt wird.}

    Da eine Bedingung für jede Entität neu überprüft werden muss, haben korrelierte Unterabfragen unter Umständen eine längere Laufzeit.

    Bei nicht korrelierten Unterabfragen wird eine Bedingung zu Beginn der Abfrage überprüft.
    Dementsprechend wird hier die Laufzeit optimiert.
\end{defi}

\begin{example}{Unterabfrage (korreliert)}
    In folgender Abfrage wird jede Attacke ausgegeben, welche eine höhere Stärke besitzt, als der Durchschnitt aller Attacken mit dem selben Typ.

    \begin{lstlisting}[language=mysql]
        SELECT
            ID,
            Name,
            Staerke,
            Typ
        FROM attacke AS attacke1
        WHERE Staerke > (
            SELECT AVG(Staerke)
            FROM attacke AS attacke1
            -- attacke2.Typ unterscheidet sich potentiell fuer jede Entitaet
            WHERE attacke1.Typ = attacke2
        );
    \end{lstlisting}

    \setcounter{rownum}{0}
    \begin{tabular}{I|ITIT}
        \rowcolor{gray!35}
        & \multicolumn{1}{T}{ID} & \multicolumn{1}{T}{Name} & \multicolumn{1}{T}{Staerke} & \multicolumn{1}{T}{Typ} \\\hline
        1 & 5 & Megahieb & 80 & Normal \\
        2 & 8 & Eishieb & 75 & Eis \\
        3 & 13 & Klingensturm & 80 & Normal \\
        \multicolumn{1}{c|}{\dots} & \multicolumn{4}{c}{\dots} \\
        230 & 825 & Astralfragmente & 120 & Geist \\
    \end{tabular}
\end{example}

\begin{defi}{Nicht skalare Unterabfrage}
    In \emph{nicht skalaren Unterabfragen} werden mehrere Attribute der Entitäten in der Unterabfrage genutzt.
    So müssen mehrere Tupel auf Gleichheit überprüft werden.
    
    Sollte eine Unterabfrage nicht einzeilig sein, so ist diese automatisch nicht skalar.
    Dementsprechend ist eine Unterabfrage ausschließlich dann skalar, wenn man exakt eine Entität mit exakt einem Attribut erhält.
\end{defi}

\begin{example}{Unterabfrage (nicht skalar)}
    Möchte man sich beispielsweise alle Pokémon ausgeben, welche exakt die selbe Typkombination wie Bisasam haben, so nutzt man:

    \begin{lstlisting}[language=mysql]
        SELECT *
        FROM pokemon
        WHERE
            (PrimaerTyp, SekundaerTyp) = (
                -- Entspricht einem Tupel aus 2 Typen
                -- Obacht: Keiner der Typen darf NULL sein
                SELECT
                    PrimaerTyp,
                    SekundaerTyp
                FROM pokemon
                WHERE Name = 'Bisasam'
            )
            AND Name <> 'Bisasam';
    \end{lstlisting}

    \setcounter{rownum}{0}
    \begin{tabular}{I|ITIIITT}
        \rowcolor{gray!35}
        & \multicolumn{1}{T}{ID} & \multicolumn{1}{T}{Name} & \multicolumn{1}{T}{Groesse} & \multicolumn{1}{T}{Gewicht} & \multicolumn{1}{T}{Generation} & \multicolumn{1}{T}{PrimaerTyp} & \multicolumn{1}{T}{SekundaerTyp} \\\hline
        1 & 2 & Bisaknosp & 1 & 13 & 1 & Pflanze & Gift \\
        2 & 3 & Bisaflor & 2 & 100 & 1 & Pflanze & Gift \\
        3 & 43 & Myrapla & 0.5 & 5.4 & 1 & Pflanze & Gift \\
        \multicolumn{1}{c|}{\dots} & \multicolumn{7}{c}{\dots} \\
        13 & 591 & Hutsassa & 0.6 & 10.5 & 5 & Pflanze & Gift \\
    \end{tabular}
\end{example}

\subsection{Funktionen}    

\begin{sql}{ANY}
    \begin{itemize}
        \item \texttt{< ANY} gibt Einträge zurück, welche kleiner als ein beliebiges Element\footnote{d.h. kleiner als das Maximum} einer Menge sind.
        \item \texttt{> ANY} gibt Einträge zurück, welche größer als ein beliebiges Element\footnote{d.h. größter als das Minimum} einer Menge sind.
        \item \texttt{= ANY} gibt Einträge zurück, welche mit einem Werte einer Menge übereinstimmen.
            Diese Bedingung entspricht dem Keyword \texttt{IN}.
    \end{itemize}
\end{sql} 

\begin{sql}{ALL}
    \begin{itemize}
        \item \texttt{< ALL} gibt Einträge zurück, welche kleiner als alle Elemente\footnote{d.h. kleiner als das Minimum} einer Menge sind.
        \item \texttt{> ALL} gibt Einträge zurück, welche größer als alle Elemente\footnote{d.h. größter als das Maximum} einer Menge sind.
    \end{itemize}
\end{sql}

\begin{sql}{EXISTS}
    Bedingungen mit \texttt{EXISTS} sind erfüllt, sollte bei einer Unterabfrage eine nicht leere Menge zurückgegeben werden.   

    \texttt{EXISTS} wird meist mit korrelierenden Unterabfragen genutzt.
\end{sql}

\begin{bonus}{INSERT INTO mit Unterabfragen}
    Wenn man automatisch Werte aus bestehenden Tabellen in neue Tabellen einfügen möchte, kann man Unterabfragen nutzen.

    Hierbei entfällt das Keyword \texttt{VALUES}:

    \begin{lstlisting}[language=mysql]
        CREATE TABLE feuerpokemon (
            ID int NOT NULL,
            Bezeichnung varchar(255),
            ...
        );
        INSERT INTO feuerpokemon
            -- Unterabfrage
            SELECT *
            FROM pokemon
            WHERE
                PrimaerTyp = 'Feuer'
                OR SekundaerTyp = 'Feuer';
    \end{lstlisting}
\end{bonus}

\subsection{Beispiele}

\begin{example}{Unterabfrage}
    Welche Kombinationen von Typen werden von mehr Pokémon genutzt, als die Kombination \texttt{('Feuer', NULL)}?

    \exampleseparator

    \begin{lstlisting}[language=mysql]
        SELECT
            PrimaerTyp,
            SekundaerTyp
        FROM pokemon
        GROUP BY
            PrimaerTyp,
            SekundaerTyp
        HAVING COUNT(*) > (
                -- COUNT(*) = 33
                SELECT COUNT(*)
                FROM pokemon
                WHERE
                    PrimaerTyp = 'Feuer'
                    AND SekundaerTyp IS NULL
                GROUP BY
                    PrimaerTyp,
                    SekundaerTyp
        );
    \end{lstlisting}

    \setcounter{rownum}{0}
    \begin{tabular}{I|TT} 
        \rowcolor{gray!35}
        & \multicolumn{1}{T}{PrimaerTyp} & \multicolumn{1}{T}{SekundaerTyp} \\\hline
        1 & Wasser & NULL \\
        2 & Normal & NULL \\
        3 & Psycho & NULL \\
        4 & Pflanze & NULL \\
    \end{tabular}
\end{example}

\begin{example}{Unterabfrage}
    Geben Sie alle Typen aus, welche nicht as Sekundärtyp in Kombination mit dem Primärtypen 'Feuer' für mindestens ein Pokémon auftreten.

    \exampleseparator

    \begin{lstlisting}[language=mysql]
        SELECT *
        FROM typ
        WHERE NOT EXISTS (
            SELECT *
            FROM pokemon
            WHERE
                PrimaerTyp = 'Feuer'
                AND SekundaerTyp = typ.bezeichnung
        );
    \end{lstlisting}

    \setcounter{rownum}{0}
    \begin{tabular}{I|T}    
        \rowcolor{gray!35}
        & \multicolumn{1}{T}{Bezeichnung} \\\hline
        1 & Eis \\
        2 & Elektro \\
        3 & Fee \\
        \multicolumn{1}{c|}{\dots} & \multicolumn{1}{c}{\dots} \\
        6 & Pflanze \\
    \end{tabular}
\end{example}

\begin{example}{Unterabfrage}
    Geben Sie alle Pokémon aus, welche exakt die selbe Typkombination wie Glumanda, also \texttt{('Feuer', NULL)}, haben.

    Da der Sekundärtyp von Glumanda \texttt{NULL} ist, muss man das Beispiel aus nicht skalaren Unterabfragen anpassen.

    \exampleseparator

    In der ersten, simplen, Lösung ersetzt man alle \texttt{NULL}-Sekundärtypen mit einem String:
    
    \begin{lstlisting}[language=mysql]
        SELECT *
        FROM pokemon
        WHERE
            (PrimaerTyp, COALESCE(SekundaerTyp, 'NULL')) = (
                SELECT
                    PrimaerTyp,
                    COALESCE(SekundaerTyp, 'NULL')
                FROM pokemon
                WHERE Name = 'Glumanda'
            )
            AND Name <> 'Glumanda';
    \end{lstlisting}

    Alternativ nutzt man korrelierende Unterabfragen:

    \begin{lstlisting}[language=mysql]
        SELECT *
        FROM pokemon AS pokemon1
        WHERE
            EXISTS (
                SELECT *
                FROM pokemon AS pokemon2
                WHERE
                    Name = 'Glumanda'
                    AND pokemon1.PrimaerTyp = pokemon2.PrimaerTyp
                    AND (
                        -- pokemon1 hat den selben Sekundaertyp wie pokemon2
                        pokemon1.SekundaerTyp = pokemon2.SekundaerTyp
                        -- der Sekundaertyp von pokemon1, sowie pokemon2 ist NULL 
                        OR (
                            pokemon1.SekundaerTyp IS NULL
                            AND pokemon2.SekundaerTyp IS NULL
                        )
                    )
            )
            AND Name <> 'Glumanda';
    \end{lstlisting}

    \setcounter{rownum}{0}
    \begin{tabular}{I|ITIIITT}
        \rowcolor{gray!35}
        & \multicolumn{1}{T}{ID} & \multicolumn{1}{T}{Name} & \multicolumn{1}{T}{Groesse} & \multicolumn{1}{T}{Gewicht} & \multicolumn{1}{T}{Generation} & \multicolumn{1}{T}{PrimaerTyp} & \multicolumn{1}{T}{SekundaerTyp} \\\hline
        1 & 5 & Glutexo & 1.1 & 19 & 1 & Feuer & NULL \\
        2 & 37 & Vulpix & 0.6 & 9.9 & 1 & Feuer & NULL \\
        3 & 38 & Vulnona & 1.1 & 19.9 & 1 & Feuer & NULL \\
        \multicolumn{1}{c|}{\dots} & \multicolumn{7}{c}{\dots} \\
        32 & 815 & Liberlo & 1.4 & 33 & 8 & Feuer & NULL \\
    \end{tabular}
\end{example}
\section{Group Functions}

\begin{sql}{GROUP BY}
    \texttt{GROUP BY} teilt die Gesamtmenge in Teilmengen bzw. Gruppierungen auf.
    
    Auf diese Teilmengen können ausschließlich Methoden speziell für Datensätze, wie beispielsweise \texttt{COUNT}, aufgerufen.
    Attributwerte, welche sich innerhalb einer Teilmenge unterscheiden, können nicht direkt aufgerufen werden.\footnote{
        Da es in der Tabelle \texttt{attacke} Attacken mit dem selben \texttt{Namen} gibt, welche sich insbesondere durch ihre \texttt{Schadensklasse} unterscheiden, kann man, wenn man nach dem \texttt{Namen} gruppiert, die jeweilige \texttt{Schadensklasse} nicht aufrufen.
        Je nach DBMS ist es trotzdem möglich die gleichbleibenden Attribute \texttt{Name}, \texttt{Typ}, \texttt{Stärke}, \texttt{Genauigkeit}, \texttt{AP} und \texttt{Generation} zu verwenden.
    }

    Gibt man bei dem Keyword \texttt{GROUP BY} mehrere Attribute an, wird der Datensatz nach dem Tupel dieser gruppiert.
    Dabei wird zunächst nach dem ersten Attribut gruppiert, folgend innerhalb dieser Teilgruppen dem Zweiten, etc. 
\end{sql}

\begin{sql}{HAVING}
    \texttt{HAVING} filtert die mit \texttt{GROUP BY} gruppierten Teilmengen.
    
    Dabei werden Bedingungen auf das Ergebnis der Abfrage angewandt.

    Gibt man sich beispielsweise alle Typen aus, welche mindestens von 25 Attacken genutzt werden, so erhält man:
    
    \begin{lstlisting}[style=SqlInputStyle]
        SELECT
            Typ,
            COUNT(*) AS Anzahl
        FROM attacke
        GROUP BY Typ
        HAVING Anzahl >= 25
        ORDER BY Anzahl DESC;
    \end{lstlisting}

    \begin{tabular}{I|TI}
        \rowcolor{gray!35}
        & \multicolumn{1}{T}{Typ} & \multicolumn{1}{T}{Anzahl} \\\hline
        1 & Normal & 188 \\
        2 & Psycho & 69 \\
        3 & Pflanze & 52 \\
        \multicolumn{1}{c|}{\dots} & \multicolumn{2}{c}{\dots} \\
        17 & Drache & 27 \\
    \end{tabular}
\end{sql}

\subsection{Funktionen}

\begin{sql}{MIN, MAX}
    \texttt{MIN} gibt den alphanumerisch ersten Eintrag eines Datensatzes zurück.

    \texttt{MAX} hingegen gibt den alphanumerisch letzten Eintrag eines Datensatzes zurück.
\end{sql}

\begin{sql}{AVG}
    \texttt{AVG} gibt den durchschnittlichen Zahlwert eines Datensatzes zurück.
\end{sql}

\begin{sql}{SUM}
    \texttt{SUM} gibt die Summe alles Zahlenwerte eines Datensatzes zurück
\end{sql}

\subsection{Beispiele}

\begin{example}{GROUP BY}
    \begin{lstlisting}[style=SqlInputStyle]
        SELECT
            typ,
            MIN(Staerke),
            MAX(Staerke),
            SUM(Staerke),
            AVG(Staerke)
        FROM attacke
        GROUP BY typ;
    \end{lstlisting}

    \begin{tabular}{I|TIIIII}
        \rowcolor{gray!35}
        & \multicolumn{1}{T}{typ} & \multicolumn{1}{T}{COUNT(*)} & \multicolumn{1}{T}{MIN(Staerke)} & \multicolumn{1}{T}{MAX(Staerke)} & \multicolumn{1}{T}{SUM(Staerke)} & \multicolumn{1}{T}{AVG(Staerke)} \\\hline
        1 & Boden & 29 & 10 & 120 & 1365 & 68.2500 \\
        2 & Drache & 27 & 10 & 185 & 2005 & 91.1364 \\
        3 & Eis & 29 & 10 & 140 & 1575 & 71.5909 \\
        4 & Elektro & 43 & 10 & 210 & 2740 & 88.3871 \\
        5 & Fee & 30 & 10 & 190 & 1250 & 89.2857 \\
        6 & Feuer & 40 & 35 & 180 & 3370 & 96.2857 \\
        7 & Flug & 30 & 10 & 140 & 1735 & 75.4348 \\
        8 & Geist & 30 & 10 & 200 & 1815 & 90.7500 \\
        9 & Gestein & 23 & 10 & 190 & 1290 & 80.6250 \\
        10 & Gift & 31 & 10 & 120 & 1080 & 63.5294 \\
        11 & Käfer & 32 & 10 & 120 & 1340 & 63.8095 \\
        12 & Kampf & 51 & 10 & 150 & 2810 & 75.9459 \\
        13 & Normal & 188 & 10 & 250 & 5441 & 72.5467 \\
        14 & Pflanze & 52 & 10 & 150 & 2560 & 77.5758 \\
        15 & Psycho & 69 & 10 & 200 & 2385 & 91.7308 \\
        16 & Stahl & 32 & 10 & 200 & 1815 & 86.4286 \\
        17 & Unlicht & 47 & 10 & 180 & 1950 & 72.2222 \\
        18 & Wasser & 42 & 10 & 195 & 2635 & 77.5000 \\
    \end{tabular}
    %\setcounter{rownum}{0}
\end{example}

\begin{example}{GROUP BY}
    Geben Sie alle Attacken aus mit einer echter Stärke (>0) geordnet aus.
    Gruppieren Sie diese nach Typ und ermitteln Sie die \texttt{MIN} und \texttt{MAX} Stärke.
    Ergänzen Sie diese um die Durchschnittsstärke und geben Sie nur die Einträge aus, bei denen diese über 75 liegt.

    \exampleseparator

    \begin{lstlisting}[style=SqlInputStyle]
        SELECT
            Typ,
            COUNT(*),
            MIN(Staerke),
            MAX(Staerke),
            AVG(Staerke)
        FROM attacke
        WHERE Staerke > 0
        GROUP BY Typ
        HAVING AVG(Staerke) > 75
        ORDER BY AVG(Staerke) DESC;
    \end{lstlisting}

    \begin{tabular}{I|TTTTT}
        \rowcolor{gray!35}  
        & \multicolumn{1}{T}{Typ} & \multicolumn{1}{T}{COUNT(*)} & \multicolumn{1}{T}{MIN(Staerke)} & \multicolumn{1}{T}{MAX(Staerke)} & \multicolumn{1}{T}{AVG(Staerke)} \\\hline
        1 & Feuer & 35 & 35 & 180 & 96.2857 \\ 
        2 & Psycho & 26 & 10 & 200 & 91.7308 \\
        3 & Drache & 22 & 10 & 185 & 91.1364 \\ 
        \multicolumn{1}{c|}{\dots} & \multicolumn{5}{c}{\dots} \\
        12 & Flug & 23 & 10 & 140 & 75.4348 \\
    \end{tabular}
\end{example}

\begin{bonus}{GROUP BY}
    Errechnen Sie

    \begin{itemize}
        \item Erwartungswert ($\mu$, E),
        \item Varianz ($\sigma^2$, Var), sowie
        \item Standardabweichung ($\sigma$, SD)
    \end{itemize}

    der jeweiligen nach Typ gruppierten Attacken.

    \exampleseparator

    \begin{lstlisting}[style=SqlInputStyle]
        SELECT
            Typ,
            ROUND(AVG(Staerke), 3) AS E,
            ROUND(AVG(POW(Staerke, 2)) - POW(AVG(Staerke), 2), 3) AS Var,
            ROUND(SQRT(AVG(POW(Staerke, 2)) - POW(AVG(Staerke), 2)), 3) AS SD
        FROM attacke
        GROUP BY Typ
        ORDER BY SD;
    \end{lstlisting}

    \begin{tabular}{I|TIII}
        \rowcolor{gray!35}
        & \multicolumn{1}{T}{Typ} & \multicolumn{1}{T}{$\mu$} & \multicolumn{1}{T}{$\sigma^2$} & \multicolumn{1}{T}{$\sigma$} \\\hline
        1 & Käfer & 63.810 & 747.395 & 27.339 \\
        2 & Boden & 68.250 & 795.688 & 28.208 \\ 
        3 & Unlicht & 72.222 & 859.88 & 29.324 \\
        \multicolumn{1}{c|}{\dots} & \multicolumn{4}{c}{\dots} \\
        18 & Geist & 90.750 & 2633.188 & 51.315 \\
    \end{tabular}
\end{bonus}

\begin{example}{GROUP BY}
    Gruppieren Sie alle Pokémon nach Generation.
    Geben Sie diese Teilmengen sortiert nach Anzahl aus.
    Ermitteln Sie zusätzlich das gerundete Durchschnittsgewicht der Pokémon jeder Generation.

    \exampleseparator

    \begin{lstlisting}[style=SqlInputStyle]
SELECT
    Generation, COUNT(*), ROUND(AVG(Gewicht), 3)
FROM pokemon
GROUP BY Generation
ORDER BY COUNT(*);
    \end{lstlisting}

    \begin{tabular}{I|III}
        \rowcolor{gray!35}
        & \multicolumn{1}{T}{Generation} & \multicolumn{1}{T}{COUNT(*)} & \multicolumn{1}{T}{ROUND(AVG(Gewicht), 3)} \\\hline
        1 & 6 & 72 & 51.401 \\
        2 & 7 & 88 & 109.661 \\
        3 & 8 & 89 & 76.267 \\
        \multicolumn{1}{c|}{\dots} & \multicolumn{3}{c}{\dots} \\
        8 & 5 & 156 & 52.403 \\
    \end{tabular}
\end{example}
\section{Join}

\begin{defi}{Kartesisches Produkt}
    Wenn man alle Entitäten einer Tabelle mit jeder Entität aus einer anderen Tabelle kombinieren möchte, benötigt man das \emph{kartesische Produkt}.

    Möchte man jede mögliche Typkombination\footnote{Sollte man das kartesische Produkt einer Tabelle mit sich selbst bilden, benötigt man Aliase für die Tabellen} erhalten, nutzt man \texttt{SELECT} zwei mal auf \texttt{Typ}:

    \mint{mysql}|SELECT * FROM typ AS typ1, typ AS typ2;|

    \lstinputlisting[style=BashOutputStyle]{includes/code/out_definition_join_kartesisches_produkt.txt}

    Wenn eine Relation verschiedener Tabellen über einen Fremdschlüssel realisiert ist, kann man Entitäten aus diesen Tabellen zueinander zuordnen.
    So wird nicht nur der Fremdschlüssel angezeigt, sondern alle zugehörigen Attribute.
    
    Da wir die aus dem kartesischen Produkt erhaltene Menge noch sinnvoll selektieren müssen, nutzen wir \texttt{WHERE}.

    In unserem Beispiel kann kein Pokemon zwei mal den selben Typen besitzen:

    \begin{minted}{mysql}
SELECT * FROM
    typ AS typ1,
    typ AS typ2
WHERE NOT typ1.Bezeichnung = typ2.Bezeichnung;
    \end{minted}

    \lstinputlisting[style=BashOutputStyle]{includes/code/out_definition_join_kartesisches_produkt_where.txt}
\end{defi}

\begin{sql}{JOIN}
    Das selbe Ergebnis erhält man, wenn man \texttt{JOIN} mit dem Schlüsselwort \texttt{ON} nutzt:\footnote{
        Hierbei unterscheiden sich die beiden Abfragen ausschließlich in der Lesbarkeit und Optimierungsmöglichkeiten für das DBMS.
        Bei \texttt{JOIN} muss das DBMS nicht alle Kombinationen ermitteln, nur um diese dann zu filtern.
    }

    \begin{minted}{mysql}
SELECT * FROM
    typ AS typ1
JOIN
    typ AS typ2
ON NOT typ1.Bezeichnung = typ2.Bezeichnung;
    \end{minted}
\end{sql}

\begin{sql}{CROSS JOIN}
    \texttt{CROSS JOIN} gibt das unselektierte kartesische Produkt aus.

    Dementsprechend sind die folgenden Abfragen erneut äquivalent:

    \begin{minted}{mysql}
SELECT * FROM typ AS typ1 CROSS JOIN typ AS typ2;
SELECT * FROM typ AS typ1 JOIN typ as typ2 ON TRUE;
SELECT * FROM typ AS typ1, typ AS typ2;
    \end{minted}
\end{sql}

\begin{sql}{INNER JOIN}
    \texttt{INNER JOIN} wird automatisch genutzt, wenn man \texttt{JOIN} nutzt.

    Dementsprechend sind folgende abfragen äquivalent:

    \begin{minted}{mysql}
SELECT * FROM typ AS typ1 JOIN typ AS typ2
    ON NOT typ1.Bezeichnung = typ2.Bezeichnung;
SELECT * FROM typ AS typ1 INNER JOIN typ AS typ2
    ON NOT typ1.Bezeichnung = typ2.Bezeichnung;
    \end{minted}

    Wenn man sich beispielsweise alle Pokémon samt ihrer Entwicklung anzeigen lassen möchte, so erhält man:

    \begin{minted}{mysql}
SELECT
    pokemon.ID, pokemon.Name, entwicklung.Von, entwicklung.Zu
FROM pokemon JOIN entwicklung
ON pokemon.ID = entwicklung.Von;
    \end{minted}

    \lstinputlisting[style=BashOutputStyle]{includes/code/out_definition_join_inner.txt}
\end{sql}

\begin{defi}{SELF JOIN}
    Als \texttt{SELF JOIN} werden \texttt{JOIN}s bezeichnet, welche eine reflexive Fremdschlüsselbeziehung abfragen.
    
    Hierbei ist zu beachten, dass \texttt{SELF} kein Keyword ist, sondern die reine Bezeichnung für solche Relationen.

    Dies wurde mit dem Beispiel \texttt{typ} auf \texttt{typ} bereits dargestellt.
\end{defi}

\begin{defi}{OUTER JOIN}
    \texttt{OUTER JOIN} berücksichtigt ebenfalls Entitäten, für die kein passendes Pendant existiert.
    Für diese Entitäten sind alle Attribute der jeweils anderen Entitätenmenge \texttt{NULL}.

    Ebenfalls \texttt{OUTER JOIN} ist keine gültige Abfrage.
    Das Keyword \texttt{OUTER} benötigt immer eine Leserichtung.

    Wenn wir erneut das Beispiel von \texttt{INNER JOIN} betrachten, sehen wir, dass das Pokémon mit der ID 3 nicht vorkommt.
    Dies ist dem Fakt geschuldet, dass dieses Pokémon keine Entwicklung besitzt.
    Sollte man dieses Pokemon trotzdem angezeigt bekommen, benötigt man \texttt{OUTER JOIN}.
\end{defi}

\begin{sql}{LEFT OUTER JOIN}
    \texttt{LEFT OUTER JOIN} beachtet alle Entitäten \emph{links} der Relation.

    Nun werden alle Pokémon dargestellt, ggf. mit \texttt{NULL}-Attributen bei Entwicklung:

    \begin{minted}{mysql}
SELECT
    pokemon.ID, pokemon.Name, entwicklung.Von, entwicklung.Zu
FROM pokemon LEFT OUTER JOIN entwicklung
ON pokemon.id = entwicklung.von;
    \end{minted}

    \lstinputlisting[style=BashOutputStyle]{includes/code/out_sql_join_left_outer.txt}
\end{sql}

\begin{sql}{RIGHT OUTER JOIN}
    \texttt{RIGHT OUTER JOIN} beachtet alle Entitäten \emph{rechts} der Relation.

    Durch Vertauschen von Entitätstypen und Leserichtung erhält man äquivalente Befehle.

    So würde unser Beispiel aus \texttt{LEFT OUTER JOIN} folgendermaßen aussehen:

    \begin{minted}{mysql}
SELECT
    pokemon.ID, pokemon.Name, entwicklung.Von, entwicklung.Zu
FROM entwicklung RIGHT OUTER JOIN pokemon
ON pokemon.id = entwicklung.von;
    \end{minted}
\end{sql}

\begin{sql}{FULL OUTER JOIN}
    \texttt{FULL OUTER JOIN} ist eine \emph{Vereinigung} von \texttt{LEFT OUTER JOIN} und \texttt{RIGHT OUTER JOIN}.
    Das bedeutet, es sind alle Entitäten beider Seiten dabei, beidseitig ggf. mit \texttt{NULL} aufgefüllt\footnote{
    Manche DBMS bieten keinen Befehl für einen \texttt{FULL OUTER JOIN}.
    In diesem Fall vereinigt man die Ergebnisse der beiden \texttt{OUTER JOIN}s mit \texttt{UNION}.}.

    \begin{minted}{mysql}
SELECT
    pokemon.ID, pokemon.Name, entwicklung.Von, entwicklung.Zu
FROM pokemon FULL OUTER JOIN entwicklung
ON pokemon.id = entwicklung.von;

SELECT
    pokemon.ID, pokemon.Name, entwicklung.Von, entwicklung.Zu
FROM pokemon LEFT OUTER JOIN entwicklung
ON pokemon.id = entwicklung.von;
UNION
SELECT
    pokemon.ID, pokemon.Name, entwicklung.Von, entwicklung.Zu
FROM pokemon RIGHT OUTER JOIN entwicklung
ON pokemon.id = entwicklung.von;
    \end{minted}
\end{sql}

\begin{bonus}{NATURAL JOIN}
    Von der Verwendung von \texttt{NATURAL JOIN}s wird abgeraten!

    Bei Tabellen mit gleichnamigen Attributen kann auf die explizite Angabe der Bedingung verzichtet werden.
    Das DBMS verwendet dann \emph{alle} gleichnamigen Attribute.
\end{bonus}

\subsection{Beispiele}

\begin{example}{JOIN}
    Geben Sie für alle Pokemon jeweils die Entwicklung mit Level, sowie die Typen aus.

    Nutzen Sie:
    \begin{enumerate}[label=\alph*)]
        \item einen \texttt{INNER JOIN} mit \texttt{WHERE}
        \item einen \texttt{INNER JOIN} mit \texttt{JOIN}
    \end{enumerate}

    Fügen Sie noch den Namen, sowie die Typen der Entwicklung an.

    \exampleseparator

    \begin{enumerate}[label=\alph*)]
        \item \begin{minted}{mysql}
SELECT
    von.ID, von.Name, von.PrimaerTyp, von.SekundaerTyp,
    entwicklung.Level,
    zu.ID, zu.Name
FROM 
    pokemon AS von,
    pokemon AS zu,
    entwicklung
WHERE von.ID = entwicklung.Von AND entwicklung.Zu = zu.ID;
            \end{minted}
        \item \begin{minted}{mysql}
SELECT
    von.ID, von.Name, von.PrimaerTyp, von.SekundaerTyp,
    entwicklung.Level,
    zu.ID, zu.Name
FROM pokemon AS von
JOIN entwicklung ON von.ID = entwicklung.Von
JOIN pokemon AS zu ON entwicklung.Zu = zu.ID;
            \end{minted}
    \end{enumerate}
    
    \lstinputlisting[style=BashOutputStyle]{includes/code/out_example_join_inner.txt}
\end{example}

\begin{example}{JOIN}
    Geben Sie zusätzlich auch Pokémon aus, die (noch) keine Entwicklung haben.

    \exampleseparator

    \begin{minted}{mysql}
SELECT
    von.ID, von.Name, von.PrimaerTyp, von.SekundaerTyp,
    entwicklung.Level,
    zu.ID, zu.Name
FROM pokemon AS von
LEFT OUTER JOIN entwicklung ON von.ID = entwicklung.Von
LEFT OUTER JOIN pokemon AS zu ON entwicklung.Zu = zu.ID;
    \end{minted}

    \lstinputlisting[style=BashOutputStyle]{includes/code/out_example_join_outer.txt}
\end{example}

\begin{example}{JOIN}
    Geben Sie alle Pokémon aus, welche bislang keine Entwicklung besitzen.

    \exampleseparator

    \begin{minted}{mysql}
SELECT
    pokemon.ID, pokemon.Name, pokemon.PrimaerTyp, pokemon.SekundaerTyp
FROM pokemon
LEFT OUTER JOIN entwicklung ON pokemon.ID = entwicklung.Von
WHERE entwicklung.Von IS NULL;
    \end{minted}

    \lstinputlisting[style=BashOutputStyle]{includes/code/out_example_join_is_null.txt}
\end{example}

\begin{example}{JOIN}
    Geben Sie alle Pokémon aus, welche sich aus Evoli entwickeln.

    \exampleseparator

    \begin{minted}{mysql}
SELECT
    zu.ID, zu.Name, zu.PrimaerTyp, zu.SekundaerTyp,
    von.ID, von.Name
FROM pokemon AS zu
JOIN entwicklung ON zu.ID = entwicklung.Zu
JOIN pokemon AS von ON entwicklung.Von = von.ID
WHERE von.Name = 'Evoli';
    \end{minted}

    \lstinputlisting[style=BashOutputStyle]{includes/code/out_example_join_evoli.txt}
\end{example}
\section{Transaktionen}
\section{Trigger, Stored Procedures}

\section{Schemas und Tabellen}

\begin{sql}{USE}
    \texttt{USE} legt ein Default-Schema fest.
    So kann bei Tabellen in SQL-Ausdrücken die explizite Angabe des Schemas entfallen.

    Möchte man also z.B. die Datenbank \texttt{db\_pokemon} auswählen, so nutzt man:

    \begin{lstlisting}[language=mysql]
        USE db_pokemon;
    \end{lstlisting}

    Folgend kann man den folgenden Befehl kürzen:

    \begin{lstlisting}[language=mysql]
        SELECT * FROM pokemon;
    \end{lstlisting}

    \begin{lstlisting}[language=mysql]
        SELECT * FROM !pokemon.!pokemon;
    \end{lstlisting}
\end{sql}

\begin{sql}{SHOW}
    \texttt{SHOW} wird genutzt um DBMS-spezifische Strukturen abzufragen.

    Dieser Befehl gibt eine Tabelle zurück.
    Dementsprechend lassen sich Projektionen sowie Selektionen auf das Ergebnis anwenden.

    Um alle Schemas bzw Datenbanken anzuzeigen nutzt man:

    \begin{lstlisting}[language=mysql]
        SHOW SCHEMAS;
    \end{lstlisting}

    \begin{lstlisting}[language=mysql]
        SHOW DATABASES;
    \end{lstlisting}

    \setcounter{rownum}{0}
    \begin{tabular}{I|T}
        \rowcolor{gray!35}
        & \multicolumn{1}{T}{Database} \\\hline
        1 & pokemon \\
    \end{tabular}

    Alternativ kann man sich innerhalb einer Datenbank alle Tabellen abfragen:

    \begin{lstlisting}[language=mysql]
        SHOW TABLES;
    \end{lstlisting}

    \begin{lstlisting}[language=mysql]
        SHOW TABLES !FROM pokemon!;
    \end{lstlisting}

    \setcounter{rownum}{0}
    \begin{tabular}{I|T}
        \rowcolor{gray!35}
        & \multicolumn{1}{T}{Tables\_in\_pokemon} \\\hline
        1 & arenaleiter \\
        2 & attacke \\
        3 & attacke\_tm \\
        4 & effektivitaet \\
        5 & entwicklung \\
        6 & generation \\
        7 & item \\
        8 & lernt \\
        9 & pokemon \\
        10 & typ \\
        11 & version \\
    \end{tabular}

    Um sich die Struktur einer Tabelle anzuzeigen nutzt man:

    \begin{lstlisting}[language=mysql]
        DESCRIBE pokemon; 
    \end{lstlisting}

    \begin{lstlisting}[language=mysql]
        SHOW COLUMNS FROM pokemon; 
    \end{lstlisting}

    \setcounter{rownum}{0}
    \begin{tabular}{I|TTTTTT}
        \rowcolor{gray!35}
        & \multicolumn{1}{T}{Field} & \multicolumn{1}{T}{Type} & \multicolumn{1}{T}{Null} & \multicolumn{1}{T}{Key} & \multicolumn{1}{T}{Default} & \multicolumn{1}{T}{Extra} \\\hline
        1 & ID & int & NO & PRI & NULL &  \\
        2 & Name & varchar(255) & NO &  & NULL &  \\
        3 & Groesse & float & NO &  & NULL &  \\
        4 & Gewicht & float & NO &  & NULL &  \\
        5 & Generation & int & NO & MUL & NULL &  \\
        6 & PrimaerTyp & varchar(255) & YES & MUL & NULL &  \\
        7 & SekundaerTyp & varchar(255) & YES & MUL & NULL &  \\
    \end{tabular}

    \begin{itemize}
        \item \emph{Field} ist dabei der Spaltenname
        \item \emph{Type} ist der Typ der Werte in der Spalte
        \item \emph{Null} gibt an, ob der Wert der Spalte NULL sein darf
        \item \emph{Key} gibt an, ob der Wert der Spalte ein
            
            \begin{itemize}
                \item \emph{PRI} Primärschlüssel
                \item \emph{MUL} Fremdschlüssel
            \end{itemize}

            ist.
    \end{itemize}
\end{sql}

\begin{sql}{DROP}
    \texttt{DROP} bzw. \texttt{DELETE} löscht DBMS-spezifische Strukturen.

    Wenn man die gesamte Datenbank pokemon löschen möchte, so nutzt man:

    \begin{lstlisting}[language=mysql]
        DROP DATABASE <Tabellenname>;
    \end{lstlisting}

    Möchte man die Tabelle \texttt{pokemon} löschen:

    \begin{lstlisting}[language=mysql]
        DROP TABLE <Tabellenname>;
    \end{lstlisting}
\end{sql}

\begin{sql}{CREATE}
    \texttt{CREATE} erstellt DBMS-spezifische Strukturen.

    Die Datenbank \texttt{pokemon} wurde wir folgt erstellt:

    \begin{lstlisting}[language=mysql]
        CREATE DATABASE pokemon;
    \end{lstlisting}

    Um Tabellen zu erstellen muss man die jeweilige Spalten beschreiben:

    \begin{lstlisting}[language=mysql]
        CREATE TABLE <Tabellenname> (
            <Spaltenname> [<Parameter>, ...],
            ...,
            PRIMARY KEY (<Spaltenname>, ..., <Spaltenname>),
            FOREIGN KEY (<Spaltenname>) REFERENCES <Tabellenname> (Spaltenname),
            ...
        );
    \end{lstlisting}

    Einige relevanten Parameter sind:

    \begin{itemize}
        \item \emph{Typ} beschreibt den Typen der Spaltenwerte.
            Die meisten Typen benötigen eine Länge; einige haben jedoch eine Standardlänge:
        
            \begin{itemize}
                \item \texttt{INT(n)} eine ganze Zahl der Länge n
                \item \texttt{FLOAT(n)} eine Zahl der Länge n
                \item \texttt{DECIMAL(n, m)} eine Zahl der Länge n mit m Nachkommastellen
                \item \texttt{CHAR(n)} ein String fester Länge n
                \item \texttt{VARCHAR(n)} ein String variabler Länge 0 bis n
                \item \texttt{BOOLEAN} bzw. \texttt{TINYINT(1)} speichert \texttt{TRUE} bzw. 1 oder \texttt{FALSE} bzw. 0
            \end{itemize}
        \item \texttt{DEFAULT} setzt einen Standardwert fest
        \item \texttt{UNIQUE} legt fest, ob der Wert einzigartig soll
        \item \texttt{NOT NULL} legt fest, ob der Wert \texttt{NULL} sein darf
        \item \texttt{CHECK(<Bedingung>)} Einzufügende Werte müssen die Bedingung erfüllen
    \end{itemize}

    Ein \texttt{PRIMARY KEY} ist ein Primärschlüssel, der die Entität eindeutig beschreibt.
    Dieser darf nie \texttt{NULL} sein.

    Ein \texttt{FOREIGN KEY} ist ein Fremdschlüssel, der Primärschlüssel einer anderen Entität referenziert.
\end{sql}

\begin{sql}{Zeit- und Datumstypen}
    Zusätzlich gibt es komplexere Typen, welche zur Speicherung von Daten genutzt werden:
    
    Dazu vorab:
    MySQL nutzt \texttt{session} und \texttt{global} Konfigurationen.

    Die \texttt{session} Konfiguration wird beim Abschalten der Datenbank auf \texttt{global} zurückgesetzt.

    \begin{lstlisting}[language=mysql]
        SELECT @@session.time_zone, @@global.time_zone;;
    \end{lstlisting}

    \setcounter{rownum}{0}
    \begin{tabular}{I|TT}
        \rowcolor{gray!35}
        & \multicolumn{1}{T}{@@session.time\_zone} & \multicolumn{1}{T}{@@global.time\_zone} \\\hline
        1 & SYSTEM & SYSTEM \\
    \end{tabular}

    Die Zeitzone kann mit folgendem Befehl angepasst werden:

    \begin{lstlisting}[language=mysql]
        SET SESSION TIME_ZONE = <Zeitzone>
    \end{lstlisting}

    Dabei kann als Zeitzone folgendes eingestellt werden:

    \begin{itemize}
        \item Eine exakte Zeitzone: \lstinline[language=mysql]{'Europe/Berlin'}
        \item Ein Offset: \lstinline[language=mysql]{'+02:00'}
    \end{itemize}

    Einige Datumstypen sind:

    \begin{itemize}
        \item \texttt{date}
        \item \texttt{time}
        \item \texttt{datetime}
    \end{itemize}

    Ein Cast in verschiedene Typen ist wie folgt möglich:\footnote{\texttt{NOW()} gibt das aktuelle Datum in der Formatierung \texttt{session.TIME\_ZONE} zurück}

    \begin{lstlisting}[language=mysql]
        SELECT
            NOW() AS 'Session',
            cast(NOW() AS DATE) AS 'Date',
            cast(NOW() AS TIME) AS 'Time',
            cast(NOW() AS DATETIME) AS 'Datetime';
    \end{lstlisting}

    \setcounter{rownum}{0}
    \begin{tabular}{I|TTTT}
        \rowcolor{gray!35}
        & \multicolumn{1}{T}{Session} & \multicolumn{1}{T}{Date} & \multicolumn{1}{T}{Time} & \multicolumn{1}{T}{Datetime} \\\hline
        1 & 1996-02-27 10:00:00 & 1996-02-27 & 10:00:00 & 1996-02-27 10:00:00 \\
    \end{tabular}

    Zusätzlich ist es ebenfalls möglich von Strings zu einem Datum zu casten:

    \begin{lstlisting}[language=mysql]
        SELECT
            STR_TO_DATE('27.02.1996', GET_FORMAT(DATE, 'EUR')) AS 'STR_TO_DATE',
            DATE_FORMAT(NOW(), GET_FORMAT(DATE, 'EUR')) AS 'DATE',
            -- %W = Wochentag, %D Tag im Monat, %M = Monat, %Y = Jahr
            DATE_FORMAT(NOW(), '%W, %D %M %Y') AS 'CUSTOMIZED DATE';
    \end{lstlisting}

    \setcounter{rownum}{0}
    \begin{tabular}{I|TTT}
        \rowcolor{gray!35}
        & \multicolumn{1}{T}{STR\_TO\_DATE} & \multicolumn{1}{T}{DATE} & \multicolumn{1}{T}{CUSTOMIZED\_DATE} \\\hline
        1 & 1996-02-27 & 27.02.1996 & Tuesday, 27th February 1996 \\
    \end{tabular}

    Ein Datum kann über \lstinline[language=mysql]{DATE_ADD()}, oder per \texttt{+} bzw. \texttt{-}, modifiziert werden:

    \begin{lstlisting}[language=mysql]
        SELECT
            NOW() !+! INTERVAL 1 YEAR,
            !DATE_ADD!(NOW(), INTERVAL 1 YEAR);
    \end{lstlisting}

    Alternativ kann ebenfalls eine Zeitspanne ermittelt werden:

    \begin{lstlisting}[language=mysql]
        SELECT
            DATEDIFF(DATE_ADD(NOW(), INTERVAL 1 YEAR), NOW()),
            TIMEDIFF(DATE_ADD(NOW(), INTERVAL 1 HOUR), NOW());
    \end{lstlisting}
\end{sql}

\begin{sql}{ALTER TABLE}
    \texttt{ALTER TABLE} modifiziert eine Tabelle.

    \texttt{ADD} wird genutzt, um Spalten hinzuzufügen.
    Die Syntax ist dabei identisch zu der Syntax beim Erstellen der Tabelle.

    \begin{lstlisting}[language=mysql]
        ALTER TABLE <Tabellenname>
        ADD (
            <Spaltenname> [<Parameter>, ...],
            ...
        );
    \end{lstlisting}

    \texttt{MODIFY} verändert existierende Spalten.

    \begin{lstlisting}[language=mysql]
        ALTER TABLE <Tabellenname>
        MODIFY <Spaltenname> [<Neuer Spaltenname>] [<Neuer Parameter>, ...];
    \end{lstlisting}

    \texttt{DROP} löscht angegebene Spalten.

    \begin{lstlisting}[language=mysql]
        ALTER TABLE <Tabellenname>
        DROP <Spaltenname>;
    \end{lstlisting}

    \texttt{RENAME TO} benennt Tabellen um.

    \begin{lstlisting}[language=mysql]
        ALTER TABLE <Tabellenname>
        RENAME TO <neuer Tabellenname>;
    \end{lstlisting}
\end{sql}

\subsection{Beispiele}

\begin{example}{Schemas und Tabellen}
    Die SQL Befehle, welche zur Erstellung unserer Beispieldatenbank genutzt wurden sind in \texttt{pokemon.sql} zu finden.

    Einige relevante Befehle für die DBMS-Struktur sind hier zusammengefasst:

    Falls \texttt{pokemon} existiert wird diese Tabelle gelöscht:

    \begin{lstlisting}[language=mysql]
        DROP TABLE IF EXISTS `pokemon`;
    \end{lstlisting}

    Im Nachhinein wird eine neue Tabelle erstellt:

    \begin{lstlisting}[language=mysql]
        CREATE TABLE `pokemon` (
            `ID` int(11) NOT NULL,
            `Name` varchar(255) NOT NULL,
            `Groesse` float NOT NULL,
            `Gewicht` float NOT NULL,
            `Generation` int(11) NOT NULL,
            `PrimaerTyp` varchar(255) DEFAULT NULL,
            `SekundaerTyp` varchar(255) DEFAULT NULL,
            PRIMARY KEY (`ID`),
            FOREIGN KEY (`PrimaerTyp`) REFERENCES `typ` (`Bezeichnung`),
            FOREIGN KEY (`SekundaerTyp`) REFERENCES `typ` (`Bezeichnung`),
            FOREIGN KEY (`Generation`) REFERENCES `generation` (`ID`)
        )
    \end{lstlisting}
\end{example}
\section{Views}
\section{Daten anlegen, verändern und löschen}
\section{Zeit- und Datumsfunktionen}

\begin{sql}{Zeit- und Datumsfunktionen}
    MySQL nutzt \texttt{session} und \texttt{global} Konfigurationen.

    Die \texttt{session} Konfiguration wird beim Abschalten der Datenbank auf \texttt{global} zurückgesetzt.

    \begin{lstlisting}[language=mysql]
        SELECT @@session.time_zone, @@global.time_zone;;
    \end{lstlisting}

    \setcounter{rownum}{0}
    \begin{tabular}{I|TT}
        \rowcolor{gray!35}
        & \multicolumn{1}{T}{@@session.time\_zone} & \multicolumn{1}{T}{@@global.time\_zone} \\\hline
        1 & SYSTEM & SYSTEM \\
    \end{tabular}

    Die Zeitzone kann mit folgendem Befehl angepasst werden:

    \begin{lstlisting}[language=mysql]
        SET SESSION TIME_ZONE = <Zeitzone>
    \end{lstlisting}

    Dabei kann als Zeitzone folgendes eingestellt werden:

    \begin{itemize}
        \item Eine exakte Zeitzone: \lstinline[language=mysql]{'Europe/Berlin'}
        \item Ein Offset: \lstinline[language=mysql]{'+02:00'}
    \end{itemize}

    Einige Datumstypen sind:

    \begin{itemize}
        \item \texttt{date}
        \item \texttt{time}
        \item \texttt{datetime}
    \end{itemize}

    Ein Cast in verschiedene Typen ist wie folgt möglich:

    \begin{lstlisting}[language=mysql]
        SELECT
            NOW() AS 'Session',
            cast(NOW() AS DATE) AS 'Date',
            cast(NOW() AS TIME) AS 'Time',
            cast(NOW() AS DATETIME) AS 'Datetime';
    \end{lstlisting}

    \setcounter{rownum}{0}
    \begin{tabular}{I|TTTT}
        \rowcolor{gray!35}
        & \multicolumn{1}{T}{Session} & \multicolumn{1}{T}{Date} & \multicolumn{1}{T}{Time} & \multicolumn{1}{T}{Datetime} \\\hline
        1 & 1996-02-27 10:00:00 & 1996-02-27 & 10:00:00 & 1996-02-27 10:00:00 \\
    \end{tabular}

    Zusätzlich ist es ebenfalls möglich von Strings zu einem Datum zu casten:

    \begin{lstlisting}[language=mysql]
        SELECT
            STR_TO_DATE('27.02.1996', GET_FORMAT(DATE, 'EUR')) AS 'STR_TO_DATE',
            DATE_FORMAT(NOW(), GET_FORMAT(DATE, 'EUR')) AS 'DATE',
            -- %W = Wochentag, %D Tag im Monat, %M = Monat, %Y = Jahr
            DATE_FORMAT(NOW(), '%W, %D %M %Y') AS 'CUSTOMIZED DATE';
    \end{lstlisting}

    \setcounter{rownum}{0}
    \begin{tabular}{I|TTT}
        \rowcolor{gray!35}
        & \multicolumn{1}{T}{STR\_TO\_DATE} & \multicolumn{1}{T}{DATE} & \multicolumn{1}{T}{CUSTOMIZED\_DATE} \\\hline
        1 & 1996-02-27 & 27.02.1996 & Tuesday, 27th February 1996 \\
    \end{tabular}

    Ein Datum kann über \lstinline[language=mysql]{DATE_ADD()}, oder per \texttt{+} bzw. \texttt{-}, modifiziert werden:

    \begin{lstlisting}[language=mysql]
        SELECT
            NOW() !+! INTERVAL 1 YEAR,
            NOW() + INTERVAL 1 HOUR,
            !DATE_ADD!(NOW(), INTERVAL 1 YEAR);
    \end{lstlisting}

    Alternativ kann ebenfalls eine Zeitspanne ermittelt werden:

    \begin{lstlisting}[language=mysql]
        SELECT
            DATEDIFF(DATE_ADD(NOW(), INTERVAL 1 YEAR), NOW()),
            TIMEDIFF(DATE_ADD(NOW(), INTERVAL 1 HOUR), NOW());
    \end{lstlisting}
\end{sql}

\begin{sql}{Rekursive Abfragen}
    Wenn man sich eine Entwicklungskette ausgeben lassen möchte, muss man einen Startpunkt angeben.
    Aus diesem wird eine Ursprungsabfrage erstellt und mit einer Rekursionsabfrage kombiniert.

    \begin{lstlisting}[language=mysql]
        -- Rekursive Abfrage
        WITH RECURSIVE !r_entwicklung!(von, zu) AS (
            (
                -- Ursprungsabfrage
                SELECT von, zu
                FROM entwicklung
                WHERE von = (
                    -- Startpunkt
                    SELECT ID
                    FROM pokemon
                    WHERE Name = 'Glumanda'
                )
            ) UNION ALL (
                -- Rekursionsabfrage
                SELECT entwicklung.von, entwicklung.zu
                FROM !r_entwicklung!
                JOIN entwicklung
                    ON r_entwicklung.Zu = entwicklung.Von
            )
        )
        -- Aufruf der rekursiven Abfrage
        SELECT
            pokemon1.ID,
            pokemon1.Name,
            pokemon2.ID,
            pokemon2.Name
        FROM r_entwicklung
        -- Join um Namen anzeigen zu koennen
        JOIN pokemon AS pokemon1
            ON r_entwicklung.Von = pokemon1.ID
        JOIN pokemon AS pokemon2
            ON r_entwicklung.Zu = pokemon2.ID;
    \end{lstlisting}
\end{sql}

\printindex
\printindex[Beispiele]

\printbibliography
\end{document}