\section{Daten anlegen, verändern und löschen}

\begin{sql}{INSERT INTO}
    \texttt{INSERT INTO} trägt Daten in eine Datenbank ein.

    \begin{lstlisting}[language=mysql]
        INSERT INTO <Tabellenname>
            (<Spaltenname>, ...)
        VALUES
            (Wert, ...),
            ...;
    \end{lstlisting}

    Wenn man alle Attribute in der vorgegebenen Reihenfolge angibt, kann man die Angabe der Spalten weglassen.

    Es ist ebenfalls möglich \texttt{Subselects} zum Herausfinden von Daten mit \texttt{INSERT INTO} zu verwenden.
\end{sql}

\begin{example}{INSERT INTO}
    Hier wird die Entität \texttt{Glumanda} erstellt:

    \begin{lstlisting}[language=mysql]
        INSERT INTO pokemon
            (ID, Name, Groesse, Gewicht, Generation, PrimaerTyp, SekundaerTyp)
        VALUES
            (4, 'Glumanda', 0.6, 8.5, 1, 'Feuer', NULL);
    \end{lstlisting}

    \begin{lstlisting}[language=mysql]
        INSERT INTO pokemon
        VALUES
            (4, 'Glumanda', 0.6, 8.5, 1, 'Feuer', NULL);
    \end{lstlisting}
\end{example}

\begin{example}{INSERT INTO}
    Da \texttt{NULL} der Standard Wert für SekundaerTyp ist, und Glumanda keinen Sekundärtyp hat, kann man \texttt{Glumanda} ebenfalls wie folgt erstellen:
    
    \begin{lstlisting}[language=mysql]
        INSERT INTO pokemon
            -- kein SekundaerTyp
            (ID, Name, Groesse, Gewicht, Generation, PrimaerTyp)
        VALUES
            (4, 'Glumanda', 0.6, 8.5, 1, 'Feuer');
    \end{lstlisting}
\end{example}

\begin{example}{INSERT INTO}
    Mehrere Pokemon ließen sich wie folgt gleichzeitig erstellen:

    \begin{lstlisting}[language=mysql]
        INSERT INTO pokemon
        VALUES
            (5, 'Glutexo', 1.1, 19, 1 'Feuer', NULL),
            (6, 'Glurak', 1.7, 90.5, 1 'Feuer', 'Flug');
    \end{lstlisting}
\end{example}

\begin{sql}{UPDATE}
    Um Werte anzupassen nutzt man \texttt{UPDATE}:

    \begin{lstlisting}[language=mysql]
        UPDATE <Tabellenname>
        SET
            <Spaltenname> = <Wert>,
            ...
        WHERE <Bedingung>
    \end{lstlisting}

    Wenn man also den Typ von Glumanda auf Wasser setzen wollen würde, nutzt man:

    \begin{lstlisting}[language=mysql]
        UPDATE pokemon
        SET PrimaerTyp = 'Wasser'
        WHERE Name = 'Glumanda'
    \end{lstlisting}

    \texttt{UPDATE} verändert alle Entitäten, auf die die Bedingung zutrifft.
    
    Dementsprechend sollte man, wenn man eine einzige bestimmte Entität anpassen möchte, den Primärschlüssel bzw. \texttt{UNIQUE} Felder abfragen.
\end{sql}

\begin{sql}{DROP}
    \texttt{DROP} bzw. \texttt{DELETE} wird genutzt um spezifische Entitäten zu löschen.

    \begin{lstlisting}[language=mysql]
        DROP FROM <Tabellenname>
        [WHERE <Bedingung>];
    \end{lstlisting}

    Möchte man nun Schiggy löschen, nutzt man:

    \begin{lstlisting}[language=mysql]
        DROP FROM pokemon
        WHERE Name = 'Schiggy';
    \end{lstlisting}
\end{sql}

\begin{sql}{TRUNCATE}
    \texttt{TRUNCATE} löscht Entitäten einer Tabelle, lässt die Tabelle an sich jedoch bestehen.

    \begin{lstlisting}[language=mysql]
        TRUNCATE TABLE <Tabellenname>        
    \end{lstlisting}
\end{sql}

\begin{bonus}{LOCK TABLES}
    Wenn man mit mehreren Benutzenden an einer Tabelle arbeitet, ist es sinnvoll eine Tabelle zu sperren.
    So kann bis zum freigeben der Tabelle niemand die Tabelle bearbeiten.

    \begin{lstlisting}[language=mysql]
        LOCK TABLES <Tabellenname> WRITE;
    \end{lstlisting}

    \begin{lstlisting}[language=mysql]
        UNLOCK TABLES;
    \end{lstlisting}
\end{bonus}
   
\subsection{Beispiele}

\begin{example}{INSERT INTO}
    Die SQL Befehle, welche zur Erstellung unserer Beispieldatenbank genutzt wurden sind in \texttt{pokemon.sql} zu finden.

    Hier ein Auszug, durch den die Ersten Pokemon erstellt wurden:

    \begin{lstlisting}[language=mysql]
        INSERT INTO `pokemon`
        VALUES
            (1,'Bisasam',0.7,6.9,1,'Pflanze','Gift'),
            (2,'Bisaknosp',1,13,1,'Pflanze','Gift'),
            (3,'Bisaflor',2,100,1,'Pflanze','Gift'),
            (4,'Glumanda',0.6,8.5,1,'Feuer',NULL),
            (5,'Glutexo',1.1,19,1,'Feuer',NULL),
            (6,'Glurak',1.7,90.5,1,'Feuer','Flug'),
            (7,'Schiggy',0.5,9,1,'Wasser',NULL),
            (8,'Schillok',1,22.5,1,'Wasser',NULL),
            (9,'Turtok',1.6,85.5,1,'Wasser',NULL);
    \end{lstlisting}
\end{example}

\begin{example}{UPDATE}
    Erhöhen Sie das Gewicht aller Pokémon um 5kg und verringern Sie gleichzeitig die Größe um 25cm.
    
    Sollte dabei die Größe unter 1cm fallen, bleibt diese bei diesem Wert.

    \exampleseparator

    \begin{lstlisting}[language=mysql]
        UPDATE pokemon
        SET
            Gewicht = Gewicht + 5,
            Groesse = MIN(Groesse - 0.25, 0.01)
    \end{lstlisting}
\end{example}

\begin{example}{UPDATE}
    Um die Spiele fair zu halten sollen alle Attacken auf die Durchschnittsstärke aller Attacken mit dem selben Typen gesetzt werden.

    \exampleseparator

    \begin{lstlisting}[language=mysql]
        UPDATE attacke AS attacke1
        SET Staerkte = (
            SELECT ROUND(AVG(Staerke), 0)
            FROM attacke AS attacke2
            WHERE attacke1.Typ = attacke2.Typ 
        );
    \end{lstlisting}
\end{example}