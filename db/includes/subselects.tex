\section{Subselects}

\begin{defi}{Unterabfrage}
    \emph{Unterabfragen} bzw. Subselects sind verschachtelte Abfragen.
    Dabei stellen die inneren Abfragen Daten bereit, welche von den Äußeren genutzt werden.
    
    Unterabfragen werden genutzt, wenn die zu vergleichenden Werte sich dynamisch verändern oder noch unbekannt sind.

    Unterabfragen lassen sich in verschiedene Kategorien aufteilen:

    \begin{itemize}
        \item Einzeilige bzw. mehrzeilige Unterabfragen
        \item Korrelierte bzw. unkorrelierte Unterabfragen
        \item Skalare bzw. nicht skalare Unterabfragen
    \end{itemize}

    Wenn man von Unterabfragen spricht, so meint man meist nicht korrelierte, skalare Unterabfragen, da diese die simpelste Implementierung sind.
\end{defi}

\begin{example}{Unterabfrage}
    So könnte man z.B. alle \texttt{ID}s von Pokémon, welche sich aus Glumanda entwickeln, folgendermaßen herausfinden:

    Zuerst müssen wir die ID von Glumanda herausfinden:

    \begin{lstlisting}[style=SqlInputStyle]
        -- ID von 'Glumanda' herausfinden
        SELECT ID
        FROM pokemon
        WHERE Name = 'Glumanda';
    \end{lstlisting}

    \begin{tabular}{I|I}
        \rowcolor{gray!35}
        & \multicolumn{1}{T}{ID} \\\hline
        1 & 4 \\
    \end{tabular}

    Diese nutzen wir nun manuell um alle Entwicklungen herauszufinden:

    \begin{lstlisting}[style=SqlInputStyle]
        -- ID von 'Glumanda' nutzen um alle Entwicklungen herauszufinden
        SELECT *
        FROM entwicklung
        WHERE Von = 4;
    \end{lstlisting}

    \begin{tabular}{I|IIIITTT}
        \rowcolor{gray!35}
        & \multicolumn{1}{T}{Von} & \multicolumn{1}{T}{Zu} & \multicolumn{1}{T}{Level} & \multicolumn{1}{T}{Item} & \multicolumn{1}{T}{GetragenesItem} & \multicolumn{1}{T}{Tageszeit} \\\hline
        1 & 4 & 5 & 16 & NULL & NULL & NULL \\
    \end{tabular}

    Diese beiden Abfragen lassen sich nun zu einer Abfrage zusammenfassen:

    \begin{lstlisting}[style=SqlInputStyle]
        -- ID von 'Glumanda' nutzen um alle Entwicklungen herauszufinden
        SELECT *
        FROM entwicklung
        WHERE Von = (
            -- ID von 'Glumanda' herausfinden
            SELECT ID
            FROM pokemon
            WHERE Name = 'Glumanda'
        );
    \end{lstlisting}
\end{example}

\begin{defi}{Mehrzeilige Unterfrage}
    In \emph{mehrzeiligen Unterabfragen} geben die inneren Abfragen mehrere Entitäten zurück.
    
    So kann man z.B. nicht auf exakte Gleichheit überprüfen, sondern muss Mengenoperationen (wie z.B. \texttt{IN}) nutzen.
\end{defi}

\begin{example}{Unterabfrage (mehrzeilig)}
    Das Resultat des ersten Beispiels nutzen wir nun um den Namen der Entwicklung herauszufinden:

    \begin{lstlisting}[style=SqlInputStyle]
        SELECT *
        FROM pokemon
        WHERE ID = (
            -- Eine Entitaet
            SELECT Zu
            FROM entwicklung
            WHERE Von = (
                SELECT ID
                FROM pokemon
                WHERE Name = 'Glumanda'
            )
        );
    \end{lstlisting}

    \begin{tabular}{I|ITIIITT}
        \rowcolor{gray!35}
        & \multicolumn{1}{T}{ID} & \multicolumn{1}{T}{Name} & \multicolumn{1}{T}{Groesse} & \multicolumn{1}{T}{Gewicht} & \multicolumn{1}{T}{Generation} & \multicolumn{1}{T}{PrimaerTyp} & \multicolumn{1}{T}{SekundaerTyp} \\\hline
        1 & 5 & Glutexo & 1.1 & 19 & 1 & Feuer & NULL \\
    \end{tabular}

    Sollten wir nun alternativ die Entwicklungen von Evoli anzeigen lassen, können wir die Bedingung \texttt{=} nicht benutzen, da die Abfrage aus \texttt{entwicklung} eine Menge an Entitäten zurück gibt.
    Dementsprechend muss man das bekannte \texttt{IN} verwenden.

    \begin{lstlisting}[style=SqlInputStyle]
        SELECT *
        FROM pokemon
        WHERE ID IN (
            -- Eine Menge an Entitaeten
            SELECT Zu
            FROM entwicklung
            WHERE Von = (
                SELECT ID
                FROM pokemon
                WHERE Name = 'Evoli'
            )
        );
    \end{lstlisting}

    \begin{tabular}{I|ITIIITT}
        \rowcolor{gray!35}
        & \multicolumn{1}{T}{ID} & \multicolumn{1}{T}{Name} & \multicolumn{1}{T}{Groesse} & \multicolumn{1}{T}{Gewicht} & \multicolumn{1}{T}{Generation} & \multicolumn{1}{T}{PrimaerTyp} & \multicolumn{1}{T}{SekundaerTyp} \\\hline
        1 & 134 & Aquana & 1 & 29 & 1 & Wasser & NULL \\
        2 & 135 & Blitza & 0.8 & 24.5 & 1 & Elektro & NULL \\
        3 & 136 & Flamara & 0.9 & 25 & 1 & Feuer & NULL \\
        4 & 196 & Psiana & 0.9 & 26.5 & 2 & Psycho & NULL \\
        5 & 197 & Nachtara & 1 & 27 & 2 & Unlicht & NULL \\
        6 & 470 & Folipurba & 1 & 25.5 & 4 & Pflanze & NULL \\
        7 & 471 & Glaziola & 0.8 & 25.9 & 4 & Eis & NULL \\
        8 & 700 & Feelinara & 1 & 23.5 & 6 & Fee & NULL \\
    \end{tabular}
\end{example}

\begin{defi}{Korrelierte Unterabfrage}
    In \emph{korrelierten Unterabfragen} (\enquote{Correlated Subselects}) verweist die innere Abfragen auf die Äußere.\footnote{Dies ist vergleichbar mit einer Variable, welche in einem inneren Scope erneut genutzt wird.}

    Da die Bedingung für jede Entität neu überprüft werden muss, benötigen korrelierte Unterabfragen mehr Rechenleistung, dementsprechend haben diese eine längere Laufzeit.
\end{defi}

\begin{example}{Unterabfrage (korreliert)}
    In folgender Abfrage wird jede Attacke ausgegeben, welche eine höhere Stärke besitzt, als der Durchschnitt aller Attacken mit dem selben Typ.

    \begin{lstlisting}[style=SqlInputStyle]
        SELECT
            ID,
            Name,
            Staerke,
            Typ
        FROM attacke AS aeussereAttacke
        WHERE Staerke > (
            SELECT AVG(Staerke)
            FROM attacke AS innereAttacke
            -- attacke.Typ unterscheidet sich potentiell fuer jede Entitaet
            WHERE aeussereAttacke.Typ = innereAttacke.Typ
        );
    \end{lstlisting}

    \begin{tabular}{I|TTTT}
        \rowcolor{gray!35}
        & \multicolumn{1}{T}{ID} & \multicolumn{1}{T}{Name} & \multicolumn{1}{T}{Staerke} & \multicolumn{1}{T}{Typ} \\\hline
        1 & 5 & Megahieb & 80 & Normal \\
        2 & 8 & Eishieb & 75 & Eis \\
        3 & 13 & Klingensturm & 80 & Normal \\
        \multicolumn{1}{c|}{\dots} & \multicolumn{4}{c}{\dots} \\
        230 & 825 & Astralfragmente & 120 & Geist \\
    \end{tabular}
\end{example}

\begin{defi}{Nicht skalare Unterabfrage}
    In \emph{nicht skalaren Unterabfragen} werden mehrere Attribute der Entitäten in der Unterabfrage genutzt.
    So müssen mehrere Tupel auf Gleichheit überprüft werden.
    
    Sollte eine Unterabfrage nicht einzeilig sein, so ist diese automatisch nicht skalar.
    Dementsprechend ist eine Unterabfrage ausschließlich dann skalar, wenn man exakt eine Entität mit exakt einem Attribut erhält.
\end{defi}

\begin{example}{Unterabfrage (nicht skalar)}
    Möchte man sich beispielsweise alle Pokémon ausgeben, welche exakt die selbe Typkombination wie Bisasam haben, so nutzt man:

    \begin{lstlisting}[style=SqlInputStyle]
        SELECT *
        FROM pokemon
        WHERE
            (PrimaerTyp, SekundaerTyp) = (
                -- Entspricht einem Tupel aus 2 Typen
                SELECT
                    PrimaerTyp,
                    SekundaerTyp
                FROM pokemon
                WHERE Name = 'Bisasam'
            )
            AND Name <> 'Bisasam';
    \end{lstlisting}

    \begin{tabular}{I|ITIIITT}
        \rowcolor{gray!35}
        & \multicolumn{1}{T}{ID} & \multicolumn{1}{T}{Name} & \multicolumn{1}{T}{Groesse} & \multicolumn{1}{T}{Gewicht} & \multicolumn{1}{T}{Generation} & \multicolumn{1}{T}{PrimaerTyp} & \multicolumn{1}{T}{SekundaerTyp} \\\hline
        1 & 2 & Bisaknosp & 1 & 13 & 1 & Pflanze & Gift \\
        2 & 3 & Bisaflor & 2 & 100 & 1 & Pflanze & Gift \\
        3 & 43 & Myrapla & 0.5 & 5.4 & 1 & Pflanze & Gift \\
        \multicolumn{1}{c|}{\dots} & \multicolumn{7}{c}{\dots} \\
        13 & 591 & Hutsassa & 0.6 & 10.5 & 5 & Pflanze & Gift \\
    \end{tabular}
\end{example}

\begin{sql}{ANY}
    \begin{itemize}
        \item \texttt{< ANY} gibt Einträge zurück, welche kleiner als ein beliebiges Element\footnote{d.h. kleiner als das Maximum} einer Menge sind.
        \item \texttt{> ANY} gibt Einträge zurück, welche größer als ein beliebiges Element\footnote{d.h. größter als das Minimum} einer Menge sind.
        \item \texttt{= ANY} gibt Einträge zurück, welche mit einem Werte einer Menge übereinstimmen.
            Diese Bedingung entspricht dem Keyword \texttt{IN}.
    \end{itemize}
\end{sql} 

\begin{sql}{ALL}
    \begin{itemize}
        \item \texttt{< ALL} gibt Einträge zurück, welche kleiner als alle Elemente\footnote{d.h. kleiner als das Minimum} einer Menge sind.
        \item \texttt{> ALL} gibt Einträge zurück, welche größer als alle Elemente\footnote{d.h. größter als das Maximum} einer Menge sind.
    \end{itemize}
\end{sql}