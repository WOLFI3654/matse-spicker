\section{Join}

\begin{defi}{Kartesisches Produkt}
    Wenn man alle Entitäten einer Tabelle mit jeder Entität aus einer anderen Tabelle kombinieren möchte, benötigt man das \emph{kartesische Produkt}.

    Möchte man jede mögliche Typkombination\footnote{Sollte man das kartesische Produkt einer Tabelle mit sich selbst bilden, benötigt man Aliase für die Tabellen} erhalten, nutzt man \texttt{SELECT} zwei mal auf \texttt{Typ}:
    
    \begin{lstlisting}[language=mysql]
        SELECT *
        FROM
            typ AS typ1,
            typ AS typ2;
    \end{lstlisting}

    \setcounter{rownum}{0}
    \begin{tabular}{I|TT}
        \rowcolor{gray!35}
        & \multicolumn{1}{T}{typ1.Bezeichnung} & \multicolumn{1}{T}{typ2.Bezeichnung} \\\hline
        1 & Boden & Boden \\
        2 & Drache & Boden \\
        3 & Eis & Boden \\
        \multicolumn{1}{c|}{\dots} & \multicolumn{2}{c}{\dots} \\
        19 & Boden & Drache \\
        20 & Drache & Drache \\
        21 & Eis & Drache \\
        \multicolumn{1}{c|}{\dots} & \multicolumn{2}{c}{\dots} \\
        324 & Wasser & Wasser \\
    \end{tabular}

    Wenn eine Relation verschiedener Tabellen über einen Fremdschlüssel realisiert ist, kann man Entitäten aus diesen Tabellen zueinander zuordnen.
    So wird nicht nur der Fremdschlüssel angezeigt, sondern alle zugehörigen Attribute.
    
    Da wir die aus dem kartesischen Produkt erhaltene Menge noch sinnvoll selektieren müssen, nutzen wir \texttt{WHERE}.

    In unserem Beispiel kann kein Pokemon zwei mal den selben Typen besitzen:

    \begin{lstlisting}[language=mysql]
        SELECT *
        FROM
            typ AS typ1,
            typ AS typ2
        WHERE NOT typ1.Bezeichnung = typ2.Bezeichnung;
    \end{lstlisting}

    \setcounter{rownum}{0}
    \begin{tabular}{I|TT}
        \rowcolor{gray!35}
        & \multicolumn{1}{T}{typ1.Bezeichnung} & \multicolumn{1}{T}{typ2.Bezeichnung} \\\hline
        1 & Drache & Boden \\
        2 & Eis & Boden \\
        3 & Elektro & Boden \\
        \multicolumn{1}{c|}{\dots} & \multicolumn{2}{c}{\dots} \\
        18 & Boden & Drache \\
        19 & Eis & Drache \\
        20 & Elektro & Drache \\
        \multicolumn{1}{c|}{\dots} & \multicolumn{2}{c}{\dots} \\
        306 & Unlicht & Wasser
    \end{tabular}
\end{defi}

\begin{sql}{JOIN}
    Das selbe Ergebnis erhält man, wenn man \texttt{JOIN} mit dem Schlüsselwort \texttt{ON} nutzt:\footnote{
        Hierbei unterscheiden sich die beiden Abfragen ausschließlich in der Lesbarkeit und Optimierungsmöglichkeiten für das DBMS.
        Bei \texttt{JOIN} muss das DBMS nicht alle Kombinationen ermitteln, nur um diese dann zu filtern.
    }

    \begin{lstlisting}[language=mysql]
        SELECT *
        FROM typ AS typ1
        JOIN typ AS typ2
            ON NOT typ1.Bezeichnung = typ2.Bezeichnung;
    \end{lstlisting}
\end{sql}

\begin{sql}{CROSS JOIN}
    \texttt{CROSS JOIN} gibt das unselektierte kartesische Produkt aus.

    Dementsprechend sind die folgenden Abfragen erneut äquivalent:

    \begin{lstlisting}[language=mysql]
        SELECT *
        FROM typ AS typ1
        CROSS JOIN typ AS typ2;
    \end{lstlisting}
        
    \begin{lstlisting}[language=mysql]
        SELECT *
        FROM typ AS typ1
        JOIN typ as typ2
            ON TRUE;
    \end{lstlisting}
    
    \begin{lstlisting}[language=mysql]
        SELECT *
        FROM
            typ AS typ1,
            typ AS typ2;
    \end{lstlisting}
\end{sql}

\begin{sql}{INNER JOIN}
    \texttt{INNER JOIN} wird automatisch genutzt, wenn man \texttt{JOIN} nutzt.

    Dementsprechend sind folgende abfragen äquivalent:

    \begin{lstlisting}[language=mysql]
        SELECT *
        FROM typ AS typ1
        JOIN typ AS typ2
            ON NOT typ1.Bezeichnung = typ2.Bezeichnung;
    \end{lstlisting}

    \begin{lstlisting}[language=mysql]
        SELECT *
        FROM typ AS typ1
        !INNER! JOIN typ AS typ2
            ON NOT typ1.Bezeichnung = typ2.Bezeichnung;
    \end{lstlisting}

    Wenn man sich beispielsweise alle Pokémon samt ihrer Entwicklung anzeigen lassen möchte, so erhält man:

    \begin{lstlisting}[language=mysql]
        SELECT
            pokemon.ID,
            pokemon.Name,
            entwicklung.Von,
            entwicklung.Zu
        FROM pokemon
        JOIN entwicklung
            ON pokemon.ID = entwicklung.Von;
    \end{lstlisting}

    \setcounter{rownum}{0}
    \begin{tabular}{I|ITII}
        \rowcolor{gray!35}
        & \multicolumn{1}{T}{ID} & \multicolumn{1}{T}{Name} & \multicolumn{1}{T}{Von} & \multicolumn{1}{T}{Zu} \\\hline
        1 & 1 & Bisasam & 1 & 2 \\
        2 & 2 & Bisaknosp & 2 & 3 \\
        3 & 4 & Glumanda & 4 & 5 \\
        \multicolumn{1}{c|}{\dots} & \multicolumn{4}{c}{\dots} \\
        429 & 891 & Dakuma & 891 & 892 \\
    \end{tabular}
\end{sql}

\begin{defi}{SELF JOIN}
    Als \texttt{SELF JOIN} werden \texttt{JOIN}s bezeichnet, welche eine reflexive Fremdschlüsselbeziehung abfragen.
    
    Hierbei ist zu beachten, dass \texttt{SELF} kein Keyword ist, sondern die reine Bezeichnung für solche Relationen.

    Dies wurde mit dem Beispiel \texttt{typ} auf \texttt{typ} bereits dargestellt.
\end{defi}

\begin{defi}{OUTER JOIN}
    \texttt{OUTER JOIN} berücksichtigt ebenfalls Entitäten, für die kein passendes Pendant existiert.
    Für diese Entitäten sind alle Attribute der jeweils anderen Entitätenmenge \texttt{NULL}.

    Ebenfalls \texttt{OUTER JOIN} ist keine gültige Abfrage.
    Das Keyword \texttt{OUTER} benötigt immer eine Leserichtung.

    Wenn wir erneut das Beispiel von \texttt{INNER JOIN} betrachten, sehen wir, dass das Pokémon mit der ID 3 nicht vorkommt.
    Dies ist dem Fakt geschuldet, dass dieses Pokémon keine Entwicklung besitzt.
    Sollte man dieses Pokemon trotzdem angezeigt bekommen, benötigt man \texttt{OUTER JOIN}.
\end{defi}

\begin{sql}{LEFT OUTER JOIN}
    \texttt{LEFT OUTER JOIN} beachtet alle Entitäten \emph{links} der Relation.

    Nun werden alle Pokémon dargestellt, ggf. mit \texttt{NULL}-Attributen bei Entwicklung:

    \begin{lstlisting}[language=mysql]
        SELECT
            pokemon.ID,
            pokemon.Name,
            entwicklung.Von,
            entwicklung.Zu
        FROM pokemon
        LEFT OUTER JOIN entwicklung
            ON pokemon.id = entwicklung.von;
    \end{lstlisting}

    \setcounter{rownum}{0}
    \begin{tabular}{I|ITII}
        \rowcolor{gray!35}
        & \multicolumn{1}{T}{ID} & \multicolumn{1}{T}{Name} & \multicolumn{1}{T}{Von} & \multicolumn{1}{T}{Zu} \\\hline
        1 & 1 & Bisasam & 1 & 2 \\
        2 & 2 & Bisaknosp & 2 & 3 \\
        3 & 3 & Bisaflor & NULL & NULL \\
        \multicolumn{1}{c|}{\dots} & \multicolumn{4}{c}{\dots} \\
        920 & 898 & Coronospa & NULL & NULL \\
    \end{tabular}
\end{sql}

\begin{sql}{RIGHT OUTER JOIN}
    \texttt{RIGHT OUTER JOIN} beachtet alle Entitäten \emph{rechts} der Relation.

    Durch Vertauschen von Entitätstypen und Leserichtung erhält man äquivalente Befehle.

    So würde unser Beispiel aus \texttt{LEFT OUTER JOIN} folgendermaßen aussehen:

    \begin{lstlisting}[language=mysql]
        SELECT
            pokemon.ID,
            pokemon.Name,
            entwicklung.Von,
            entwicklung.Zu
        FROM entwicklung
        RIGHT OUTER JOIN pokemon
            ON pokemon.id = entwicklung.von;
    \end{lstlisting}
\end{sql}

\begin{sql}{FULL OUTER JOIN}
    \texttt{FULL OUTER JOIN} ist eine \emph{Vereinigung} von \texttt{LEFT OUTER JOIN} und \texttt{RIGHT OUTER JOIN}.
    Das bedeutet, es sind alle Entitäten beider Seiten dabei, beidseitig ggf. mit \texttt{NULL} aufgefüllt\footnote{
    Manche DBMS bieten keinen Befehl für einen \texttt{FULL OUTER JOIN}.
    In diesem Fall vereinigt man die Ergebnisse der beiden \texttt{OUTER JOIN}s mit \texttt{UNION}.}.

    \begin{lstlisting}[language=mysql]
        SELECT
            pokemon.ID,
            pokemon.Name,
            entwicklung.Von,
            entwicklung.Zu
        FROM pokemon
        !FULL OUTER! JOIN entwicklung
            ON pokemon.id = entwicklung.von;
    \end{lstlisting}

    \begin{lstlisting}[language=mysql]
        SELECT
            pokemon.ID,
            pokemon.Name,
            entwicklung.Von,
            entwicklung.Zu
        FROM pokemon
        !LEFT OUTER! JOIN entwicklung
            ON pokemon.id = entwicklung.von
        !UNION!
        SELECT
            pokemon.ID,
            pokemon.Name,
            entwicklung.Von,
            entwicklung.Zu
        FROM pokemon
        !RIGHT OUTER! JOIN entwicklung
            ON pokemon.id = entwicklung.von;
    \end{lstlisting}
\end{sql}

\begin{bonus}{NATURAL JOIN}
    Von der Verwendung von \texttt{NATURAL JOIN}s wird abgeraten!

    Bei Tabellen mit gleichnamigen Attributen kann auf die explizite Angabe der Bedingung verzichtet werden.
    Das DBMS verwendet dann \emph{alle} gleichnamigen Attribute.
\end{bonus}

\subsection{Beispiele}

\begin{example}{JOIN}
    Geben Sie für alle Pokemon jeweils die Entwicklung mit Level, sowie die Typen aus.

    Nutzen Sie:
    \begin{enumerate}[label=\alph*)]
        \item einen \texttt{INNER JOIN} mit \texttt{WHERE}
        \item einen \texttt{INNER JOIN} mit \texttt{JOIN}
    \end{enumerate}

    Fügen Sie noch den Namen, sowie die Typen der Entwicklung an.

    \exampleseparator

    \begin{enumerate}[label=\alph*)]
        \item \texttt{WHERE}
            
            \begin{lstlisting}[language=mysql]
                SELECT
                    von.ID,
                    von.Name,
                    von.PrimaerTyp,
                    von.SekundaerTyp,
                    entwicklung.Level,
                    zu.ID,
                    zu.Name
                FROM 
                    pokemon AS von,
                    pokemon AS zu,
                    entwicklung
                WHERE
                    von.ID = entwicklung.Von
                    AND entwicklung.Zu = zu.ID;
            \end{lstlisting}
        \item \texttt{JOIN}
        
            \begin{lstlisting}[language=mysql]
                SELECT
                    von.ID,
                    von.Name,
                    von.PrimaerTyp,
                    von.SekundaerTyp,
                    entwicklung.Level,
                    zu.ID,
                    zu.Name
                FROM pokemon
                AS von
                JOIN entwicklung
                    ON von.ID = entwicklung.Von
                JOIN pokemon AS zu
                    ON entwicklung.Zu = zu.ID;
            \end{lstlisting}
    \end{enumerate}
    
    \setcounter{rownum}{0}
    \begin{tabular}{I|ITTTIIT}
        \rowcolor{gray!35}
        & \multicolumn{1}{T}{von.ID} & \multicolumn{1}{T}{von.Name} & \multicolumn{1}{T}{PrimaerTyp} & \multicolumn{1}{T}{SekundaerTyp} & \multicolumn{1}{T}{Level} & \multicolumn{1}{T}{zu.ID} & \multicolumn{1}{T}{zu.Name} \\\hline
        1 & 1 & Bisasam & Pflanze & Gift & 16 & 2 & Bisaknosp \\
        2 & 2 & Bisaknosp & Pflanze & Gift & 32 & 3 & Bisaflor \\
        3 & 4 & Glumanda & Feuer & NULL & 16 & 5 & Glutexo \\
        \multicolumn{1}{c|}{\dots} & \multicolumn{7}{c}{\dots} \\
        429 & 891 & Dakuma & Kampf & NULL & NULL & 892 & Wulaosu \\
    \end{tabular}
\end{example}

\begin{example}{JOIN}
    Geben Sie zusätzlich auch Pokémon aus, die (noch) keine Entwicklung haben.

    \exampleseparator

    \begin{lstlisting}[language=mysql]
        SELECT
            von.ID,
            von.Name,
            von.PrimaerTyp,
            von.SekundaerTyp,
            entwicklung.Level,
            zu.ID,
            zu.Name
        FROM pokemon AS von
        LEFT OUTER JOIN entwicklung
            ON von.ID = entwicklung.Von
        LEFT OUTER JOIN pokemon AS zu
            ON entwicklung.Zu = zu.ID;
    \end{lstlisting}

    \setcounter{rownum}{0}
    \begin{tabular}{I|ITTTIIT}
        \rowcolor{gray!35}
        & \multicolumn{1}{T}{von.ID} & \multicolumn{1}{T}{von.Name} & \multicolumn{1}{T}{PrimaerTyp} & \multicolumn{1}{T}{SekundaerTyp} & \multicolumn{1}{T}{Level} & \multicolumn{1}{T}{zu.ID} & \multicolumn{1}{T}{zu.Name} \\\hline
        1 & 1 & Bisasam & Pflanze & Gift & 16 & 2 & Bisaknosp \\
        2 & 2 & Bisaknosp & Pflanze & Gift & 32 & 3 & Bisaflor \\
        3 & 3 & Bisaflor & Pflanze & Gift & NULL & NULL & NULL \\
        \multicolumn{1}{c|}{\dots} & \multicolumn{7}{c}{\dots} \\
        920 & 898 & Coronospa & Psycho & Pflanze & NULL & NULL & NULL \\
    \end{tabular}
\end{example}

\begin{example}{JOIN}
    Geben Sie alle Pokémon aus, welche bislang keine Entwicklung besitzen.

    \exampleseparator

    \begin{lstlisting}[language=mysql]
        SELECT
            pokemon.ID,
            pokemon.Name,
            pokemon.PrimaerTyp,
            pokemon.SekundaerTyp
        FROM pokemon
        LEFT OUTER JOIN entwicklung
            ON pokemon.ID = entwicklung.Von
        WHERE entwicklung.Von IS NULL;
    \end{lstlisting}

    \setcounter{rownum}{0}
    \begin{tabular}{I|ITTT}
        \rowcolor{gray!35}
        & \multicolumn{1}{T}{ID} & \multicolumn{1}{T}{Name} & \multicolumn{1}{T}{PrimaerTyp} & \multicolumn{1}{T}{SekundaerTyp} \\\hline
        1 & 3 & Bisaflor & Pflanze & Gift \\
        2 & 6 & Glurak & Feuer & Flug \\
        3 & 9 & Turtok & Wasser & NULL \\
        \multicolumn{1}{c|}{\dots} & \multicolumn{4}{c}{\dots} \\
        491 & 898 & Coronospa & Psycho & Pflanze \\
    \end{tabular}
\end{example}

\begin{example}{JOIN}
    Geben Sie alle Pokémon aus, welche sich aus Evoli entwickeln.

    \exampleseparator

    \begin{lstlisting}[language=mysql]
        SELECT
            zu.ID,
            zu.Name,
            zu.PrimaerTyp,
            zu.SekundaerTyp,
            von.ID,
            von.Name
        FROM pokemon AS zu
        JOIN entwicklung
            ON zu.ID = entwicklung.Zu
        JOIN pokemon AS von
            ON entwicklung.Von = von.ID
        WHERE von.Name = 'Evoli';
    \end{lstlisting}

    \setcounter{rownum}{0}
    \begin{tabular}{I|TTTTTT}
        \rowcolor{gray!35}
        & \multicolumn{1}{T}{von.ID} & \multicolumn{1}{T}{von.Name} & \multicolumn{1}{T}{PrimaerTyp} & \multicolumn{1}{T}{SekundaerTyp} & \multicolumn{1}{T}{zu.ID} & \multicolumn{1}{T}{zu.Name} \\\hline
        1 & 134 & Aquana & Wasser & NULL & 133 & Evoli \\
        2 & 135 & Blitza & Elektro & NULL & 133 & Evoli \\
        3 & 136 & Flamara & Feuer & NULL & 133 & Evoli \\
        4 & 196 & Psiana & Psycho & NULL & 133 & Evoli \\
        5 & 197 & Nachtara & Unlicht & NULL & 133 & Evoli \\
        6 & 470 & Folipurba & Pflanze & NULL & 133 & Evoli \\
        7 & 471 & Glaziola & Eis & NULL & 133 & Evoli \\
        8 & 700 & Feelinara & Fee & NULL & 133 & Evoli \\
    \end{tabular}
\end{example}