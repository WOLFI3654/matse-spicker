\section{Transaktionen}

\begin{defi}{Transaktion}
    In einer \emph{Transaktion} modifiziert der Benutzende die Daten bis zum \texttt{COMMIT} ausschließlich in seiner Session.
    
    In allen anderen Sessions ist die Datenbank unverändert.
\end{defi}

\begin{sql}{Transaktion}
    Um eine Transaktion zu starten, nutzt man:

    \begin{lstlisting}[language=mysql]
        START TRANSACTION;
    \end{lstlisting}

    Nun kann man Daten einfügen, löschen oder verändern bzw. die DBMS-spezifische Struktur anpassen.
    
    Sollten dabei etwaige Fehler auftreten, welche irreparable Schäden an der Datenbank verursachen, ist erst mal nur die Session der Verursachenden betroffen.

    Um die Transaktion zu abzuschließen, nutzt man:

    \begin{lstlisting}[language=mysql]
        COMMIT;
    \end{lstlisting}

    Alternativ bricht man die Transaktion folgendermaßen ab:

    \begin{lstlisting}[language=mysql]
        ROLLBACK;
    \end{lstlisting}
\end{sql}

\begin{sql}{autocommit}
    Um jeden Befehl sofort zu verarbeiten muss die Einstellung \texttt{autocommit} auf \texttt{1} gesetzt werden.
    
    Standardmäßig ist der Wert bereits auf \texttt{1}, jedoch nutzen wir um sicher zu gehen:

    \begin{lstlisting}[language=mysql]
        SET autocommit = 1;
    \end{lstlisting}

    Wenn der Wert auf \texttt{0} gesetzt wurde, werden Änderungen nur in der eigenen \texttt{session} vorgenommen.
    Wenn man seine Änderungen veröffentlichen möchte, nutzt man \texttt{COMMIT;}.
\end{sql}

\begin{sql}{foreign\_key\_checks}
    Der folgende Befehl schaltet das automatische Überprüfen von Fremdschlüssel-Beziehungen aus:

    \begin{lstlisting}[language=mysql]
        SET foreign_key_checks = 0
    \end{lstlisting}

    Analog schaltet man die Überprüfung folgendermaßen wieder ein:
    
    \begin{lstlisting}[language=mysql]
        SET foreign_key_checks = 1
    \end{lstlisting}

    Dieser Befehl birgt auf den Ersten Blick große Gefahren, ist jedoch vor allem bei Selbstrelationen nötig, wenn sich noch nicht existierende Entitäten gegenseitig referenzieren.
\end{sql}